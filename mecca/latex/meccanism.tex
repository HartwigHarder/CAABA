\documentclass[landscape]{article}

\makeatletter
\def\shownote#1{\expandafter\gdef\csname note#1\endcsname{}}%
\def\note#1#2{\@ifundefined{note#1}{}{#1: #2}}%
\def\showhenrynote#1{\expandafter\gdef\csname henrynote#1\endcsname{}}%
\def\henrynote#1#2{\@ifundefined{henrynote#1}{}{\kpp{#1}: #2}}%
\def\showalphanote#1{\expandafter\gdef\csname alphanote#1\endcsname{}}%
\def\alphanote#1#2{\@ifundefined{alphanote#1}{}{\kpp{#1}: #2}}%
\makeatother

\def\myhline{\ifvmode\hline\fi}
\def\aq{(aq)}

\usepackage{multicol}
\usepackage{longtable}
\usepackage{natbib}
\usepackage{chem}
\usepackage{rotating} % loads graphics
%\usepackage{color}

\textwidth26cm
\textheight16cm
\topmargin-10mm
\oddsidemargin-15mm
\parindent0mm
\parskip1.0ex plus0.5ex minus0.5ex

\begin{document}

% This file was created automatically by xmecca, DO NOT EDIT!
% xmecca was run on 2010-08-26 at 14:15:43 by caaba
\def\meccaversion{\code{2.7b}}
\def\kppversion{\code{2.2.1_rs5}}
\def\wanted{\code{Tr && G && !S && !Cl && !Br && !I && !Hg}}
\def\apn{0}
\def\gasspc{448}
\def\aqspc{0}
\def\allspc{448}
\def\Geqns{246}
\def\Aeqns{0}
\def\Heqns{0}
\def\Jeqns{72}
\def\HETeqns{0}
\def\EQeqns{0}
\def\IEXeqns{0}
\def\Deqns{0}
\def\alleqns{318}


\thispagestyle{empty}
\begin{rotate}{-90}
\begin{minipage}{15cm}
\vspace{-30cm}
\begin{center}
  \LARGE {\bf The Chemical Mechanism of MECCA}\\[3mm]
  \Large KPP version: {\kppversion}\\[2mm]
  \Large MECCA version: {\meccaversion}\\[2mm]
  \Large Date: \today.\\[2mm]
  \Large Selected reactions:\\
  ``\wanted''\\[2mm]
  Number of aerosol phases: \apn\\[2mm]
  Number of species in selected mechanism:\\
  \begin{tabular}{lr}
  Gas phase:     & \gasspc\\
  Aqueous phase: & \aqspc\\
  All species:   & \allspc\\
  \end{tabular}\\[2mm]
  Number of reactions in selected mechanism:\\
  \begin{tabular}{lr}
    Gas phase (Gnnn):       & \Geqns\\
    Aqueous phase (Annn):   & \Aeqns\\
    Henry (Hnnn):           & \Heqns\\
    Photolysis (Jnnn):      & \Jeqns\\
    Heterogeneous (HETnnn): & \HETeqns\\
    Equilibria (EQnn):      & \EQeqns\\
    Dummy (Dnn):            & \Deqns\\
    All equations:          & \alleqns
  \end{tabular}\\[30mm]
  Further information can be found in the article ``Technical Note: The
  new comprehensive atmospheric chemistry module MECCA'' by R.\ Sander
  et al. (Atmos.\ Chem.\ Phys.\ {\bf 5}, 445-450, 2005), available at
  \url{http://www.atmos-chem-phys.net/5/445}.
\end{center}
\end{minipage}
\end{rotate}
\newpage

% define latex names of kpp species
% this file was created automatically by spc2tex, do not edit!
\makeatletter
\def\defkpp#1#2{\expandafter\def\csname #1\endcsname{\chem{#2}}}%
\def\kpp#1{\@ifundefined{#1}{\errmessage{#1 undefined}}{\csname #1\endcsname}}%
\defkpp{hv}{h\nu}%
\defkpp{PROD}{products}%
\defkpp{O1D}{O(^1D)}%
\defkpp{O3P}{O(^3P)}%
\defkpp{O2}{O_2}%
\defkpp{O3}{O_3}%
\defkpp{H}{H}%
\defkpp{H2}{H_2}%
\defkpp{OH}{OH}%
\defkpp{HO2}{HO_2}%
\defkpp{H2O}{H_2O}%
\defkpp{H2O2}{H_2O_2}%
\defkpp{N}{N}%
\defkpp{N2}{N_2}%
\defkpp{NH3}{NH_3}%
\defkpp{N2O}{N_2O}%
\defkpp{NO}{NO}%
\defkpp{NO2}{NO_2}%
\defkpp{NO3}{NO_3}%
\defkpp{N2O5}{N_2O_5}%
\defkpp{HONO}{HONO}%
\defkpp{HNO3}{HNO_3}%
\defkpp{HNO4}{HNO_4}%
\defkpp{NH2}{NH_2}%
\defkpp{HNO}{HNO}%
\defkpp{NHOH}{NHOH}%
\defkpp{NH2O}{NH_2O}%
\defkpp{NH2OH}{NH_2OH}%
\defkpp{CH3O2}{CH_3O_2}%
\defkpp{CH3OH}{CH_3OH}%
\defkpp{CH3OOH}{CH_3OOH}%
\defkpp{CH4}{CH_4}%
\defkpp{CO}{CO}%
\defkpp{CO2}{CO_2}%
\defkpp{HCHO}{HCHO}%
\defkpp{HCOOH}{HCOOH}%
\defkpp{LCARBON}{LCARBON}%
\defkpp{C2H2}{C_2H_2}%
\defkpp{C2H4}{C_2H_4}%
\defkpp{C2H5O2}{C_2H_5O_2}%
\defkpp{C2H5OOH}{C_2H_5OOH}%
\defkpp{C2H6}{C_2H_6}%
\defkpp{CH3CHO}{CH_3CHO}%
\defkpp{CH3CO2H}{CH_3COOH}%
\defkpp{CH3CO3}{CH_3C(O)OO}%
\defkpp{CH3CO3H}{CH_3C(O)OOH}%
\defkpp{ETHGLY}{ETHGLY}%
\defkpp{ETHOHNO3}{ETHOHNO3}%
\defkpp{GLYOX}{GLYOX}%
\defkpp{HCOCO2H}{HCOCO_2H}%
\defkpp{HCOCO3}{HCOCO_3}%
\defkpp{HCOCO3H}{HCOCO_3H}%
\defkpp{HOCH2CH2O}{HOCH_2CH_2O}%
\defkpp{HOCH2CH2O2}{HOCH_2CH_2O_2}%
\defkpp{HOCH2CHO}{HOCH_2CHO}%
\defkpp{HOCH2CO2H}{HOCH_2CO_2H}%
\defkpp{HOCH2CO3}{HOCH_2CO_3}%
\defkpp{HOCH2CO3H}{HOCH_2CO_3H}%
\defkpp{HYETHO2H}{HYETHO2H}%
\defkpp{PAN}{PAN}%
\defkpp{PHAN}{PHAN}%
\defkpp{ACETOL}{CH_3COCH_2OH}%
\defkpp{C3H6}{C_3H_6}%
\defkpp{C3H8}{C_3H_8}%
\defkpp{CH3COCH2O2}{CH_3COCH_2O_2}%
\defkpp{CH3COCH3}{CH_3COCH_3}%
\defkpp{HOCH2COCHO}{HOCH2COCHO}%
\defkpp{HOCH2COCO2H}{HOCH2COCO2H}%
\defkpp{HYPERACET}{CH_3COCH_2O_2H}%
\defkpp{HYPROPO2}{HYPROPO2}%
\defkpp{HYPROPO2H}{HYPROPO2H}%
\defkpp{IC3H7NO3}{iC_3H_7ONO_2}%
\defkpp{IC3H7O2}{iC_3H_7O_2}%
\defkpp{IC3H7OOH}{iC_3H_7OOH}%
\defkpp{MGLYOX}{MGLYOX}%
\defkpp{NOA}{NOA}%
\defkpp{PR2O2HNO3}{PR2O2HNO3}%
\defkpp{PRONO3BO2}{PRONO3BO2}%
\defkpp{BIACET}{BIACET}%
\defkpp{BIACETOH}{BIACETOH}%
\defkpp{CO2H3CHO}{CO2H3CHO}%
\defkpp{CO2H3CO3}{CO2H3CO3}%
\defkpp{CO2H3CO3H}{CO2H3CO3H}%
\defkpp{HO12CO3C4}{HO12CO3C4}%
\defkpp{LC4H9NO3}{LC4H9NO3}%
\defkpp{LC4H9O2}{LC_4H_9O_2}%
\defkpp{LC4H9OOH}{LC_4H_9OOH}%
\defkpp{LHMVKABO2}{LHMVKABO2}%
\defkpp{LHMVKABOOH}{LHMVKABOOH}%
\defkpp{LMVKOHABO2}{LMVKOHABO2}%
\defkpp{LMVKOHABOOH}{LMVKOHABOOH}%
\defkpp{MACO2H}{MACO2H}%
\defkpp{MACO3}{MACO3}%
\defkpp{MACO3H}{MACO3H}%
\defkpp{MACR}{MACR}%
\defkpp{MACRO2}{MACRO2}%
\defkpp{MACROH}{MACROH}%
\defkpp{MACROOH}{MACROOH}%
\defkpp{MEK}{MEK}%
\defkpp{LMEKO2}{LMEKO2}%
\defkpp{LMEKOOH}{LMEKOOH}%
\defkpp{MPAN}{MPAN}%
\defkpp{MVK}{MVK}%
\defkpp{MVKOH}{MVKOH}%
\defkpp{NC4H10}{nC_4H_{10}}%
\defkpp{C59O2}{C59O2}%
\defkpp{C59OOH}{C59OOH}%
\defkpp{C5H8}{C_5H_8}%
\defkpp{HCOC5}{HCOC5}%
\defkpp{ISOPAOH}{ISOPAOH}%
\defkpp{ISOPBNO3}{ISOPBNO3}%
\defkpp{ISOPBO2}{ISOPBO2}%
\defkpp{ISOPBOH}{ISOPBOH}%
\defkpp{ISOPBOOH}{ISOPBOOH}%
\defkpp{ISOPDNO3}{ISOPDNO3}%
\defkpp{ISOPDO2}{ISOPDO2}%
\defkpp{ISOPDOH}{ISOPDOH}%
\defkpp{ISOPDOOH}{ISOPDOOH}%
\defkpp{LC578O2}{LC578O2}%
\defkpp{LC578OOH}{LC578OOH}%
\defkpp{LC5PAN1719}{LC5PAN1719}%
\defkpp{LHC4ACCHO}{LHC4ACCHO}%
\defkpp{LHC4ACCO2H}{LHC4ACCO2H}%
\defkpp{LHC4ACCO3}{LHC4ACCO3}%
\defkpp{LHC4ACCO3H}{LHC4ACCO3H}%
\defkpp{LISOPACNO3}{LISOPACNO3}%
\defkpp{LISOPACO2}{LISOPACO2}%
\defkpp{LISOPACOOH}{LISOPACOOH}%
\defkpp{LNISO3}{LNISO3}%
\defkpp{LNISOOH}{LNISOOH}%
\defkpp{NC4CHO}{NC4CHO}%
\defkpp{NISOPO2}{NISOPO2}%
\defkpp{NISOPOOH}{NISOPOOH}%
\defkpp{Cl}{Cl}%
\defkpp{Cl2}{Cl_2}%
\defkpp{ClO}{ClO}%
\defkpp{HCl}{HCl}%
\defkpp{HOCl}{HOCl}%
\defkpp{Cl2O2}{Cl_2O_2}%
\defkpp{OClO}{OClO}%
\defkpp{ClNO2}{ClNO_2}%
\defkpp{ClNO3}{ClNO_3}%
\defkpp{CCl4}{CCl_4}%
\defkpp{CH3Cl}{CH_3Cl}%
\defkpp{CH3CCl3}{CH_3CCl_3}%
\defkpp{CF2Cl2}{CF_2Cl_2}%
\defkpp{CFCl3}{CFCl_3}%
\defkpp{Br}{Br}%
\defkpp{Br2}{Br_2}%
\defkpp{BrO}{BrO}%
\defkpp{HBr}{HBr}%
\defkpp{HOBr}{HOBr}%
\defkpp{BrNO2}{BrNO_2}%
\defkpp{BrNO3}{BrNO_3}%
\defkpp{BrCl}{BrCl}%
\defkpp{CH3Br}{CH_3Br}%
\defkpp{CF3Br}{CF_3Br}%
\defkpp{CF2ClBr}{CF_2ClBr}%
\defkpp{CHCl2Br}{CHCl_2Br}%
\defkpp{CHClBr2}{CHClBr_2}%
\defkpp{CH2ClBr}{CH_2ClBr}%
\defkpp{CH2Br2}{CH_2Br_2}%
\defkpp{CHBr3}{CHBr_3}%
\defkpp{I}{I}%
\defkpp{I2}{I_2}%
\defkpp{IO}{IO}%
\defkpp{OIO}{OIO}%
\defkpp{I2O2}{I_2O_2}%
\defkpp{HI}{HI}%
\defkpp{HOI}{HOI}%
\defkpp{HIO3}{HIO_3}%
\defkpp{INO2}{INO_2}%
\defkpp{INO3}{INO_3}%
\defkpp{CH3I}{CH_3I}%
\defkpp{CH2I2}{CH_2I_2}%
\defkpp{C3H7I}{C_3H_7I}%
\defkpp{ICl}{ICl}%
\defkpp{CH2ClI}{CH_2ClI}%
\defkpp{IBr}{IBr}%
\defkpp{S}{S}%
\defkpp{SO}{SO}%
\defkpp{SO2}{SO_2}%
\defkpp{SH}{SH}%
\defkpp{H2SO4}{H_2SO_4}%
\defkpp{CH3SO3H}{CH_3SO_3H}%
\defkpp{DMS}{DMS}%
\defkpp{DMSO}{DMSO}%
\defkpp{CH3SO2}{CH_3SO_2}%
\defkpp{CH3SO3}{CH_3SO_3}%
\defkpp{OCS}{OCS}%
\defkpp{SF6}{SF_6}%
\defkpp{Hg}{Hg}%
\defkpp{HgO}{HgO}%
\defkpp{HgCl}{HgCl}%
\defkpp{HgCl2}{HgCl_2}%
\defkpp{HgBr}{HgBr}%
\defkpp{HgBr2}{HgBr_2}%
\defkpp{ClHgBr}{ClHgBr}%
\defkpp{BrHgOBr}{BrHgOBr}%
\defkpp{ClHgOBr}{ClHgOBr}%
\defkpp{NO3m_cs}{NO_3^-\aq}%
\defkpp{Hp_cs}{H^+\aq}%
\defkpp{RGM_cs}{Hg\aq}%
\defkpp{IPART}{I_{part}}%
\defkpp{Dummy}{Dummy}%
\defkpp{O3s}{O_3(s)}%
\defkpp{LO3s}{LO_3(s)}%
\defkpp{LHOC3H6O2}{CH_3CH(O_2)CH_2OH}%
\defkpp{LHOC3H6OOH}{CH_3CH(OOH)CH_2OH}%
\defkpp{ISO2}{ISO2}%
\defkpp{ISON}{ISON}%
\defkpp{ISOOH}{ISOOH}%
\defkpp{MVKO2}{MVKO2}%
\defkpp{MVKOOH}{MVKOOH}%
\defkpp{NACA}{NACA}%
\defkpp{O2_a01}{O_2\aq}%
\defkpp{O3_a01}{O_3\aq}%
\defkpp{OH_a01}{OH\aq}%
\defkpp{HO2_a01}{HO_2\aq}%
\defkpp{H2O_a01}{H_2O\aq}%
\defkpp{H2O2_a01}{H_2O_2\aq}%
\defkpp{NH3_a01}{NH_3\aq}%
\defkpp{NO_a01}{NO\aq}%
\defkpp{NO2_a01}{NO_2\aq}%
\defkpp{NO3_a01}{NO_3\aq}%
\defkpp{HONO_a01}{HONO\aq}%
\defkpp{HNO3_a01}{HNO_3\aq}%
\defkpp{HNO4_a01}{HNO_4\aq}%
\defkpp{N2O5_a01}{N_2O_5\aq}%
\defkpp{CH3OH_a01}{CH_3OH\aq}%
\defkpp{HCOOH_a01}{HCOOH\aq}%
\defkpp{HCHO_a01}{HCHO\aq}%
\defkpp{CH3O2_a01}{CH_3OO\aq}%
\defkpp{CH3OOH_a01}{CH_3OOH\aq}%
\defkpp{CO2_a01}{CO_2\aq}%
\defkpp{CH3CO2H_a01}{CH_3COOH\aq}%
\defkpp{PAN_a01}{PAN\aq}%
\defkpp{C2H5O2_a01}{C_2H_5O_2\aq}%
\defkpp{CH3CHO_a01}{CH_3CHO\aq}%
\defkpp{CH3COCH3_a01}{CH_3COCH_3\aq}%
\defkpp{Cl_a01}{Cl\aq}%
\defkpp{Cl2_a01}{Cl_2\aq}%
\defkpp{HCl_a01}{HCl\aq}%
\defkpp{HOCl_a01}{HOCl\aq}%
\defkpp{Br_a01}{Br\aq}%
\defkpp{Br2_a01}{Br_2\aq}%
\defkpp{HBr_a01}{HBr\aq}%
\defkpp{HOBr_a01}{HOBr\aq}%
\defkpp{BrCl_a01}{BrCl\aq}%
\defkpp{I2_a01}{I_2\aq}%
\defkpp{IO_a01}{IO\aq}%
\defkpp{HI_a01}{HI\aq}%
\defkpp{HOI_a01}{HOI\aq}%
\defkpp{ICl_a01}{ICl\aq}%
\defkpp{IBr_a01}{IBr\aq}%
\defkpp{HIO3_a01}{HIO_3\aq}%
\defkpp{SO2_a01}{SO_2\aq}%
\defkpp{H2SO4_a01}{H_2SO_4\aq}%
\defkpp{DMSO_a01}{DMSO\aq}%
\defkpp{Hg_a01}{Hg\aq}%
\defkpp{HgO_a01}{HgO\aq}%
\defkpp{HgOH_a01}{HgOH\aq}%
\defkpp{HgOHOH_a01}{Hg(OH)_2\aq}%
\defkpp{HgOHCl_a01}{Hg(OH)Cl\aq}%
\defkpp{HgCl2_a01}{HgCl_2\aq}%
\defkpp{HgBr2_a01}{HgBr_2\aq}%
\defkpp{HgSO3_a01}{HgSO_3\aq}%
\defkpp{ClHgBr_a01}{ClHgBr\aq}%
\defkpp{BrHgOBr_a01}{BrHgOBr\aq}%
\defkpp{ClHgOBr_a01}{ClHgOBr\aq}%
\defkpp{O2m_a01}{O_2^-\aq}%
\defkpp{OHm_a01}{OH^-\aq}%
\defkpp{Hp_a01}{H^+\aq}%
\defkpp{NH4p_a01}{NH_4^+\aq}%
\defkpp{NO2m_a01}{NO_2^-\aq}%
\defkpp{NO3m_a01}{NO_3^-\aq}%
\defkpp{NO4m_a01}{NO_4^-\aq}%
\defkpp{CO3m_a01}{CO_3^-\aq}%
\defkpp{HCOOm_a01}{HCOO^-\aq}%
\defkpp{HCO3m_a01}{HCO_3^-\aq}%
\defkpp{CH3COOm_a01}{CH_3COO^-\aq}%
\defkpp{Clm_a01}{Cl^-\aq}%
\defkpp{Cl2m_a01}{Cl_2^-\aq}%
\defkpp{ClOm_a01}{ClO^-\aq}%
\defkpp{ClOHm_a01}{ClOH^-\aq}%
\defkpp{Brm_a01}{Br^-\aq}%
\defkpp{Br2m_a01}{Br_2^-\aq}%
\defkpp{BrOm_a01}{BrO^-\aq}%
\defkpp{BrOHm_a01}{BrOH^-\aq}%
\defkpp{BrCl2m_a01}{BrCl_2^-\aq}%
\defkpp{Br2Clm_a01}{Br_2Cl^-\aq}%
\defkpp{Im_a01}{I^-\aq}%
\defkpp{IO2m_a01}{IO_2^-\aq}%
\defkpp{IO3m_a01}{IO_3^-\aq}%
\defkpp{ICl2m_a01}{ICl_2^-\aq}%
\defkpp{IClBrm_a01}{IClBr^-\aq}%
\defkpp{IBr2m_a01}{IBr_2^-\aq}%
\defkpp{SO3m_a01}{SO_3^-\aq}%
\defkpp{SO3mm_a01}{SO_3^{2-}\aq}%
\defkpp{SO4m_a01}{SO_4^-\aq}%
\defkpp{SO4mm_a01}{SO_4^{2-}\aq}%
\defkpp{SO5m_a01}{SO_5^-\aq}%
\defkpp{HSO3m_a01}{HSO_3^-\aq}%
\defkpp{HSO4m_a01}{HSO_4^-\aq}%
\defkpp{HSO5m_a01}{HSO_5^-\aq}%
\defkpp{CH3SO3m_a01}{CH_3SO_3^-\aq}%
\defkpp{CH2OHSO3m_a01}{CH_2OHSO_3^-\aq}%
\defkpp{Hgp_a01}{Hg^+\aq}%
\defkpp{Hgpp_a01}{Hg^{2+}\aq}%
\defkpp{HgOHp_a01}{HgOH^+\aq}%
\defkpp{HgClp_a01}{HgCl^+\aq}%
\defkpp{HgCl3m_a01}{HgCl_3^-\aq}%
\defkpp{HgCl4mm_a01}{HgCl_4^{2-}\aq}%
\defkpp{HgBrp_a01}{HgBr^+\aq}%
\defkpp{HgBr3m_a01}{HgBr_3^-\aq}%
\defkpp{HgBr4mm_a01}{HgBr_4^{2-}\aq}%
\defkpp{HgSO32mm_a01}{Hg(SO_3)_2^{2-}\aq}%
\defkpp{D1O_a01}{D_1O\aq}%
\defkpp{D2O_a01}{D_2O\aq}%
\defkpp{DAHp_a01}{DAH^+\aq}%
\defkpp{DA_a01}{DA\aq}%
\defkpp{DAm_a01}{DA^-\aq}%
\defkpp{DGtAi_a01}{DGtAi\aq}%
\defkpp{DGtAs_a01}{DGtAs\aq}%
\defkpp{PROD1_a01}{PROD1\aq}%
\defkpp{PROD2_a01}{PROD2\aq}%
\defkpp{Nap_a01}{Na^+\aq}%
\makeatother


\begin{longtable}{llp{9cm}p{7cm}p{5cm}}
\caption{Gas phase reactions}\\
\hline
\# & labels & reaction & rate coefficient & reference\\
\hline
\endfirsthead
\caption{Gas phase reactions (... continued)}\\
\hline
\# & labels & reaction & rate coefficient & reference\\
\hline
\endhead
\hline
\endfoot
% this file was created automatically by eqn2tex, do not edit!
\code{G1000} & StTrG &  \kpp{O2} + \kpp{O1D} $\rightarrow$ \kpp{O3P} + \kpp{O2} & \code{3.3E-11*EXP(LOG(1.1)*mcexp(24))*EXP(55./temp)} & \citet{1945}\\
\code{G1001} & StTrG &  \kpp{O2} + \kpp{O3P} $\rightarrow$ \kpp{O3} & \code{6.E-34*EXP(LOG(1.1)*mcexp(25))*((temp/300.)**(-2.4))*cair} & \citet{1945}\\
\myhline
\code{G2100} & StTrG &  \kpp{H} + \kpp{O2} $\rightarrow$ \kpp{HO2} & \code{k_3rd(temp,cair,4.4E-32,1.3,4.7E-11,0.2,0.6)*EXP(LOG(1.3)*mcexp(26))} & \citet{1945}\\
\code{G2104} & StTrG &  \kpp{OH} + \kpp{O3} $\rightarrow$ \kpp{HO2} + \kpp{O2} & \code{1.7E-12*EXP(LOG(1.2)*mcexp(27))*EXP(-940./temp)} & \citet{1945}\\
\code{G2105} & StTrG &  \kpp{OH} + \kpp{H2} $\rightarrow$ \kpp{H2O} + \kpp{H} & \code{2.8E-12*EXP(LOG(1.05)*mcexp(28))*EXP(-1800./temp)} & \citet{1945}\\
\code{G2107} & StTrG &  \kpp{HO2} + \kpp{O3} $\rightarrow$ \kpp{OH} + 2 \kpp{O2} & \code{1.E-14*EXP(LOG(1.15)*mcexp(29))*EXP(-490./temp)} & \citet{1945}\\
\code{G2109} & StTrG &  \kpp{HO2} + \kpp{OH} $\rightarrow$ \kpp{H2O} + \kpp{O2} & \code{4.8E-11*EXP(LOG(1.25)*mcexp(30))*EXP(250./temp)} & \citet{1945}\\
\code{G2110} & StTrG &  \kpp{HO2} + \kpp{HO2} $\rightarrow$ \kpp{H2O2} + \kpp{O2} & \code{k_HO2_HO2} & \citet{1599}, \citet{165}$^*$\shownote{G2110}\\
\code{G2111} & StTrG &  \kpp{H2O} + \kpp{O1D} $\rightarrow$ 2 \kpp{OH} & \code{1.63E-10*EXP(LOG(1.15)*mcexp(31))*EXP(60./temp)} & \citet{1945}\\
\code{G2112} & StTrG &  \kpp{H2O2} + \kpp{OH} $\rightarrow$ \kpp{H2O} + \kpp{HO2} & \code{1.8E-12*EXP(LOG(1.25)*mcexp(32))} & \citet{1945}\\
\myhline
\code{G3101} & StTrG &  \kpp{N2} + \kpp{O1D} $\rightarrow$ \kpp{O3P} + \kpp{N2} & \code{2.15E-11*EXP(LOG(1.10)*mcexp(33))*EXP(110./temp)} & \citet{1945}\\
\code{G3103} & StTrGN &  \kpp{NO} + \kpp{O3} $\rightarrow$ \kpp{NO2} + \kpp{O2} & \code{3.E-12*EXP(LOG(1.1)*mcexp(34))*EXP(-1500./temp)} & \citet{1945}\\
\code{G3106} & StTrGN &  \kpp{NO2} + \kpp{O3} $\rightarrow$ \kpp{NO3} + \kpp{O2} & \code{1.2E-13*EXP(LOG(1.15)*mcexp(35))*EXP(-2450./temp)} & \citet{1945}\\
\code{G3108} & StTrGN &  \kpp{NO3} + \kpp{NO} $\rightarrow$ 2 \kpp{NO2} & \code{1.5E-11*EXP(LOG(1.3)*mcexp(36))*EXP(170./temp)} & \citet{1945}\\
\code{G3109} & StTrGN &  \kpp{NO3} + \kpp{NO2} $\rightarrow$ \kpp{N2O5} & \code{k_NO3_NO2} & \citet{1945}$^*$\shownote{G3109}\\
\code{G3110} & StTrGN &  \kpp{N2O5} $\rightarrow$ \kpp{NO2} + \kpp{NO3} & \code{k_NO3_NO2/(2.7E-27*EXP(LOG(1.2)*mcexp(37))*EXP(11000./temp))} & \citet{1945}$^*$\shownote{G3110}\\
\code{G3200} & TrGN &  \kpp{NO} + \kpp{OH} $\rightarrow$ \kpp{HONO} & \code{k_3rd(temp,cair,7.0E-31,2.6,3.6E-11,0.1,0.6)*EXP(LOG(1.2)*mcexp(38))} & \citet{1945}\\
\code{G3201} & StTrGN &  \kpp{NO} + \kpp{HO2} $\rightarrow$ \kpp{NO2} + \kpp{OH} & \code{3.5E-12*EXP(250./temp)*EXP(LOG(1.15)*mcexp(39))} & \citet{1945}\\
\code{G3202} & StTrGN &  \kpp{NO2} + \kpp{OH} $\rightarrow$ \kpp{HNO3} & \code{k_3rd(temp,cair,1.8E-30,3.0,2.8E-11,0.,0.6)*EXP(LOG(1.3)*mcexp(40))} & \citet{1945}\\
\code{G3203} & StTrGN &  \kpp{NO2} + \kpp{HO2} $\rightarrow$ \kpp{HNO4} & \code{k_NO2_HO2} & \citet{1945}$^*$\shownote{G3203}\\
\code{G3204} & TrGN &  \kpp{NO3} + \kpp{HO2} $\rightarrow$ \kpp{NO2} + \kpp{OH} + \kpp{O2} & \code{3.5E-12*EXP(LOG(1.5)*mcexp(41))} & \citet{1945}\\
\code{G3205} & TrGN &  \kpp{HONO} + \kpp{OH} $\rightarrow$ \kpp{NO2} + \kpp{H2O} & \code{1.8E-11*EXP(LOG(1.5)*mcexp(42))*EXP(-390./temp)} & \citet{1945}\\
\code{G3206} & StTrGN &  \kpp{HNO3} + \kpp{OH} $\rightarrow$ \kpp{H2O} + \kpp{NO3} & \code{k_HNO3_OH} & \citet{1945}$^*$\shownote{G3206}\\
\code{G3207} & StTrGN &  \kpp{HNO4} $\rightarrow$ \kpp{NO2} + \kpp{HO2} & \code{k_NO2_HO2/(2.1E-27*EXP(LOG(1.3)*mcexp(43))*EXP(10900./temp))} & \citet{1945}$^*$\shownote{G3207}\\
\code{G3208} & StTrGN &  \kpp{HNO4} + \kpp{OH} $\rightarrow$ \kpp{NO2} + \kpp{H2O} & \code{1.3E-12*EXP(LOG(1.3)*mcexp(44))*EXP(380./temp)} & \citet{1945}\\
\code{G3209} & TrGN &  \kpp{NH3} + \kpp{OH} $\rightarrow$ \kpp{NH2} + \kpp{H2O} & \code{1.7E-12*EXP(LOG(1.25)*mcexp(45))*EXP(-710./temp)} & \citet{2415}\\
\code{G3210} & TrGN &  \kpp{NH2} + \kpp{O3} $\rightarrow$ \kpp{NH2O} + \kpp{O2} & \code{4.3E-12*EXP(LOG(1.25)*mcexp(46))*EXP(-930./temp)} & \citet{2415}\\
\code{G3211} & TrGN &  \kpp{NH2} + \kpp{HO2} $\rightarrow$ \kpp{NH2O} + \kpp{OH} & \code{4.8E-07*EXP(LOG(1.25)*mcexp(47))*EXP(-628./temp)*temp**(-1.32)} & \citet{2415}\\
\code{G3212} & TrGN &  \kpp{NH2} + \kpp{HO2} $\rightarrow$ \kpp{HNO} + \kpp{H2O} & \code{9.4E-09*EXP(LOG(1.25)*mcexp(48))*EXP(-356./temp)*temp**(-1.12)} & \citet{2415}\\
\code{G3213} & TrGN &  \kpp{NH2} + \kpp{NO} $\rightarrow$ \kpp{HO2} + \kpp{OH} + \kpp{N2} & \code{1.92E-12*EXP(LOG(1.25)*mcexp(49))*((temp/298.)**(-1.5))} & \citet{2415}\\
\code{G3214} & TrGN &  \kpp{NH2} + \kpp{NO} $\rightarrow$ \kpp{N2} + \kpp{H2O} & \code{1.41E-11*EXP(LOG(1.25)*mcexp(50))*((temp/298.)**(-1.5))} & \citet{2415}\\
\code{G3215} & TrGN &  \kpp{NH2} + \kpp{NO2} $\rightarrow$ \kpp{N2O} + \kpp{H2O} & \code{1.2E-11*EXP(LOG(1.25)*mcexp(51))*((temp/298.)**(-2.0))} & \citet{2415}\\
\code{G3216} & TrGN &  \kpp{NH2} + \kpp{NO2} $\rightarrow$ \kpp{NH2O} + \kpp{NO} & \code{0.8E-11*EXP(LOG(1.25)*mcexp(52))*((temp/298.)**(-2.0))} & \citet{2415}\\
\code{G3217} & TrGN &  \kpp{NH2O} + \kpp{O3} $\rightarrow$ \kpp{NH2} + \kpp{O2} & \code{1.2E-14*EXP(LOG(1.25)*mcexp(53))} & \citet{2415}\\
\code{G3218} & TrGN &  \kpp{NH2O} $\rightarrow$ \kpp{NHOH} & \code{1.3E3*EXP(LOG(1.25)*mcexp(54))} & \citet{2415}\\
\code{G3219} & TrGN &  \kpp{HNO} + \kpp{OH} $\rightarrow$ \kpp{NO} + \kpp{H2O} & \code{8.0E-11*EXP(LOG(1.25)*mcexp(55))*EXP(-500./temp)} & \citet{2415}\\
\code{G3220} & TrGN &  \kpp{HNO} + \kpp{NHOH} $\rightarrow$ \kpp{NH2OH} + \kpp{NO} & \code{1.66E-12*EXP(LOG(1.25)*mcexp(56))*EXP(-1500./temp)} & \citet{2415}\\
\code{G3221} & TrGN &  \kpp{HNO} + \kpp{NO2} $\rightarrow$ \kpp{HONO} + \kpp{NO} & \code{1.0E-12*EXP(LOG(1.25)*mcexp(57))*EXP(-1000./temp)} & \citet{2415}\\
\code{G3222} & TrGN &  \kpp{NHOH} + \kpp{OH} $\rightarrow$ \kpp{HNO} + \kpp{H2O} & \code{1.66E-12*EXP(LOG(1.25)*mcexp(58))} & \citet{2415}\\
\code{G3223} & TrGN &  \kpp{NH2OH} + \kpp{OH} $\rightarrow$ \kpp{NHOH} + \kpp{H2O} & \code{4.13E-11*EXP(LOG(1.25)*mcexp(59))*EXP(-2138./temp)} & \citet{2415}\\
\code{G3224} & TrGN &  \kpp{HNO} + \kpp{O2} $\rightarrow$ \kpp{HO2} + \kpp{NO} & \code{3.65E-14*EXP(LOG(1.25)*mcexp(60))*EXP(-4600./temp)} & \citet{2415}\\
\myhline
\code{G4101} & StTrG &  \kpp{CH4} + \kpp{OH} $\rightarrow$ \kpp{CH3O2} + \kpp{H2O} & \code{1.85E-20*EXP(LOG(1.2)*mcexp(61))*EXP(2.82*log(temp)-987./temp)} & \citet{1627}\\
\code{G4102} & TrG &  \kpp{CH3OH} + \kpp{OH} $\rightarrow$ \kpp{HCHO} + \kpp{HO2} & \code{2.9E-12*EXP(LOG(1.10)*mcexp(62))*EXP(-345./temp)} & \citet{1945}\\
\code{G4103} & StTrG &  \kpp{CH3O2} + \kpp{HO2} $\rightarrow$ \kpp{CH3OOH} + \kpp{O2} & \code{4.1E-13*EXP(LOG(1.3)*mcexp(63))*EXP(750./temp)} & \citet{1945}$^*$\shownote{G4103}\\
\code{G4104} & StTrGN &  \kpp{CH3O2} + \kpp{NO} $\rightarrow$ \kpp{HCHO} + \kpp{NO2} + \kpp{HO2} & \code{2.8E-12*EXP(LOG(1.15)*mcexp(64))*EXP(300./temp)} & \citet{1945}\\
\code{G4105} & TrGN &  \kpp{CH3O2} + \kpp{NO3} $\rightarrow$ \kpp{HCHO} + \kpp{HO2} + \kpp{NO2} & \code{1.3E-12*EXP(0.3*LOG(10.)*mcexp(65))} & \citet{1759}\\
\code{G4106a} & StTrG &  \kpp{CH3O2} $\rightarrow$ \kpp{HCHO} + \kpp{HO2} & \code{2.*RO2*9.5E-14*EXP(LOG(1.2)*mcexp(66))*EXP(390./temp)/(1.+1./26.2*EXP(LOG(1.25)*mcexp(67))*EXP(1130./temp))} & \citet{1945}\\
\code{G4106b} & StTrG &  \kpp{CH3O2} $\rightarrow$ .5 \kpp{HCHO} + .5 \kpp{CH3OH} + .5 \kpp{O2} & \code{2.*RO2*9.5E-14*EXP(LOG(1.2)*mcexp(68))*EXP(390./temp)/(1.+26.2*EXP(LOG(1.25)*mcexp(69))*EXP(-1130./temp))} & \citet{1945}\\
\code{G4107} & StTrG &  \kpp{CH3OOH} + \kpp{OH} $\rightarrow$ .7 \kpp{CH3O2} + .3 \kpp{HCHO} + .3 \kpp{OH} + \kpp{H2O} & \code{k_CH3OOH_OH} & \citet{1945}$^*$\shownote{G4107}\\
\code{G4108} & StTrG &  \kpp{HCHO} + \kpp{OH} $\rightarrow$ \kpp{CO} + \kpp{H2O} + \kpp{HO2} & \code{9.52E-18*EXP(LOG(1.05)*mcexp(70))*EXP(2.03*log(temp)+636./temp)} & \citet{1634}\\
\code{G4109} & TrGN &  \kpp{HCHO} + \kpp{NO3} $\rightarrow$ \kpp{HNO3} + \kpp{CO} + \kpp{HO2} & \code{3.4E-13*EXP(LOG(1.3)*mcexp(71))*EXP(-1900./temp)} & \citet{1945}$^*$\shownote{G4109}\\
\code{G4110} & StTrG &  \kpp{CO} + \kpp{OH} $\rightarrow$ \kpp{H} + \kpp{CO2} & \code{(1.57E-13+cair*3.54E-33)*EXP(LOG(1.15)*mcexp(72))} & \citet{1628}\\
\code{G4111} & TrG &  \kpp{HCOOH} + \kpp{OH} $\rightarrow$ \kpp{CO2} + \kpp{HO2} + \kpp{H2O} & \code{4.0E-13*EXP(LOG(1.2)*mcexp(73))} & \citet{1945}\\
\code{G4200} & TrGC &  \kpp{C2H6} + \kpp{OH} $\rightarrow$ \kpp{C2H5O2} + \kpp{H2O} & \code{1.49E-17*EXP(LOG(1.25)*mcexp(74))*temp*temp*EXP(-499./temp)} & \citet{1627}\\
\code{G4201} & TrGC &  \kpp{C2H4} + \kpp{O3} $\rightarrow$ \kpp{HCHO} + .63 \kpp{CO} + .13 \kpp{HO2} + 0.23125 \kpp{HCOOH} + 0.13875 \kpp{HCHO} + 0.13875 \kpp{H2O2} + .13 \kpp{OH} & \code{1.2E-14*EXP(LOG(1.25)*mcexp(75))*EXP(-2630./temp)} & \citet{1945}$^*$\shownote{G4201}\\
\code{G4202} & TrGC &  \kpp{C2H4} + \kpp{OH} $\rightarrow$ \kpp{HOCH2CH2O2} & \code{k_3rd(temp,cair,1.0E-28,4.5,8.8E-12,0.85,0.6)*EXP(LOG(1.25)*mcexp(76))} & \citet{1945}\\
\code{G4203} & TrGC &  \kpp{C2H5O2} + \kpp{HO2} $\rightarrow$ \kpp{C2H5OOH} & \code{7.5E-13*EXP(LOG(1.25)*mcexp(77))*EXP(700./temp)} & \citet{1945}\\
\code{G4204} & TrGNC &  \kpp{C2H5O2} + \kpp{NO} $\rightarrow$ \kpp{CH3CHO} + \kpp{HO2} + \kpp{NO2} & \code{2.6E-12*EXP(LOG(1.25)*mcexp(78))*EXP(365./temp)} & \citet{1945}\\
\code{G4205} & TrGNC &  \kpp{C2H5O2} + \kpp{NO3} $\rightarrow$ \kpp{CH3CHO} + \kpp{HO2} + \kpp{NO2} & \code{2.3E-12*EXP(LOG(1.25)*mcexp(79))} & \citet{1207}\\
\code{G4206} & TrGC &  \kpp{C2H5O2} $\rightarrow$ .98 \kpp{CH3CHO} + .38 \kpp{HO2} + .02 \kpp{HOCH2CH2O2} & \code{3.1E-13*EXP(LOG(1.25)*mcexp(80))*RO2} & \citet{2419}$^*$\shownote{G4206}\\
\code{G4207} & TrGC &  \kpp{C2H5OOH} + \kpp{OH} $\rightarrow$ .43 \kpp{C2H5O2} + .43 \kpp{H2O} + .57 \kpp{CH3CHO} + .57 \kpp{OH} & \code{0.6*k_CH3OOH_OH + 8.01E-12} & see note\shownote{G4207}\\
\code{G4208} & TrGC &  \kpp{CH3CHO} + \kpp{OH} $\rightarrow$ \kpp{CH3CO3} + \kpp{H2O} & \code{4.4E-12*EXP(0.08*LOG(10.)*mcexp(81))*EXP(365./temp)} & \citet{1759}\\
\code{G4209} & TrGNC &  \kpp{CH3CHO} + \kpp{NO3} $\rightarrow$ \kpp{CH3CO3} + \kpp{HNO3} & \code{KNO3AL} & \citet{1945}\\
\code{G4210} & TrGC &  \kpp{CH3CO2H} + \kpp{OH} $\rightarrow$ \kpp{CH3O2} + \kpp{CO2} + \kpp{H2O} & \code{4.2E-14*EXP(0.15*LOG(10.)*mcexp(82))*EXP(855./temp)} & \citet{1759}\\
\code{G4211a} & TrGC &  \kpp{CH3CO3} + \kpp{HO2} $\rightarrow$ \kpp{CH3CO3H} & \code{4.3E-13*EXP(LOG(1.25)*mcexp(83))*EXP(1040./temp)/(1.+1./37.*EXP(660./temp))} & \citet{1613}\\
\code{G4211b} & TrGC &  \kpp{CH3CO3} + \kpp{HO2} $\rightarrow$ \kpp{CH3CO2H} + \kpp{O3} & \code{4.3E-13*EXP(LOG(1.25)*mcexp(84))*EXP(1040./temp)/(1.+37.*EXP(-660./temp))} & \citet{1613}\\
\code{G4212} & TrGNC &  \kpp{CH3CO3} + \kpp{NO} $\rightarrow$ \kpp{CH3O2} + \kpp{CO2} + \kpp{NO2} & \code{8.1E-12*EXP(LOG(1.25)*mcexp(85))*EXP(270./temp)} & \citet{1613}\\
\code{G4213} & TrGNC &  \kpp{CH3CO3} + \kpp{NO2} $\rightarrow$ \kpp{PAN} & \code{k_CH3CO3_NO2} & \citet{1945}\\
\code{G4214} & TrGNC &  \kpp{CH3CO3} + \kpp{NO3} $\rightarrow$ \kpp{CH3O2} + \kpp{NO2} + \kpp{CO2} & \code{4.E-12*EXP(LOG(1.25)*mcexp(86))} & \citet{1617}\\
\code{G4217} & TrGC &  \kpp{CH3CO3} $\rightarrow$ .7 \kpp{CH3O2} + .7 \kpp{CO2} + .3 \kpp{CH3CO2H} & \code{1.00E-11*EXP(LOG(1.25)*mcexp(87))*RO2} & \citet{2419}\\
\code{G4218} & TrGC &  \kpp{CH3CO3H} + \kpp{OH} $\rightarrow$ \kpp{CH3CO3} + \kpp{H2O} & \code{0.6*k_CH3OOH_OH} & \citet{2419}$^*$\shownote{G4218}\\
\code{G4220} & TrGNC &  \kpp{PAN} + \kpp{OH} $\rightarrow$ \kpp{HCHO} + \kpp{CO} + \kpp{NO2} + \kpp{H2O} & \code{9.50E-13*EXP(LOG(1.25)*mcexp(88))*EXP(-650./temp)} & \citet{2419}\\
\code{G4221} & TrGNC &  \kpp{PAN} $\rightarrow$ \kpp{CH3CO3} + \kpp{NO2} & \code{k_PAN_M} & \citet{1945}$^*$\shownote{G4221}\\
\code{G4222} & TrGC &  \kpp{C2H2} + \kpp{OH} $\rightarrow$ 0.636 \kpp{GLYOX} + 0.636 \kpp{OH} + 0.364 \kpp{HCOOH} + 0.364 \kpp{CO} + 0.364 \kpp{HO2} & \code{k_3rd(temp,cair,5.5e-30,0.0,8.3e-13,-2.,0.6)*EXP(LOG(1.25)*mcexp(89))} & \citet{1945}\\
\code{G4223} & TrGC &  \kpp{HOCH2CHO} + \kpp{OH} $\rightarrow$ .8 \kpp{HOCH2CO3} + .2 \kpp{GLYOX} + .2 \kpp{HO2} + \kpp{H2O} & \code{1.00E-11*EXP(LOG(1.25)*mcexp(90))} & \citet{2419}\\
\code{G4224} & TrGNC &  \kpp{HOCH2CHO} + \kpp{NO3} $\rightarrow$ \kpp{HOCH2CO3} + \kpp{HNO3} & \code{KNO3AL} & \citet{2419}\\
\code{G4225} & TrGC &  \kpp{HOCH2CO3} $\rightarrow$ .7 \kpp{HCHO} + .7 \kpp{CO2} + .7 \kpp{HO2} + .3 \kpp{HOCH2CO2H} & \code{1.00E-11*EXP(LOG(1.25)*mcexp(91))*RO2} & \citet{2419}\\
\code{G4226} & TrGC &  \kpp{HOCH2CO3} + \kpp{HO2} $\rightarrow$ .71 \kpp{HOCH2CO3H} + .29 \kpp{HOCH2CO2H} + .29 \kpp{O3} & \code{KAPHO2} & \citet{2419}\\
\code{G4227} & TrGNC &  \kpp{HOCH2CO3} + \kpp{NO} $\rightarrow$ \kpp{NO2} + \kpp{HO2} + \kpp{HCHO} + \kpp{CO2} & \code{KAPNO} & \citet{2419}\\
\code{G4228} & TrGNC &  \kpp{HOCH2CO3} + \kpp{NO2} $\rightarrow$ \kpp{PHAN} & \code{k_CH3CO3_NO2} & \citet{2419}\\
\code{G4229} & TrGNC &  \kpp{HOCH2CO3} + \kpp{NO3} $\rightarrow$ \kpp{NO2} + \kpp{HO2} + \kpp{HCHO} + \kpp{CO2} & \code{KRO2NO3*1.60} & \citet{2419}\\
\code{G4230} & TrGC &  \kpp{HOCH2CO2H} + \kpp{OH} $\rightarrow$ \kpp{HCHO} + \kpp{HO2} + \kpp{CO2} + \kpp{H2O} & \code{2.73E-12*EXP(LOG(1.25)*mcexp(92))} & \citet{2419}\\
\code{G4231} & TrGC &  \kpp{HOCH2CO3H} + \kpp{OH} $\rightarrow$ \kpp{HOCH2CO3} + \kpp{H2O} & \code{6.19E-12*EXP(LOG(1.25)*mcexp(93))} & \citet{2419}\\
\code{G4232} & TrGNC &  \kpp{PHAN} $\rightarrow$ \kpp{HOCH2CO3} + \kpp{NO2} & \code{k_PAN_M} & \citet{2419}\\
\code{G4233} & TrGNC &  \kpp{PHAN} + \kpp{OH} $\rightarrow$ \kpp{HCHO} + \kpp{CO} + \kpp{NO2} + \kpp{H2O} & \code{1.12E-12*EXP(LOG(1.25)*mcexp(94))} & \citet{2419}\\
\code{G4234} & TrGC &  \kpp{GLYOX} + \kpp{OH} $\rightarrow$ 1.2 \kpp{CO} + .6 \kpp{HO2} + .4 \kpp{HCOCO3} + \kpp{H2O} & \code{1.14E-11*EXP(LOG(1.25)*mcexp(95))} & \citet{2419}\\
\code{G4235} & TrGNC &  \kpp{GLYOX} + \kpp{NO3} $\rightarrow$ 1.2 \kpp{CO} + .6 \kpp{HO2} + .4 \kpp{HCOCO3} + \kpp{HNO3} & \code{KNO3AL} & \citet{2419}\\
\code{G4236} & TrGC &  \kpp{HCOCO3} $\rightarrow$ .7 \kpp{CO} + .7 \kpp{HO2} + .7 \kpp{CO2} + .3 \kpp{HCOCO2H} & \code{1.00E-11*EXP(LOG(1.25)*mcexp(96))*RO2} & \citet{2419}\\
\code{G4237} & TrGC &  \kpp{HCOCO3} + \kpp{HO2} $\rightarrow$ .71 \kpp{HCOCO3H} + .29 \kpp{HCOCO2H} + .29 \kpp{O3} & \code{KAPHO2} & \citet{2419}\\
\code{G4238} & TrGNC &  \kpp{HCOCO3} + \kpp{NO} $\rightarrow$ \kpp{HO2} + \kpp{CO} + \kpp{NO2} + \kpp{CO2} & \code{KAPNO} & \citet{2419}\\
\code{G4239} & TrGNC &  \kpp{HCOCO3} + \kpp{NO3} $\rightarrow$ \kpp{HO2} + \kpp{CO} + \kpp{NO2} + \kpp{CO2} & \code{KRO2NO3*1.60} & \citet{2419}\\
\code{G4240} & TrGC &  \kpp{HCOCO2H} + \kpp{OH} $\rightarrow$ \kpp{CO} + \kpp{HO2} + \kpp{CO2} + \kpp{H2O} & \code{1.23E-11*EXP(LOG(1.25)*mcexp(97))} & \citet{2419}\\
\code{G4241} & TrGC &  \kpp{HCOCO3H} + \kpp{OH} $\rightarrow$ \kpp{HCOCO3} + \kpp{H2O} & \code{1.58E-11*EXP(LOG(1.25)*mcexp(98))} & \citet{2419}\\
\code{G4242} & TrGC &  \kpp{HOCH2CH2O2} $\rightarrow$ .6 \kpp{HOCH2CH2O} + .2 \kpp{HOCH2CHO} + .2 \kpp{ETHGLY} & \code{2.00E-12*EXP(LOG(1.25)*mcexp(99))*RO2} & \citet{2419}\\
\code{G4243} & TrGNC &  \kpp{HOCH2CH2O2} + \kpp{NO} $\rightarrow$ .24875 \kpp{HO2} + .4975 \kpp{HCHO} + .74625 \kpp{HOCH2CH2O} + .995 \kpp{NO2} + .005 \kpp{ETHOHNO3} & \code{KRO2NO} & \citet{2419}$^*$\shownote{G4243}\\
\code{G4244} & TrGC &  \kpp{HOCH2CH2O2} + \kpp{HO2} $\rightarrow$ \kpp{HYETHO2H} & \code{2.00E-13*EXP(LOG(1.25)*mcexp(100))*EXP(1250./temp)} & \citet{2419}\\
\code{G4245} & TrGNC &  \kpp{ETHOHNO3} + \kpp{OH} $\rightarrow$ \kpp{HOCH2CHO} + \kpp{NO2} + \kpp{H2O} & \code{8.40E-13*EXP(LOG(1.25)*mcexp(101))} & \citet{2419}\\
\code{G4246a} & TrGC &  \kpp{HYETHO2H} + \kpp{OH} $\rightarrow$ \kpp{HOCH2CH2O2} + \kpp{H2O} & \code{0.6*k_CH3OOH_OH} & \citet{2419}$^*$\shownote{G4246}\\
\code{G4246b} & TrGC &  \kpp{HYETHO2H} + \kpp{OH} $\rightarrow$ \kpp{HOCH2CHO} + \kpp{OH} + \kpp{H2O} & \code{1.38E-11*EXP(LOG(1.25)*mcexp(102))} & \citet{2419}\\
\code{G4247a} & TrGC &  \kpp{HOCH2CH2O} $\rightarrow$ \kpp{HO2} + \kpp{HOCH2CHO} & \code{6.00E-14*EXP(LOG(1.25)*mcexp(103))*EXP(-550./temp)*C(ind_O2)} & \citet{2419}\\
\code{G4247b} & TrGC &  \kpp{HOCH2CH2O} $\rightarrow$ \kpp{HO2} + \kpp{HCHO} + \kpp{HCHO} & \code{9.50E13*EXP(LOG(1.25)*mcexp(104))*EXP(-5988./temp)} & \citet{2419}\\
\code{G4248} & TrGC &  \kpp{ETHGLY} + \kpp{OH} $\rightarrow$ \kpp{HOCH2CHO} + \kpp{HO2} + \kpp{H2O} & \code{7.70E-12*EXP(LOG(1.25)*mcexp(105))} & \citet{2419}\\
\code{G4300} & TrGC &  \kpp{C3H8} + \kpp{OH} $\rightarrow$ .736 \kpp{IC3H7O2} + .264 \kpp{C2H5O2} + .264 \kpp{CO2} + .264 \kpp{HO2} + \kpp{H2O} & \code{1.55E-17*EXP(LOG(1.25)*mcexp(106))*temp*temp*EXP(-61./temp)} & \citet{2419}$^*$\shownote{G4300}\\
\code{G4301} & TrGC &  \kpp{C3H6} + \kpp{O3} $\rightarrow$ .28 \kpp{CH3O2} + .1 \kpp{CH4} + .075 \kpp{CH3CO2H} + .56 \kpp{CO} + .075 \kpp{HCOOH} + .09 \kpp{H2O2} + .28 \kpp{HO2} + .2 \kpp{CO2} + .545 \kpp{CH3CHO} + .545 \kpp{HCHO} + .36 \kpp{OH} & \code{6.5E-15*EXP(LOG(1.25)*mcexp(107))*EXP(-1900./temp)} & \citet{1945}$^*$\shownote{G4301}\\
\code{G4302} & TrGC &  \kpp{C3H6} + \kpp{OH} $\rightarrow$ \kpp{HYPROPO2} & \code{k_3rd(temp,cair,8.E-27,3.5,3.E-11,0.,0.5)*EXP(LOG(1.25)*mcexp(108))} & \citet{1207}\\
\code{G4303} & TrGNC &  \kpp{C3H6} + \kpp{NO3} $\rightarrow$ \kpp{PRONO3BO2} & \code{4.6E-13*EXP(LOG(1.25)*mcexp(109))*EXP(-1155./temp)} & \citet{1207}\\
\code{G4304} & TrGC &  \kpp{IC3H7O2} + \kpp{HO2} $\rightarrow$ \kpp{IC3H7OOH} & \code{1.9E-13*EXP(LOG(1.25)*mcexp(110))*EXP(1300./temp)} & \citet{964}$^*$\shownote{G4304}\\
\code{G4305} & TrGNC &  \kpp{IC3H7O2} + \kpp{NO} $\rightarrow$ .96 \kpp{CH3COCH3} + .96 \kpp{HO2} + .96 \kpp{NO2} + .04 \kpp{IC3H7NO3} & \code{2.7E-12*EXP(LOG(1.25)*mcexp(111))*EXP(360./temp)} & \citet{1207}\\
\code{G4306} & TrGC &  \kpp{IC3H7O2} $\rightarrow$ \kpp{CH3COCH3} + .8 \kpp{HO2} & \code{4.E-14*EXP(LOG(1.25)*mcexp(112))*RO2} & \citet{2419}$^*$\shownote{G4306}\\
\code{G4307} & TrGC &  \kpp{IC3H7OOH} + \kpp{OH} $\rightarrow$ .27 \kpp{IC3H7O2} + .73 \kpp{CH3COCH3} + .73 \kpp{OH} + \kpp{H2O} & \code{1.66E-11*EXP(LOG(1.25)*mcexp(113)) + 0.6*k_CH3OOH_OH} & \citet{2419}$^*$\shownote{G4307}\\
\code{G4311} & TrGC &  \kpp{CH3COCH3} + \kpp{OH} $\rightarrow$ \kpp{CH3COCH2O2} + \kpp{H2O} & \code{(1.33E-13+3.82E-11*EXP(-2000./temp))*EXP(LOG(1.25)*mcexp(114))} & \citet{1945}\\
\code{G4312} & TrGC &  \kpp{CH3COCH2O2} + \kpp{HO2} $\rightarrow$ \kpp{HYPERACET} & \code{8.6E-13*EXP(LOG(1.25)*mcexp(115))*EXP(700./temp)} & \citet{1613}\\
\code{G4313} & TrGNC &  \kpp{CH3COCH2O2} + \kpp{NO} $\rightarrow$ \kpp{CH3CO3} + \kpp{HCHO} + \kpp{NO2} & \code{2.9E-12*EXP(LOG(1.25)*mcexp(116))*EXP(300./temp)} & \citet{1945}\\
\code{G4314} & TrGC &  \kpp{CH3COCH2O2} $\rightarrow$ .6 \kpp{CH3CO3} + .6 \kpp{HCHO} + .2 \kpp{MGLYOX} + .2 \kpp{ACETOL} & \code{7.5E-13*EXP(LOG(1.25)*mcexp(117))*EXP(500./temp)*2.*RO2} & \citet{1613}\\
\code{G4315a} & TrGC &  \kpp{HYPERACET} + \kpp{OH} $\rightarrow$ \kpp{CH3COCH2O2} + \kpp{H2O} & \code{0.6*k_CH3OOH_OH} & see note\shownote{G4315}\\
\code{G4315b} & TrGC &  \kpp{HYPERACET} + \kpp{OH} $\rightarrow$ \kpp{MGLYOX} + \kpp{OH} + \kpp{H2O} & \code{8.39E-12*EXP(LOG(1.25)*mcexp(118))} & \citet{2419}\\
\code{G4316} & TrGC &  \kpp{ACETOL} + \kpp{OH} $\rightarrow$ \kpp{MGLYOX} + \kpp{HO2} + \kpp{H2O} & \code{3.E-12*EXP(LOG(1.25)*mcexp(119))} & \citet{1207}\\
\code{G4317} & TrGC &  \kpp{MGLYOX} + \kpp{OH} $\rightarrow$ \kpp{CH3CO3} + \kpp{CO} & \code{8.4E-13*EXP(LOG(1.25)*mcexp(120))*EXP(830./temp)} & \citet{1616}\\
\code{G4320} & TrGNC &  \kpp{IC3H7NO3} + \kpp{OH} $\rightarrow$ \kpp{CH3COCH3} + \kpp{NO2} & \code{6.2E-13*EXP(LOG(1.25)*mcexp(121))*EXP(-230./temp)} & \citet{1207}\\
\code{G4321} & TrGNC &  \kpp{CH3COCH2O2} + \kpp{NO3} $\rightarrow$ \kpp{CH3CO3} + \kpp{HCHO} + \kpp{NO2} & \code{KRO2NO3} & \citet{2419}\\
\code{G4322} & TrGC &  \kpp{HYPROPO2} $\rightarrow$ \kpp{CH3CHO} + \kpp{HCHO} + \kpp{HO2} & \code{8.80E-13*EXP(LOG(1.25)*mcexp(122))*RO2} & \citet{2419}\\
\code{G4323} & TrGC &  \kpp{HYPROPO2} + \kpp{HO2} $\rightarrow$ \kpp{HYPROPO2H} & \code{KRO2HO2*0.520} & \citet{2419}\\
\code{G4324} & TrGNC &  \kpp{HYPROPO2} + \kpp{NO} $\rightarrow$ \kpp{CH3CHO} + \kpp{HCHO} + \kpp{HO2} + \kpp{NO2} & \code{KRO2NO} & \citet{2419}\\
\code{G4325} & TrGNC &  \kpp{HYPROPO2} + \kpp{NO3} $\rightarrow$ \kpp{CH3CHO} + \kpp{HCHO} + \kpp{HO2} + \kpp{NO2} & \code{KRO2NO3} & \citet{2419}\\
\code{G4326a} & TrGC &  \kpp{HYPROPO2H} + \kpp{OH} $\rightarrow$ \kpp{HYPROPO2} & \code{1.90E-12*EXP(LOG(1.25)*mcexp(123))*EXP(190./temp)} & \citet{2419}\\
\code{G4326b} & TrGC &  \kpp{HYPROPO2H} + \kpp{OH} $\rightarrow$ \kpp{ACETOL} + \kpp{OH} & \code{2.44E-11*EXP(LOG(1.25)*mcexp(124))} & \citet{2419}\\
\code{G4327} & TrGNC &  \kpp{PRONO3BO2} + \kpp{HO2} $\rightarrow$ \kpp{PR2O2HNO3} & \code{KRO2HO2*0.520} & \citet{2419}\\
\code{G4328} & TrGNC &  \kpp{PRONO3BO2} + \kpp{NO} $\rightarrow$ \kpp{NOA} + \kpp{HO2} + \kpp{NO2} & \code{KRO2NO} & \citet{2419}\\
\code{G4329} & TrGNC &  \kpp{PRONO3BO2} + \kpp{NO3} $\rightarrow$ \kpp{NOA} + \kpp{HO2} + \kpp{NO2} & \code{KRO2NO3} & \citet{2419}\\
\code{G4330a} & TrGNC &  \kpp{PR2O2HNO3} + \kpp{OH} $\rightarrow$ \kpp{PRONO3BO2} & \code{1.90E-12*EXP(LOG(1.25)*mcexp(125))*EXP(190./temp)} & \citet{2419}\\
\code{G4330b} & TrGNC &  \kpp{PR2O2HNO3} + \kpp{OH} $\rightarrow$ \kpp{NOA} + \kpp{OH} & \code{3.47E-12*EXP(LOG(1.25)*mcexp(126))} & \citet{2419}\\
\code{G4331} & TrGNC &  \kpp{MGLYOX} + \kpp{NO3} $\rightarrow$ \kpp{CH3CO3} + \kpp{CO} + \kpp{HNO3} & \code{KNO3AL*2.4} & \citet{2419}\\
\code{G4332} & TrGNC &  \kpp{NOA} + \kpp{OH} $\rightarrow$ \kpp{MGLYOX} + \kpp{NO2} & \code{1.30E-13*EXP(LOG(1.25)*mcexp(127))} & \citet{2419}\\
\code{G4333} & TrGC &  \kpp{HOCH2COCHO} + \kpp{OH} $\rightarrow$ \kpp{HOCH2CO3} + \kpp{CO} & \code{1.44E-11*EXP(LOG(1.25)*mcexp(128))} & \citet{2419}\\
\code{G4334} & TrGNC &  \kpp{HOCH2COCHO} + \kpp{NO3} $\rightarrow$ \kpp{HOCH2CO3} + \kpp{CO} + \kpp{HNO3} & \code{KNO3AL*2.4} & \citet{2419}\\
\code{G4335} & TrGC &  \kpp{HOCH2COCO2H} + \kpp{OH} $\rightarrow$ \kpp{HOCH2CO3} + \kpp{CO2} & \code{2.89E-12*EXP(LOG(1.25)*mcexp(129))} & \citet{2419}\\
\code{G4400} & TrGC &  \kpp{NC4H10} + \kpp{OH} $\rightarrow$ \kpp{LC4H9O2} + \kpp{H2O} & \code{1.81E-17*EXP(LOG(1.25)*mcexp(130))*temp*temp*EXP(114./temp)} & \citet{1627}$^*$\shownote{G4400}\\
\code{G4401} & TrGC &  \kpp{LC4H9O2} $\rightarrow$ 0.254 \kpp{CO2} + 0.5552 \kpp{MEK} + 0.5552 \kpp{HO2} + 0.3178 \kpp{CH3CHO} + 0.4448 \kpp{C2H5O2} & \code{2.5E-13*EXP(LOG(1.25)*mcexp(131))*RO2} & \citet{2419}$^*$\shownote{G4401}\\
\code{G4402} & TrGC &  \kpp{LC4H9O2} + \kpp{HO2} $\rightarrow$ \kpp{LC4H9OOH} & \code{KRO2HO2*0.625} & \citet{2419}\\
\code{G4403} & TrGNC &  \kpp{LC4H9O2} + \kpp{NO} $\rightarrow$ 0.9172 \kpp{NO2} + 0.233 \kpp{CO2} + 0.5092 \kpp{MEK} + 0.5092 \kpp{HO2} + 0.2915 \kpp{CH3CHO} + 0.408 \kpp{C2H5O2} + 0.0828 \kpp{LC4H9NO3} & \code{KRO2NO} & \citet{2419}$^*$\shownote{G4403}\\
\code{G4404} & TrGC &  \kpp{LC4H9OOH} + \kpp{OH} $\rightarrow$ 0.2285796 \kpp{LC4H9O2} + 0.7117253 \kpp{MEK} + 0.1193902 \kpp{CO2} + 0.0596951 \kpp{C2H5O2} + 0.7714204 \kpp{OH} + \kpp{H2O} & \code{2.636E-11*EXP(LOG(1.25)*mcexp(132))} & \citet{2419}$^*$\shownote{G4404}\\
\code{G4405} & TrGC &  \kpp{MVK} + \kpp{O3} $\rightarrow$ 0.28 \kpp{CH3CO3} + 0.56 \kpp{CO} + 0.225 \kpp{LCARBON} + 0.075 \kpp{HCOOH} + 0.09 \kpp{H2O2} + 0.28 \kpp{HO2} + 0.1 \kpp{CO2} + 0.1 \kpp{CH3CHO} + 0.645 \kpp{HCHO} + 0.36 \kpp{OH} + 0.545 \kpp{MGLYOX} & \code{7.51E-16*EXP(LOG(1.25)*mcexp(133))*EXP(-1521./temp)} & \citet{2419}\\
\code{G4406} & TrGC &  \kpp{MVK} + \kpp{OH} $\rightarrow$ \kpp{LHMVKABO2} & \code{4.13E-12*EXP(LOG(1.25)*mcexp(134))*EXP(452./temp)} & \citet{2419}\\
\code{G4413} & TrGC &  \kpp{MEK} + \kpp{OH} $\rightarrow$ \kpp{LMEKO2} + \kpp{H2O} & \code{3.24E-18*EXP(LOG(1.25)*mcexp(135))*temp*temp*EXP(414./temp)} & \citet{2419}$^*$\shownote{G4413}\\
\code{G4414} & TrGC &  \kpp{LMEKO2} + \kpp{HO2} $\rightarrow$ \kpp{LMEKOOH} & \code{KRO2HO2*0.625} & \citet{2419}\\
\code{G4415} & TrGNC &  \kpp{LMEKO2} + \kpp{NO} $\rightarrow$ 0.538 \kpp{HCHO} + 0.538 \kpp{CO2} + 0.459 \kpp{HOCH2CH2O2} + 0.079 \kpp{C2H5O2} + 0.462 \kpp{CH3CO3} + 0.462 \kpp{CH3CHO} + \kpp{NO2} & \code{KRO2NO} & \citet{2419}$^*$\shownote{G4415}\\
\code{G4416} & TrGC &  \kpp{LMEKOOH} + \kpp{OH} $\rightarrow$ 0.40851 \kpp{CH3COCH2O2} + 0.350196 \kpp{BIACET} + 0.807212 \kpp{OH} + 0.048506 \kpp{C2H5O2} + 0.505522 \kpp{CO2} + 0.192788 \kpp{LMEKO2} + \kpp{H2O} & \code{3.786E-11*EXP(LOG(1.25)*mcexp(136))} & \citet{2419}$^*$\shownote{G4416}\\
\code{G4417} & TrGNC &  \kpp{LC4H9NO3} + \kpp{OH} $\rightarrow$ 0.91423 \kpp{MEK} + 0.08577 \kpp{C2H5O2} + 0.17154 \kpp{CO2} + \kpp{NO2} + \kpp{H2O} & \code{9.598E-13*EXP(LOG(1.25)*mcexp(137))} & \citet{2419}$^*$\shownote{G4417}\\
\code{G4418} & TrGNC &  \kpp{MPAN} + \kpp{OH} $\rightarrow$ \kpp{ACETOL} + \kpp{CO} + \kpp{NO2} & \code{3.2E-11*EXP(LOG(1.25)*mcexp(138))} & \citet{1629}\\
\code{G4419} & TrGNC &  \kpp{MPAN} $\rightarrow$ \kpp{MACO3} + \kpp{NO2} & \code{k_PAN_M} & see note\shownote{G4419}\\
\code{G4420} & TrGC &  \kpp{LMEKO2} $\rightarrow$ 0.538 \kpp{HCHO} + 0.538 \kpp{CO2} + 0.459 \kpp{HOCH2CH2O2} + 0.079 \kpp{C2H5O2} + 0.462 \kpp{CH3CO3} + 0.462 \kpp{CH3CHO} & \code{1.483E-12*EXP(LOG(1.25)*mcexp(139))*RO2} & \citet{2419}$^*$\shownote{G4420}\\
\code{G4421} & TrGC &  \kpp{MACR} + \kpp{OH} $\rightarrow$ .57 \kpp{MACO3} + .43 \kpp{MACRO2} & \code{1.86E-11*EXP(LOG(1.25)*mcexp(140))*EXP(175./temp)} & \citet{2419}\\
\code{G4422} & TrGC &  \kpp{MACR} + \kpp{O3} $\rightarrow$ .59 \kpp{MGLYOX} + .41 \kpp{CH3CO3} + .03375 \kpp{HCOOH} + .55625 \kpp{HCHO} + .82 \kpp{CO} + .12375 \kpp{H2O2} + .41 \kpp{HO2} + .82 \kpp{OH} & \code{1.36E-15*EXP(LOG(1.25)*mcexp(141))*EXP(-2112./temp)} & \citet{2419}\\
\code{G4423} & TrGNC &  \kpp{MACR} + \kpp{NO3} $\rightarrow$ \kpp{MACO3} + \kpp{HNO3} & \code{KNO3AL*2.0} & \citet{2419}\\
\code{G4424} & TrGC &  \kpp{MACO3} $\rightarrow$ .7 \kpp{CH3CO3} + .7 \kpp{HCHO} + .7 \kpp{CO2} + .3 \kpp{MACO2H} & \code{1.00E-11*EXP(LOG(1.25)*mcexp(142))*RO2} & \citet{2419}\\
\code{G4425} & TrGC &  \kpp{MACO3} + \kpp{HO2} $\rightarrow$ .71 \kpp{MACO3H} + .29 \kpp{MACO2H} + .29 \kpp{O3} & \code{KAPHO2} & \citet{2419}\\
\code{G4426} & TrGNC &  \kpp{MACO3} + \kpp{NO} $\rightarrow$ \kpp{CH3CO3} + \kpp{HCHO} + \kpp{NO2} + \kpp{CO2} & \code{8.70E-12*EXP(LOG(1.25)*mcexp(143))*EXP(290./temp)} & \citet{2419}\\
\code{G4427} & TrGNC &  \kpp{MACO3} + \kpp{NO2} $\rightarrow$ \kpp{MPAN} & \code{k_CH3CO3_NO2} & \citet{2419}\\
\code{G4428} & TrGNC &  \kpp{MACO3} + \kpp{NO3} $\rightarrow$ \kpp{CH3CO3} + \kpp{HCHO} + \kpp{NO2} + \kpp{CO2} & \code{KRO2NO3*1.60} & \citet{2419}\\
\code{G4429} & TrGC &  \kpp{MACRO2} $\rightarrow$ .7 \kpp{ACETOL} + .7 \kpp{HCHO} + .7 \kpp{HO2} + .3 \kpp{MACROH} & \code{9.20E-14*EXP(LOG(1.25)*mcexp(144))*RO2} & \citet{2419}\\
\code{G4430} & TrGC &  \kpp{MACRO2} + \kpp{HO2} $\rightarrow$ \kpp{MACROOH} & \code{KRO2HO2*0.625} & \citet{2419}\\
\code{G4431} & TrGNC &  \kpp{MACRO2} + \kpp{NO} $\rightarrow$ \kpp{ACETOL} + \kpp{HCHO} + \kpp{HO2} + \kpp{NO2} & \code{KRO2NO} & \citet{2419}\\
\code{G4432} & TrGNC &  \kpp{MACRO2} + \kpp{NO3} $\rightarrow$ \kpp{ACETOL} + \kpp{HCHO} + \kpp{HO2} + \kpp{NO2} & \code{KRO2NO3} & \citet{2419}\\
\code{G4433} & TrGC &  \kpp{MACROOH} + \kpp{OH} $\rightarrow$ \kpp{MACRO2} & \code{2.82E-11*EXP(LOG(1.25)*mcexp(145))} & \citet{2419}\\
\code{G4434} & TrGC &  \kpp{MACROH} + \kpp{OH} $\rightarrow$ \kpp{ACETOL} + \kpp{HCHO} + \kpp{HO2} & \code{2.46E-11*EXP(LOG(1.25)*mcexp(146))} & \citet{2419}\\
\code{G4435} & TrGC &  \kpp{MACO2H} + \kpp{OH} $\rightarrow$ \kpp{CH3CO3} + \kpp{HCHO} + \kpp{CO2} & \code{1.51E-11*EXP(LOG(1.25)*mcexp(147))} & \citet{2419}\\
\code{G4436} & TrGC &  \kpp{MACO3H} + \kpp{OH} $\rightarrow$ \kpp{MACO3} & \code{1.87E-11*EXP(LOG(1.25)*mcexp(148))} & \citet{2419}\\
\code{G4437} & TrGC &  \kpp{LHMVKABO2} $\rightarrow$ 0.06 \kpp{CO2H3CHO} + 0.18 \kpp{HO2} + 0.18 \kpp{HCHO} + 0.18 \kpp{MGLYOX} + 0.42 \kpp{CH3CO3} + .42 \kpp{HOCH2CHO}+ .2 \kpp{HO12CO3C4} + .14 \kpp{BIACETOH} & \code{(.3*2.00E-12 + .7*8.80E-13)*EXP(LOG(1.25)*mcexp(149))*RO2} & \citet{2419}$^*$\shownote{G4437}\\
\code{G4438} & TrGC &  \kpp{LHMVKABO2} + \kpp{HO2} $\rightarrow$ \kpp{LHMVKABOOH} & \code{KRO2HO2*0.625} & \citet{2419}\\
\code{G4439} & TrGNC &  \kpp{LHMVKABO2} + \kpp{NO} $\rightarrow$ .3 \kpp{MGLYOX} + .7 \kpp{HOCH2CHO} + .7 \kpp{CH3CO3} + .3 \kpp{HCHO} + .3 \kpp{HO2} + \kpp{NO2} & \code{KRO2NO} & \citet{2419}$^*$\shownote{G4439}\\
\code{G4440} & TrGNC &  \kpp{LHMVKABO2} + \kpp{NO3} $\rightarrow$ .3 \kpp{MGLYOX} + .7 \kpp{HOCH2CHO} + .7 \kpp{CH3CO3} + .3 \kpp{HCHO} + .3 \kpp{HO2} + \kpp{NO2} & \code{KRO2NO3} & \citet{2419}$^*$\shownote{G4440}\\
\code{G4441} & TrGC &  \kpp{LHMVKABOOH} + \kpp{OH} $\rightarrow$ .3 \kpp{CO2H3CHO} + .7 \kpp{BIACETOH} + \kpp{OH} & \code{4.496E-11*EXP(LOG(1.25)*mcexp(150))} & \citet{2419}$^*$\shownote{G4441}\\
\code{G4442} & TrGC &  \kpp{MVKOH} + \kpp{OH} $\rightarrow$ \kpp{LMVKOHABO2} & \code{4.60E-12*EXP(LOG(1.25)*mcexp(151))*EXP(452./temp)} & \citet{2419}\\
\code{G4443} & TrGC &  \kpp{MVKOH} + \kpp{O3} $\rightarrow$ 0.56 \kpp{CO} + 0.545 \kpp{HOCH2COCHO} + 0.075 \kpp{HOCH2COCO2H} + 0.075 \kpp{HCOOH} + 0.09 \kpp{H2O2} + 0.28 \kpp{HOCH2CO3} + 0.28 \kpp{HO2} + 0.2 \kpp{CO2} + 0.545 \kpp{HCHO} + 0.36 \kpp{OH} + 0.1 \kpp{HOCH2CHO} & \code{7.51E-16*EXP(LOG(1.25)*mcexp(152))*EXP(-1521./temp)} & \citet{2419}\\
\code{G4444} & TrGC &  \kpp{LMVKOHABO2} $\rightarrow$ .7 \kpp{HOCH2CHO} + .7 \kpp{HOCH2CO3} + .3 \kpp{HOCH2COCHO} + .3 \kpp{HCHO} + .3 \kpp{HO2} & \code{(0.3*2.00E-12+0.7*8.80E-13)*EXP(LOG(1.25)*mcexp(153))*RO2} & \citet{2419}$^*$\shownote{G4444}\\
\code{G4445} & TrGC &  \kpp{LMVKOHABO2} + \kpp{HO2} $\rightarrow$ \kpp{LMVKOHABOOH} & \code{KRO2HO2*0.625} & \citet{2419}\\
\code{G4446} & TrGNC &  \kpp{LMVKOHABO2} + \kpp{NO} $\rightarrow$ .3 \kpp{HOCH2COCHO} + .3 \kpp{HCHO} + .3 \kpp{HO2} + .7 \kpp{HOCH2CHO} + .7 \kpp{HOCH2CO3} + \kpp{NO2} & \code{KRO2NO} & \citet{2419}$^*$\shownote{G4446}\\
\code{G4447} & TrGNC &  \kpp{LMVKOHABO2} + \kpp{NO3} $\rightarrow$ .3 \kpp{HOCH2COCHO} + .3 \kpp{HCHO} + .3 \kpp{HO2} + .7 \kpp{HOCH2CHO} + .7 \kpp{HOCH2CO3} + \kpp{NO2} & \code{KRO2NO3} & \citet{2419}$^*$\shownote{G4447}\\
\code{G4448} & TrGC &  \kpp{LMVKOHABOOH} + \kpp{OH} $\rightarrow$ .7 \kpp{HO12CO3C4} + .3 \kpp{CO2H3CHO} + \kpp{OH} & \code{5.98E-11*EXP(LOG(1.25)*mcexp(154))} & \citet{2419}$^*$\shownote{G4448}\\
\code{G4449} & TrGC &  \kpp{CO2H3CHO} + \kpp{OH} $\rightarrow$ \kpp{CO2H3CO3} & \code{2.45E-11*EXP(LOG(1.25)*mcexp(155))} & \citet{2419}\\
\code{G4450} & TrGNC &  \kpp{CO2H3CHO} + \kpp{NO3} $\rightarrow$ \kpp{CO2H3CO3} + \kpp{HNO3} & \code{KNO3AL*4.0} & \citet{2419}\\
\code{G4451} & TrGC &  \kpp{CO2H3CO3} $\rightarrow$ \kpp{MGLYOX} + \kpp{HO2} + \kpp{CO2} & \code{1.00E-11*EXP(LOG(1.25)*mcexp(156))*RO2} & \citet{2419}\\
\code{G4452} & TrGC &  \kpp{CO2H3CO3} + \kpp{HO2} $\rightarrow$ \kpp{CO2H3CO3H} & \code{KAPHO2} & \citet{2419}\\
\code{G4453} & TrGNC &  \kpp{CO2H3CO3} + \kpp{NO} $\rightarrow$ \kpp{MGLYOX} + \kpp{HO2} + \kpp{NO2} + \kpp{CO2} & \code{KAPNO} & \citet{2419}\\
\code{G4454} & TrGNC &  \kpp{CO2H3CO3} + \kpp{NO3} $\rightarrow$ \kpp{MGLYOX} + \kpp{HO2} + \kpp{NO2} + \kpp{CO2} & \code{KRO2NO3*1.60} & \citet{2419}\\
\code{G4455} & TrGC &  \kpp{CO2H3CO3H} + \kpp{OH} $\rightarrow$ \kpp{CO2H3CO3} & \code{7.34E-12*EXP(LOG(1.25)*mcexp(157))} & \citet{2419}\\
\code{G4456} & TrGC &  \kpp{HO12CO3C4} + \kpp{OH} $\rightarrow$ \kpp{BIACETOH} + \kpp{HO2} & \code{1.88E-11*EXP(LOG(1.25)*mcexp(158))} & \citet{2419}\\
\code{G4500} & TrGC &  \kpp{C5H8} + \kpp{O3} $\rightarrow$ .051 \kpp{CH3O2} + .1575 \kpp{CH3CO3} + .054 \kpp{LHMVKABO2} + .522 \kpp{CO} + .06875 \kpp{HCOOH} + .11 \kpp{H2O2} + .32475 \kpp{MACR} + .1275 \kpp{C3H6} + .2625 \kpp{HO2} + .255 \kpp{CO2} + .74975 \kpp{HCHO} + .04125 \kpp{MACO2H} + .27 \kpp{OH} + .244 \kpp{MVK} & \code{7.86E-15*EXP(LOG(1.25)*mcexp(159))*EXP(-1913./temp)} & \citet{2419}\\
\code{G4501} & TrGC &  \kpp{C5H8} + \kpp{OH} $\rightarrow$ .25 \kpp{LISOPACO2} + .491 \kpp{ISOPBO2} + .259 \kpp{ISOPDO2} & \code{2.54E-11*EXP(LOG(1.25)*mcexp(160))*EXP(410./temp)} & \citet{964}\\
\code{G4509} & TrGNC &  \kpp{C5H8} + \kpp{NO3} $\rightarrow$ \kpp{NISOPO2} & \code{3.03E-12*EXP(LOG(1.25)*mcexp(161))*EXP(-446./temp)} & \citet{2419}\\
\code{G4510} & TrGC &  \kpp{LISOPACO2} $\rightarrow$ .9 \kpp{LHC4ACCHO} + .8 \kpp{HO2} + .1 \kpp{ISOPAOH} & \code{2.4E-12*EXP(LOG(1.25)*mcexp(162))*RO2} & \citet{2419}\\
\code{G4511} & TrGC &  \kpp{LISOPACO2} + \kpp{HO2} $\rightarrow$ \kpp{LISOPACOOH} & \code{0.706*KRO2HO2} & \citet{2419}\\
\code{G4512} & TrGNC &  \kpp{LISOPACO2} + \kpp{NO} $\rightarrow$ .892 \kpp{LHC4ACCHO} + .892 \kpp{HO2} + .892 \kpp{NO2} + .108 \kpp{LISOPACNO3} & \code{KRO2NO} & \citet{2419}\\
\code{G4513} & TrGNC &  \kpp{LISOPACO2} + \kpp{NO3} $\rightarrow$ \kpp{LHC4ACCHO} + \kpp{HO2} + \kpp{NO2} & \code{KRO2NO3} & \citet{2419}\\
\code{G4514} & TrGC &  \kpp{LISOPACOOH} + \kpp{OH} $\rightarrow$ \kpp{LHC4ACCHO} + \kpp{OH} & \code{1.07E-10*EXP(LOG(1.25)*mcexp(163))} & \citet{2419}\\
\code{G4515} & TrGC &  \kpp{ISOPAOH} + \kpp{OH} $\rightarrow$ \kpp{LHC4ACCHO} + \kpp{HO2} & \code{9.30E-11*EXP(LOG(1.25)*mcexp(164))} & \citet{2419}\\
\code{G4516} & TrGNC &  \kpp{LISOPACNO3} + \kpp{OH} $\rightarrow$ \kpp{LHC4ACCHO} + \kpp{NO2} & \code{8.91E-11*EXP(LOG(1.25)*mcexp(165))} & \citet{2419}\\
\code{G4517} & TrGC &  \kpp{ISOPBO2} $\rightarrow$ .6 \kpp{MVK} + .2 \kpp{MVKOH} + .6 \kpp{HCHO} + .6 \kpp{HO2} + .2 \kpp{CH3O2} + .2 \kpp{ISOPBOH} & \code{8.E-13*EXP(LOG(1.25)*mcexp(166))*RO2} & \citet{2419}\\
\code{G4518} & TrGC &  \kpp{ISOPBO2} + \kpp{HO2} $\rightarrow$ \kpp{ISOPBOOH} & \code{0.706*KRO2HO2} & \citet{2419}\\
\code{G4519} & TrGNC &  \kpp{ISOPBO2} + \kpp{NO} $\rightarrow$ .696 \kpp{MVK} + .232 \kpp{MVKOH} + .696 \kpp{HCHO} + .696 \kpp{HO2} + .232 \kpp{CH3O2} + .928 \kpp{NO2} + .072 \kpp{ISOPBNO3} & \code{KRO2NO} & \citet{2419}\\
\code{G4520} & TrGNC &  \kpp{ISOPBO2} + \kpp{NO3} $\rightarrow$ .75 \kpp{MVK} + .25 \kpp{MVKOH} + .75 \kpp{HCHO} + .75 \kpp{HO2} + .25 \kpp{CH3O2} + \kpp{NO2} & \code{KRO2NO3} & \citet{2419}\\
\code{G4521} & TrGC &  \kpp{ISOPBOOH} + \kpp{OH} $\rightarrow$ \kpp{ISOPBO2} & \code{4.2E-11*EXP(LOG(1.25)*mcexp(167))} & \citet{2419}\\
\code{G4522} & TrGC &  \kpp{ISOPBOH} + \kpp{OH} $\rightarrow$ .75 \kpp{MVK} + .25 \kpp{MVKOH} + .75 \kpp{HCHO} + .75 \kpp{HO2} + .25 \kpp{CH3O2} & \code{3.85E-11*EXP(LOG(1.25)*mcexp(168))} & \citet{2419}\\
\code{G4523} & TrGNC &  \kpp{ISOPBNO3} + \kpp{OH} $\rightarrow$ \kpp{MVK} + \kpp{HCHO} + \kpp{NO2} & \code{3.55E-11*EXP(LOG(1.25)*mcexp(169))} & \citet{2419}\\
\code{G4524} & TrGC &  \kpp{ISOPDO2} $\rightarrow$ .8 \kpp{MACR} + .8 \kpp{HCHO} + .8 \kpp{HO2} + .1 \kpp{HCOC5} + .1 \kpp{ISOPDOH} & \code{2.9E-12*EXP(LOG(1.25)*mcexp(170))*RO2} & \citet{2419}\\
\code{G4525} & TrGC &  \kpp{ISOPDO2} + \kpp{HO2} $\rightarrow$ \kpp{ISOPDOOH} & \code{0.706*KRO2HO2} & \citet{2419}\\
\code{G4526} & TrGNC &  \kpp{ISOPDO2} + \kpp{NO} $\rightarrow$ .855 \kpp{MACR} + .855 \kpp{HCHO} + .855 \kpp{HO2} + .855 \kpp{NO2} + .145 \kpp{ISOPDNO3} & \code{KRO2NO} & \citet{2419}\\
\code{G4527} & TrGNC &  \kpp{ISOPDO2} + \kpp{NO3} $\rightarrow$ \kpp{MACR} + \kpp{HCHO} + \kpp{HO2} + \kpp{NO2} & \code{KRO2NO3} & \citet{2419}\\
\code{G4528} & TrGC &  \kpp{ISOPDOOH} + \kpp{OH} $\rightarrow$ \kpp{HCOC5} + \kpp{OH} & \code{1.07E-10*EXP(LOG(1.25)*mcexp(171))} & \citet{2419}\\
\code{G4529} & TrGC &  \kpp{ISOPDOH} + \kpp{OH} $\rightarrow$ \kpp{HCOC5} + \kpp{HO2} & \code{7.38E-11*EXP(LOG(1.25)*mcexp(172))} & \citet{2419}\\
\code{G4530} & TrGNC &  \kpp{ISOPDNO3} + \kpp{OH} $\rightarrow$ \kpp{HCOC5} + \kpp{NO2} & \code{6.1E-11*EXP(LOG(1.25)*mcexp(173))} & \citet{2419}\\
\code{G4531} & TrGNC &  \kpp{NISOPO2} $\rightarrow$ .8 \kpp{NC4CHO} + .6 \kpp{HO2} + .2 \kpp{LISOPACNO3} & \code{1.3E-12*EXP(LOG(1.25)*mcexp(174))*RO2} & \citet{2419}\\
\code{G4532} & TrGNC &  \kpp{NISOPO2} + \kpp{HO2} $\rightarrow$ \kpp{NISOPOOH} & \code{.706*KRO2HO2} & \citet{2419}\\
\code{G4533} & TrGNC &  \kpp{NISOPO2} + \kpp{NO} $\rightarrow$ \kpp{NC4CHO} + \kpp{HO2} + \kpp{NO2} & \code{KRO2NO} & \citet{2419}\\
\code{G4534} & TrGNC &  \kpp{NISOPO2} + \kpp{NO3} $\rightarrow$ \kpp{NC4CHO} + \kpp{HO2} + \kpp{NO2} & \code{KRO2NO3} & \citet{2419}\\
\code{G4535} & TrGNC &  \kpp{NISOPOOH} + \kpp{OH} $\rightarrow$ \kpp{NC4CHO} + \kpp{OH} & \code{1.03E-10*EXP(LOG(1.25)*mcexp(175))} & \citet{2419}\\
\code{G4536} & TrGNC &  \kpp{NC4CHO} + \kpp{OH} $\rightarrow$ \kpp{LNISO3} & \code{4.16E-11*EXP(LOG(1.25)*mcexp(176))} & \citet{2419}\\
\code{G4537} & TrGNC &  \kpp{NC4CHO} + \kpp{O3} $\rightarrow$ .445 \kpp{NO2} + .89 \kpp{CO} + .075625 \kpp{H2O2} + .034375 \kpp{HCOCO2H} + .555 \kpp{NOA} + .445 \kpp{HO2} + .520625 \kpp{GLYOX} + .89 \kpp{OH} + .445 \kpp{MGLYOX} & \code{2.40E-17*EXP(LOG(1.25)*mcexp(177))} & \citet{2419}\\
\code{G4538} & TrGNC &  \kpp{NC4CHO} + \kpp{NO3} $\rightarrow$ \kpp{LNISO3} + \kpp{HNO3} & \code{KNO3AL*4.25} & \citet{2419}\\
\code{G4539} & TrGNC &  \kpp{LNISO3} + \kpp{HO2} $\rightarrow$ \kpp{LNISOOH} & \code{.5*.706*KRO2HO2 + .5*KAPHO2} & \citet{2419}\\
\code{G4540} & TrGNC &  \kpp{LNISO3} + \kpp{NO} $\rightarrow$ \kpp{NOA} + .5 \kpp{GLYOX} + .5 \kpp{CO} + \kpp{HO2} + \kpp{NO2} + .5 \kpp{CO2} & \code{.5*KAPNO +.5*KRO2NO} & \citet{2419}\\
\code{G4541} & TrGNC &  \kpp{LNISO3} + \kpp{NO3} $\rightarrow$ \kpp{NOA} + .5 \kpp{GLYOX} + .5 \kpp{CO} + \kpp{HO2} + \kpp{NO2} + .5 \kpp{CO2} & \code{1.3*KRO2NO3} & \citet{2419}\\
\code{G4542} & TrGNC &  \kpp{LNISOOH} + \kpp{OH} $\rightarrow$ \kpp{LNISO3} & \code{2.65E-11*EXP(LOG(1.25)*mcexp(178))} & \citet{2419}\\
\code{G4543} & TrGC &  \kpp{LHC4ACCHO} + \kpp{OH} $\rightarrow$ .52 \kpp{LC578O2} + .48 \kpp{LHC4ACCO3} & \code{4.52E-11*EXP(LOG(1.25)*mcexp(179))} & \citet{2419}\\
\code{G4544} & TrGC &  \kpp{LHC4ACCHO} + \kpp{O3} $\rightarrow$ .2225 \kpp{CH3CO3} + .89 \kpp{CO} + .0171875 \kpp{HOCH2CO2H} + .075625 \kpp{H2O2} + .0171875 \kpp{HCOCO2H} + .2775 \kpp{ACETOL} + .6675 \kpp{HO2} + .2603125 \kpp{GLYOX} + .2225 \kpp{HCHO} + .89 \kpp{OH} + .2603125 \kpp{HOCH2CHO} + .5 \kpp{MGLYOX} & \code{2.40E-17*EXP(LOG(1.25)*mcexp(180))} & \citet{2419}\\
\code{G4545} & TrGNC &  \kpp{LHC4ACCHO} + \kpp{NO3} $\rightarrow$ \kpp{LHC4ACCO3} + \kpp{HNO3} & \code{KNO3AL*4.25} & \citet{2419}\\
\code{G4546} & TrGC &  \kpp{LC578O2} $\rightarrow$ .5 \kpp{ACETOL} + .5 \kpp{MGLYOX} + .5 \kpp{GLYOX} + .5 \kpp{HOCH2CHO} + \kpp{HO2} & \code{9.20E-14*EXP(LOG(1.25)*mcexp(181))*RO2} & \citet{2419}\\
\code{G4547} & TrGC &  \kpp{LC578O2} + \kpp{HO2} $\rightarrow$ \kpp{LC578OOH} & \code{KRO2HO2*0.706} & \citet{2419}\\
\code{G4548} & TrGNC &  \kpp{LC578O2} + \kpp{NO} $\rightarrow$ .5 \kpp{ACETOL} + .5 \kpp{MGLYOX} + .5 \kpp{GLYOX} + .5 \kpp{HOCH2CHO} + \kpp{HO2} + \kpp{NO2} & \code{KRO2NO} & \citet{2419}\\
\code{G4549} & TrGNC &  \kpp{LC578O2} + \kpp{NO3} $\rightarrow$ .5 \kpp{ACETOL} + .5 \kpp{MGLYOX} + .5 \kpp{GLYOX} + .5 \kpp{HOCH2CHO} + \kpp{HO2} + \kpp{NO2} & \code{KRO2NO3} & \citet{2419}\\
\code{G4550} & TrGC &  \kpp{LC578OOH} + \kpp{OH} $\rightarrow$ \kpp{LC578O2} & \code{3.16E-11*EXP(LOG(1.25)*mcexp(182))} & \citet{2419}\\
\code{G4551} & TrGC &  \kpp{LHC4ACCO3} $\rightarrow$ .3 \kpp{LHC4ACCO2H} + .35 \kpp{ACETOL} + .35 \kpp{HOCH2CHO} + .35 \kpp{CH3CO3} + .35 \kpp{CO} + .35 \kpp{HO2} + .7 \kpp{CO2} & \code{1.00E-11*EXP(LOG(1.25)*mcexp(183))*RO2} & \citet{2419}\\
\code{G4552} & TrGC &  \kpp{LHC4ACCO3} + \kpp{HO2} $\rightarrow$ .71 \kpp{LHC4ACCO3H} + .29 \kpp{LHC4ACCO2H} + .29 \kpp{O3} & \code{KAPHO2} & \citet{2419}\\
\code{G4553} & TrGNC &  \kpp{LHC4ACCO3} + \kpp{NO} $\rightarrow$ .5 \kpp{ACETOL} + .5 \kpp{HOCH2CHO} + .5 \kpp{CH3CO3} + .5 \kpp{CO} + .5 \kpp{HO2} + \kpp{NO2} + \kpp{CO2} & \code{KAPNO} & \citet{2419}\\
\code{G4554} & TrGNC &  \kpp{LHC4ACCO3} + \kpp{NO2} $\rightarrow$ \kpp{LC5PAN1719} & \code{k_CH3CO3_NO2} & \citet{2419}\\
\code{G4555} & TrGNC &  \kpp{LHC4ACCO3} + \kpp{NO3} $\rightarrow$ .5 \kpp{ACETOL} + .5 \kpp{HOCH2CHO} + .5 \kpp{CH3CO3} + .5 \kpp{CO} + .5 \kpp{HO2} + \kpp{NO2} + \kpp{CO2} & \code{1.6*KRO2NO3} & \citet{2419}\\
\code{G4556} & TrGC &  \kpp{LHC4ACCO2H} + \kpp{OH} $\rightarrow$ .5 \kpp{ACETOL} + .5 \kpp{HOCH2CHO} + .5 \kpp{CH3CO3} + .5 \kpp{CO} + .5 \kpp{HO2} + \kpp{CO2} & \code{2.52E-11*EXP(LOG(1.25)*mcexp(184))} & \citet{2419}\\
\code{G4557} & TrGC &  \kpp{LHC4ACCO3H} + \kpp{OH} $\rightarrow$ \kpp{LHC4ACCO3} & \code{2.88E-11*EXP(LOG(1.25)*mcexp(185))} & \citet{2419}\\
\code{G4558} & TrGNC &  \kpp{LC5PAN1719} $\rightarrow$ \kpp{LHC4ACCO3} + \kpp{NO2} & \code{k_PAN_M} & \citet{2419}\\
\code{G4559} & TrGNC &  \kpp{LC5PAN1719} + \kpp{OH} $\rightarrow$ .5 \kpp{MACROH} + .5 \kpp{HO12CO3C4} + \kpp{CO} + \kpp{NO2} & \code{2.52E-11*EXP(LOG(1.25)*mcexp(186))} & \citet{2419}\\
\code{G4560} & TrGC &  \kpp{HCOC5} + \kpp{OH} $\rightarrow$ \kpp{C59O2} & \code{3.81E-11*EXP(LOG(1.25)*mcexp(187))} & \citet{2419}\\
\code{G4561} & TrGC &  \kpp{C59O2} $\rightarrow$ \kpp{ACETOL} + \kpp{HOCH2CO3} & \code{9.20E-14*EXP(LOG(1.25)*mcexp(188))*RO2} & \citet{2419}\\
\code{G4562} & TrGC &  \kpp{C59O2} + \kpp{HO2} $\rightarrow$ \kpp{C59OOH} & \code{KRO2HO2*0.706} & \citet{2419}\\
\code{G4563} & TrGNC &  \kpp{C59O2} + \kpp{NO} $\rightarrow$ \kpp{ACETOL} + \kpp{HOCH2CO3} + \kpp{NO2} & \code{KRO2NO} & \citet{2419}\\
\code{G4564} & TrGNC &  \kpp{C59O2} + \kpp{NO3} $\rightarrow$ \kpp{ACETOL} + \kpp{HOCH2CO3} + \kpp{NO2} & \code{KRO2NO3} & \citet{2419}\\
\code{G4565} & TrGC &  \kpp{C59OOH} + \kpp{OH} $\rightarrow$ \kpp{C59O2} & \code{9.7E-12*EXP(LOG(1.25)*mcexp(189))} & \citet{2419}\\
\myhline
\myhline
\myhline
\myhline
\myhline

\end{longtable}

\newpage\begin{multicols}{3}
$^*$Notes:

Rate coefficients for three-body reactions are defined via the function
\code{k_3rd}($T$, $M$, $k_0^{300}$, $n$, $k_{\rm inf}^{300}$, $m$,
$f_{\rm c}$). In the code, the temperature $T$ is called \code{temp} and
the concentration of ``air molecules'' $M$ is called \code{cair}. Using
the auxiliary variables $k_0(T)$, $k_{\rm inf}(T)$, and $k_{\rm ratio}$,
\code{k_3rd} is defined as:
\begin{eqnarray}
  k_0(T)              & = & k_0^{300} \times \left( \frac{300
                            \unit{K}}{T} \right)^n\\
  k_{\rm inf}(T)      & = & k_{\rm inf}^{300} \times \left( \frac{300
                            \unit{K}}{T} \right)^m\\
  k_{\rm ratio}       & = & \frac{k_0(T) M}{k_{\rm inf}(T)}\\
  \mbox{\code{k_3rd}} & = & \frac{k_0(T) M}{1+k_{\rm ratio}} \times f_{\rm
                            c}^{\left( \frac{1}{1+(\log_{10}(k_{\rm
                            ratio}))^2} \right)}
\end{eqnarray}

A similar function, called \code{k_3rd_iupac} here, is used by
\citet{1745} for three-body reactions. It has the same function
parameters as \code{k_3rd} and it is defined as:
\begin{eqnarray}
  k_0(T)                    & = & k_0^{300} \times \left( \frac{300
                                  \unit{K}}{T} \right)^n\\
  k_{\rm inf}(T)            & = & k_{\rm inf}^{300} \times \left( \frac{300
                                  \unit{K}}{T} \right)^m\\
  k_{\rm ratio}             & = & \frac{k_0(T) M}{k_{\rm inf}(T)}\\
  N                         & = & 0.75 - 1.27 \times \log_{10}(f_{\rm c})\\
  \mbox{\code{k_3rd_iupac}} & = & \frac{k_0(T) M}{1+k_{\rm ratio}} \times f_{\rm
                                  c}^{\left( \frac{1}{1+(\log_{10}(k_{\rm
                                  ratio})/N)^2} \right)}
\end{eqnarray}

\note{G1002}{The path leading to 2 \kpp{O3P} + \kpp{O2} results in a
  null cycle regarding odd oxygen and is neglected.}

\note{G2110}{The rate coefficient is: \code{k_HO2_HO2} =
  \code{(1.5E-12*EXP(19./temp)+1.7E-33*EXP(1000./temp)*cair)*
    (1.+1.4E-21*EXP(2200./temp)*C(ind_H2O))}. The value for the first
  (pressure-independent) part is from \citet{1599}, the water term from
  \citet{165}.}

\note{G3109}{The rate coefficient is: \code{k_NO3_NO2} =
  \code{k_3rd(temp,cair,2.E-30,4.4,1.4E-12,0.7,0.6)}.}

\note{G3110}{The rate coefficient is defined as backward reaction
  divided by equilibrium constant.}

\note{G3203}{The rate coefficient is: \code{k_NO2_HO2} =
  \code{k_3rd(temp,cair,1.8E-31,3.2,4.7E-12,1.4,0.6)}.}

\note{G3206}{The rate coefficient is: \code{k_HNO3_OH} = \code{2.4E-14 *
    EXP(460./temp) + 1./ ( 1./(6.5E-34 * EXP(1335./temp)*cair) +
    1./(2.7E-17 * EXP(2199./temp)) )}}

\note{G3207}{The rate coefficient is defined as backward reaction
  divided by equilibrium constant.}

\note{G4103}{\citet{1945} recommend a zero product yield for
  \chem{HCHO}.}

\note{G4107}{The rate coefficient is: \code{k_CH3OOH_OH} =
  \code{3.8E-12*EXP(200./temp)}.}

\note{G4109}{The same temperature dependence assumed as for
  \kpp{CH3CHO}+\kpp{NO3}.}

\note{G4201}{The product distribution is from \citet{1612} (see also
  \citet{1624}).}

\note{G4206}{The rate coefficient was calculated by von Kuhlmann (pers.\ 
  comm.\ 2004) using self reactions of \chem{CH_3OO} and \chem{C_2H_5OO}
  from \citet{1555} and geometric mean as suggested by \citet{471} and
  \citet{1632}. The product distribution (branching=0.5/0.25/0.25) is
  calculated by von Kuhlmann (pers.\ comm.\ 2004) based on \citet{597}
  and \citet{1613}.}

\note{G4207}{Same value as for G4107: \kpp{CH3OOH}+\kpp{OH} assumed.}

\note{G4213}{The rate coefficient is: \code{k_PA_NO2} =
  \code{k_3rd(temp,cair,8.5E-29,6.5,1.1E-11,1.,0.6)}.}

\note{G4216}{The value \code{1.0E-11} is from \citet{1207}, the temperature
  dependence from \citet{1632}.}

\note{G4218}{Same value as for G4107: \kpp{CH3OOH}+\kpp{OH} assumed.}

\note{G4219}{According to \citet{1614}, the same value as for
  \kpp{CH3CHO}+\kpp{OH} can be assumed.}

\note{G4220}{This is 50\% of the upper limit given by \citet{1555}, as
  suggested by \citet{1612}.}

\note{G4221}{The rate coefficient is: \code{k_PAN_M} =
  \code{k_PA_NO2/9.E-29*EXP(-14000./temp)}, i.e.\ the rate coefficient
  is defined as backward reaction divided by equilibrium constant.}

\note{G4301}{The product distribution is for terminal olefin carbons
  from \citet{1623}.}

\note{G4304}{The rate coefficient is: \code{k_PrO2_HO2} =
  \code{1.9E-13*EXP(1300./temp)}. Value for generic \chem{RO_2} +
  \chem{HO_2} reaction from \citet{964} is used.}

\note{G4305}{The rate coefficient is: \code{k_PrO2_NO} =
  \code{2.7E-12*EXP(360./temp)}.}

\note{G4306}{The rate coefficient is: \code{k_PrO2_CH3O2} =
  \code{9.46E-14*EXP(431./temp)}. The product distribution is from
  \citet{1612}.}

\note{G4307}{Same value as for G4107: \kpp{CH3OOH}+\kpp{OH} assumed.}

\note{G4309}{The products are from \citet{1612}.}

\note{G4315}{Same value as for G4107: \kpp{CH3OOH}+\kpp{OH} assumed.}

\note{G4319}{Same value as for \kpp{PAN} assumed.}

\note{G4401}{Same value as for propyl group assumed (\code{k_PrO2_CH3O2}).}

\note{G4402}{Same value as for propyl group assumed (\code{k_PrO2_HO2}).}

\note{G4403}{Same value as for propyl group assumed (\code{k_PrO2_NO}).}

\note{G4404}{Same value as for G4107: \kpp{CH3OOH}+\kpp{OH} assumed.}

\note{G4409}{The factor 0.25 was recommended by Uli Poeschl (pers.\ 
  comm.\ 2004).}

\note{G4414}{Same value as for propyl group assumed (\code{k_PrO2_HO2}).}

\note{G4415}{Same value as for propyl group assumed (\code{k_PrO2_NO}).}

\note{G4416}{Same value as for G4107: \kpp{CH3OOH}+\kpp{OH} assumed.}

\note{G4417}{Value for \chem{C_4H_9ONO_2} used here.}

\note{G4503}{Same temperature dependence assumed as for other
  \chem{RO_2}+\kpp{HO2} reactions.}

\note{G4504}{Yield of 12~\% \chem{RONO_2} assumed as suggested in Table
  2 of \citet{1631}.}

\note{G6102}{The rate coefficient is: \code{k_ClO_ClO} =
  \code{k_3rd_iupac(temp,cair,2.E-32,4.,1.E-11,0.,0.45)}.}

\note{G6103}{The rate coefficient is defined as backward reaction
  divided by equilibrium constant.}

\note{G6204}{At low temperatures, there may be a minor reaction channel
  leading to \kpp{O3}+\kpp{HCl}. See \citet{1626} for details. It is
  neglected here.}

\note{G6402}{The initial products are probably \chem{HCl} and
  \chem{CH_2OOH} \citep{1759}. It is assumed that \chem{CH_2OOH}
  dissociates into \chem{HCHO} and \chem{OH}.}

\note{G6405}{Average of reactions with \chem{CH_3Br} and \chem{CH_3F}
  from \citet{1945} (B.\ Steil, pers.\ comm.).}

\note{G6407}{Rough extrapolation from reactions with \chem{CH_3CF_3},
  \chem{CH_3CClF_2}, and \chem{CH_3CCl_2F} from \citet{1945}.}

\note{G7302}{The rate coefficient is: \code{k_BrO_NO2} =
  \code{k_3rd(temp,cair,5.2E-31,3.2,6.9E-12,2.9,0.6)}.}

\note{G7303}{The rate coefficient is defined as backward reaction
  \citep{1845} divided by equilibrium constant \citep{1054}.}

\note{G7407}{It is assumed that the reaction liberates all \kpp{Br}
  atoms. The fate of the carbon atom is currently not considered.}

\note{G7408}{It is assumed that the reaction liberates all \kpp{Br}
  atoms. The fate of the carbon atom is currently not considered.}

\note{G7605}{Same value as for G7408: \kpp{CH2Br2}+\kpp{OH} assumed. It
  is assumed that the reaction liberates all \kpp{Br} atoms but not
  \kpp{Cl}. The fate of the carbon atom is currently not considered.}

\note{G7606}{Same value as for G7408: \kpp{CH2Br2}+\kpp{OH} assumed. It
  is assumed that the reaction liberates all \kpp{Br} atoms but not
  \kpp{Cl}. The fate of the carbon atom is currently not considered.}

\note{G7607}{It is assumed that the reaction liberates all \kpp{Br}
  atoms but not \kpp{Cl}. The fate of the carbon atom is currently not
  considered.}

\note{G8102}{It is assumed that the reaction produces new particles.}

\note{G8103}{The yield of 38~\unit{\%} \chem{OIO} is from \citet{1845}.
  It is assumed here that the remaining 62~\unit{\%} produce 2 \chem{I}
  + \chem{O_2}.}

\note{G8300}{The rate coefficient is: \code{k_I_NO2} =
  \code{k_3rd_iupac(temp,cair,3.E-31,1.,6.6E-11,0.,0.63)}.}

\note{G8305}{The rate coefficient is defined as backward reaction
  \citep{1845} divided by equilibrium constant \citep{1644}.}

\note{G8306}{According to John Plane and John Crowley (pers.\ comm.\
  2007), the rate coefficient of \code{1.1E15*EXP(-12060./temp)}
  suggested by \citet{1845} is wrong.}

\note{G8401}{The rate coefficient is from \citet{2079}, the yield of
  \chem{I} atoms is a lower limit given on page 2170 of \citet{2089}.}

\note{G8402}{The products are from \citet{2087}.}

\note{G8701}{80\% \chem{Br} + \chem{OIO} production is from
  \citet{1845}. The remaining channels are assumed to produce \chem{Br}
  + \chem{I} + \chem{O_2}.}

\note{G9400a}{Abstraction path. The assumed reaction sequence (omitting
  \chem{H_2O} and \chem{O_2} as products) according to \citet{243} is:
  \begin{eqnarray*}
    \chem{DMS} + \chem{OH}         & \TO & \chem{CH_3SCH_2}\\
    \chem{CH_3SCH_2} + \chem{O_2}  & \TO & \chem{CH_3SCH_2OO}\\
    \chem{CH_3SCH_2OO} + \chem{NO} & \TO & \chem{CH_3SCH_2O} + \chem{NO_2}\\
    \chem{CH_3SCH_2O}              & \TO & \chem{CH_3S} + \chem{HCHO}\\
    \chem{CH_3S} + \chem{O_3}      & \TO & \chem{CH_3SO}\\
    \chem{CH_3SO} + \chem{O_3}     & \TO & \chem{CH_3SO_2}\\
    \hline                                                  
    \chem{DMS}+\chem{OH}+\chem{NO}+2\chem{O_3} & \TO &
    \chem{CH_3SO_2}+\chem{HCHO}+\chem{NO_2}
  \end{eqnarray*}
  Neglecting the effect on \chem{O_3} and \chem{NO_x}, the remaining
  reaction is:
  $$\chem{DMS} + \chem{OH} + \chem{O_3} \TO \chem{CH_3SO_2} + \chem{HCHO}$$}

\note{G9400}{Addition path. The rate coefficient is: \code{k_DMS_OH} =
  \code{1.0E-39*EXP(5820./temp)*C(ind_O2)/
    (1.+5.0E-30*EXP(6280./temp)*C(ind_O2))}.}

\note{G10201}{Upper limit.}

\end{multicols}

\clearpage

\begin{longtable}{llp{10cm}p{6cm}p{4cm}}
\caption{Photolysis reactions}\\
\hline
\# & labels & reaction & rate coefficient & reference\\
\hline
\endfirsthead
\caption{Photolysis reactions (... continued)}\\
\hline
\# & labels & reaction & rate coefficient & reference\\
\hline
\endhead
\hline
\endfoot
% this file was created automatically by eqn2tex, do not edit!
\code{J1000} & StTrGJ &  \kpp{O2} + \kpp{hv} $\rightarrow$ \kpp{O3P} + \kpp{O3P} & \code{jx(ip_O2)*EXP(LOG(1.25)*mcexp(190))} & see note\shownote{J1000}\\
\code{J1001a} & StTrGJ &  \kpp{O3} + \kpp{hv} $\rightarrow$ \kpp{O1D} & \code{jx(ip_O1D)*EXP(LOG(1.25)*mcexp(191))} & see note\shownote{J1001}\\
\code{J1001b} & StTrGJ &  \kpp{O3} + \kpp{hv} $\rightarrow$ \kpp{O3P} & \code{jx(ip_O3P)*EXP(LOG(1.25)*mcexp(192))} & see note\shownote{J1001}\\
\myhline
\code{J2101} & StTrGJ &  \kpp{H2O2} + \kpp{hv} $\rightarrow$ 2 \kpp{OH} & \code{jx(ip_H2O2)*EXP(LOG(1.25)*mcexp(193))} & see note\shownote{J2101}\\
\myhline
\code{J3101} & StTrGNJ &  \kpp{NO2} + \kpp{hv} $\rightarrow$ \kpp{NO} + \kpp{O3P} & \code{jx(ip_NO2)*EXP(LOG(1.25)*mcexp(194))} & see note\shownote{J3101}\\
\code{J3103a} & StTrGNJ &  \kpp{NO3} + \kpp{hv} $\rightarrow$ \kpp{NO2} + \kpp{O3P} & \code{jx(ip_NO2O)*EXP(LOG(1.25)*mcexp(195))} & see note\shownote{J3103}\\
\code{J3103b} & StTrGNJ &  \kpp{NO3} + \kpp{hv} $\rightarrow$ \kpp{NO} & \code{jx(ip_NOO2)*EXP(LOG(1.25)*mcexp(196))} & see note\shownote{J3103}\\
\code{J3104a} & StTrGNJ &  \kpp{N2O5} + \kpp{hv} $\rightarrow$ \kpp{NO2} + \kpp{NO3} & \code{jx(ip_N2O5)*EXP(LOG(1.25)*mcexp(197))} & see note\shownote{J3104}\\
\code{J3200} & TrGJ &  \kpp{HONO} + \kpp{hv} $\rightarrow$ \kpp{NO} + \kpp{OH} & \code{jx(ip_HONO)*EXP(LOG(1.25)*mcexp(198))} & see note\shownote{J3200}\\
\code{J3201} & StTrGNJ &  \kpp{HNO3} + \kpp{hv} $\rightarrow$ \kpp{NO2} + \kpp{OH} & \code{jx(ip_HNO3)*EXP(LOG(1.25)*mcexp(199))} & see note\shownote{J3201}\\
\code{J3202} & StTrGNJ &  \kpp{HNO4} + \kpp{hv} $\rightarrow$ .667 \kpp{NO2} + .667 \kpp{HO2} + .333 \kpp{NO3} + .333 \kpp{OH} & \code{jx(ip_HNO4)*EXP(LOG(1.25)*mcexp(200))} & see note\shownote{J3202}\\
\myhline
\code{J4100} & StTrGJ &  \kpp{CH3OOH} + \kpp{hv} $\rightarrow$ \kpp{HCHO} + \kpp{OH} + \kpp{HO2} & \code{jx(ip_CH3OOH)*EXP(LOG(1.25)*mcexp(201))} & see note\shownote{J4100}\\
\code{J4101a} & StTrGJ &  \kpp{HCHO} + \kpp{hv} $\rightarrow$ \kpp{H2} + \kpp{CO} & \code{jx(ip_COH2)*EXP(LOG(1.25)*mcexp(202))} & see note\shownote{J4101}\\
\code{J4101b} & StTrGJ &  \kpp{HCHO} + \kpp{hv} $\rightarrow$ \kpp{H} + \kpp{CO} + \kpp{HO2} & \code{jx(ip_CHOH)*EXP(LOG(1.25)*mcexp(203))} & see note\shownote{J4101}\\
\code{J4200} & TrGCJ &  \kpp{C2H5OOH} + \kpp{hv} $\rightarrow$ \kpp{CH3CHO} + \kpp{HO2} + \kpp{OH} & \code{jx(ip_CH3OOH)*EXP(LOG(1.25)*mcexp(204))} & \citet{1612}$^*$\shownote{J4200}\\
\code{J4201} & TrGCJ &  \kpp{CH3CHO} + \kpp{hv} $\rightarrow$ \kpp{CH3O2} + \kpp{HO2} + \kpp{CO} & \code{jx(ip_CH3CHO)*EXP(LOG(1.25)*mcexp(205))} & see note\shownote{J4201}\\
\code{J4202} & TrGCJ &  \kpp{CH3CO3H} + \kpp{hv} $\rightarrow$ \kpp{CH3O2} + \kpp{OH} + \kpp{CO2} & \code{jx(ip_CH3CO3H)*EXP(LOG(1.25)*mcexp(206))} & see note\shownote{J4202}\\
\code{J4204} & TrGNCJ &  \kpp{PAN} + \kpp{hv} $\rightarrow$ \kpp{CH3CO3} + \kpp{NO2} & \code{jx(ip_PAN)*EXP(LOG(1.25)*mcexp(207))} & see note\shownote{J4204}\\
\code{J4205} & TrGCJ &  \kpp{HOCH2CHO} + \kpp{hv} $\rightarrow$ \kpp{HO2} + \kpp{HCHO} + \kpp{HO2} + \kpp{CO} & \code{jx(ip_HOCH2CHO)*EXP(LOG(1.25)*mcexp(208))} & see note\shownote{J4205}\\
\code{J4206} & TrGCJ &  \kpp{HOCH2CO3H} + \kpp{hv} $\rightarrow$ \kpp{HCHO} + \kpp{HO2} + \kpp{OH} + \kpp{CO2} & \code{jx(ip_CH3OOH)*EXP(LOG(1.25)*mcexp(209))} & \citet{2419}$^*$\shownote{J4206}\\
\code{J4207} & TrGCJ &  \kpp{PHAN} + \kpp{hv} $\rightarrow$ \kpp{HOCH2CO3} + \kpp{NO2} & \code{jx(ip_PAN)*EXP(LOG(1.25)*mcexp(210))} & see note\shownote{J4207}\\
\code{J4208} & TrGCJ &  \kpp{GLYOX} + \kpp{hv} $\rightarrow$ 2 \kpp{CO} + 2 \kpp{HO2} & \code{jx(ip_GLYOX)*EXP(LOG(1.25)*mcexp(211))} & see note\shownote{J4208}\\
\code{J4209} & TrGNCJ &  \kpp{HCOCO2H} + \kpp{hv} $\rightarrow$ 2 \kpp{HO2} + \kpp{CO} + \kpp{CO2} & \code{jx(ip_MGLYOX)*EXP(LOG(1.25)*mcexp(212))} & \citet{2419}$^*$\shownote{J4209}\\
\code{J4210} & TrGNCJ &  \kpp{HCOCO3H} + \kpp{hv} $\rightarrow$ \kpp{HO2} + \kpp{CO} + \kpp{OH} + \kpp{CO2} & \code{(jx(ip_CH3OOH)+jx(ip_HOCH2CHO))*EXP(LOG(1.25)*mcexp(213))} & \citet{2419}$^*$\shownote{J4210}\\
\code{J4211} & TrGCJ &  \kpp{HYETHO2H} + \kpp{hv} $\rightarrow$ \kpp{HOCH2CH2O} + \kpp{OH} & \code{jx(ip_CH3OOH)*EXP(LOG(1.25)*mcexp(214))} & \citet{2419}$^*$\shownote{J4211}\\
\code{J4212} & TrGCJ &  \kpp{ETHOHNO3} + \kpp{hv} $\rightarrow$ \kpp{HO2} + 2 \kpp{HCHO} + \kpp{NO2} & \code{J_IC3H7NO3} & see note\shownote{J4212}\\
\code{J4300} & TrGCJ &  \kpp{IC3H7OOH} + \kpp{hv} $\rightarrow$ \kpp{CH3COCH3} + \kpp{HO2} + \kpp{OH} & \code{jx(ip_CH3OOH)*EXP(LOG(1.25)*mcexp(215))} & \citet{1612}$^*$\shownote{J4300}\\
\code{J4301} & TrGCJ &  \kpp{CH3COCH3} + \kpp{hv} $\rightarrow$ \kpp{CH3CO3} + \kpp{CH3O2} & \code{jx(ip_CH3COCH3)*EXP(LOG(1.25)*mcexp(216))} & see note\shownote{J4301}\\
\code{J4302} & TrGCJ &  \kpp{ACETOL} + \kpp{hv} $\rightarrow$ \kpp{CH3CO3} + \kpp{HCHO} + \kpp{HO2} & \code{J_ACETOL} & see note\shownote{J4302}\\
\code{J4303} & TrGCJ &  \kpp{MGLYOX} + \kpp{hv} $\rightarrow$ \kpp{CH3CO3} + \kpp{CO} + \kpp{HO2} & \code{jx(ip_MGLYOX)*EXP(LOG(1.25)*mcexp(217))} & see note\shownote{J4303}\\
\code{J4304} & TrGCJ &  \kpp{HYPERACET} + \kpp{hv} $\rightarrow$ \kpp{CH3CO3} + \kpp{HCHO} + \kpp{OH} & \code{jx(ip_CH3OOH)*EXP(LOG(1.25)*mcexp(218))+J_ACETOL} & \citet{2419}$^*$\shownote{J4304}\\
\code{J4306} & TrGNCJ &  \kpp{IC3H7NO3} + \kpp{hv} $\rightarrow$ \kpp{CH3COCH3} + \kpp{NO2} + \kpp{HO2} & \code{J_IC3H7NO3} & \citet{1584}$^*$\shownote{J4306}\\
\code{J4307} & TrGCJ &  \kpp{NOA} + \kpp{hv} $\rightarrow$ \kpp{CH3CO3} + \kpp{HCHO} + \kpp{NO2} & \code{J_IC3H7NO3+jx(ip_CH3COCH3)*EXP(LOG(1.25)*mcexp(219))} & see note\shownote{J4307}\\
\code{J4308} & TrGCJ &  \kpp{HOCH2COCO2H} + \kpp{hv} $\rightarrow$ \kpp{HOCH2CO3} + \kpp{HO2} + \kpp{CO2} & \code{jx(ip_MGLYOX)*EXP(LOG(1.25)*mcexp(220))} & \citet{2419}$^*$\shownote{J4308}\\
\code{J4309} & TrGCJ &  \kpp{HYPROPO2H} + \kpp{hv} $\rightarrow$ \kpp{CH3CHO} + \kpp{HCHO} + \kpp{HO2} + \kpp{OH} & \code{jx(ip_CH3OOH)*EXP(LOG(1.25)*mcexp(221))} & \citet{2419}$^*$\shownote{J4309}\\
\code{J4310} & TrGNCJ &  \kpp{PR2O2HNO3} + \kpp{hv} $\rightarrow$ \kpp{NOA} + \kpp{HO2} + \kpp{OH} & \code{jx(ip_CH3OOH)*EXP(LOG(1.25)*mcexp(222))} & \citet{2419}$^*$\shownote{J4310}\\
\code{J4311} & TrGCJ &  \kpp{HOCH2COCHO} + \kpp{hv} $\rightarrow$ \kpp{HOCH2CO3} + \kpp{CO} + \kpp{HO2} & \code{jx(ip_MGLYOX)*EXP(LOG(1.25)*mcexp(223))} & \citet{2419}$^*$\shownote{J4311}\\
\code{J4400} & TrGCJ &  \kpp{LC4H9OOH} + \kpp{hv} $\rightarrow$ \kpp{OH} + 0.254 \kpp{CO2} + 0.5552 \kpp{MEK} + 0.5552 \kpp{HO2} + 0.3178 \kpp{CH3CHO} + 0.4448 \kpp{C2H5O2} & \code{jx(ip_CH3OOH)*EXP(LOG(1.25)*mcexp(224))} & \citet{2419}$^*$\shownote{J4400}\\
\code{J4401} & TrGCJ &  \kpp{MVK} + \kpp{hv} $\rightarrow$ .5 \kpp{C3H6} + .5 \kpp{CH3CO3} + .5 \kpp{HCHO} + \kpp{CO} + .5 \kpp{HO2} & \code{jx(ip_MVK)*EXP(LOG(1.25)*mcexp(225))} & see note\shownote{J4401}\\
\code{J4403} & TrGCJ &  \kpp{MEK} + \kpp{hv} $\rightarrow$ \kpp{CH3CO3} + \kpp{C2H5O2} & \code{0.42*jx(ip_CHOH)*EXP(LOG(1.25)*mcexp(226))} & \citet{1584}$^*$\shownote{J4403}\\
\code{J4404} & TrGCJ &  \kpp{LMEKOOH} + \kpp{hv} $\rightarrow$ 0.538 \kpp{HCHO} + 0.538 \kpp{CO2} + 0.459 \kpp{HOCH2CH2O2} + 0.079 \kpp{C2H5O2} + 0.462 \kpp{CH3CO3} + 0.462 \kpp{CH3CHO} + \kpp{OH} & \code{jx(ip_CH3OOH)*EXP(LOG(1.25)*mcexp(227))} & \citet{2419}$^*$\shownote{J4404}\\
\code{J4405} & TrGCJ &  \kpp{BIACET} + \kpp{hv} $\rightarrow$ 2 \kpp{CH3CO3} & \code{2.15*jx(ip_MGLYOX)*EXP(LOG(1.25)*mcexp(228))} & see note\shownote{J4405}\\
\code{J4406} & TrGNCJ &  \kpp{LC4H9NO3} + \kpp{hv} $\rightarrow$ \kpp{NO2} + 0.254 \kpp{CO2} + 0.5552 \kpp{MEK} + 0.5552 \kpp{HO2} + 0.3178 \kpp{CH3CHO} + 0.4448 \kpp{C2H5O2} & \code{J_IC3H7NO3} & see note\shownote{J4406}\\
\code{J4407} & TrGNCJ &  \kpp{MPAN} + \kpp{hv} $\rightarrow$ \kpp{MACO3} + \kpp{NO2} & \code{jx(ip_PAN)*EXP(LOG(1.25)*mcexp(229))} & see note\shownote{J4407}\\
\code{J4408} & TrGCJ &  \kpp{LMVKOHABOOH} + \kpp{hv} $\rightarrow$ .3 \kpp{HOCH2COCHO} + .3 \kpp{HCHO} + .3 \kpp{HO2} + .7 \kpp{HOCH2CHO} + .7 \kpp{HOCH2CO3} + \kpp{OH} & \code{J_ACETOL+jx(ip_CH3OOH)*EXP(LOG(1.25)*mcexp(230))} & \citet{2419}$^*$\shownote{J4408}\\
\code{J4409} & TrGCJ &  \kpp{CO2H3CO3H} + \kpp{hv} $\rightarrow$ \kpp{MGLYOX} + \kpp{HO2} + \kpp{OH} + \kpp{CO2} & \code{jx(ip_CH3OOH)*EXP(LOG(1.25)*mcexp(231))} & \citet{2419}$^*$\shownote{J4409}\\
\code{J4410} & TrGCJ &  \kpp{CO2H3CO3H} + \kpp{hv} $\rightarrow$ \kpp{CH3CO3} + \kpp{HO2} + \kpp{HCOCO3H} & \code{J_ACETOL} & \citet{2419}$^*$\shownote{J4410}\\
\code{J4411} & TrGCJ &  \kpp{MACR} + \kpp{hv} $\rightarrow$ .5 \kpp{MACO3} + .5 \kpp{CH3CO3} + .5 \kpp{HCHO} + .5 \kpp{CO} + \kpp{HO2} & \code{jx(ip_MACR)*EXP(LOG(1.25)*mcexp(232))} & see note\shownote{J4411}\\
\code{J4412} & TrGCJ &  \kpp{MACROOH} + \kpp{hv} $\rightarrow$ \kpp{ACETOL} + \kpp{HCHO} + \kpp{HO2} + \kpp{OH} & \code{jx(ip_CH3OOH)*EXP(LOG(1.25)*mcexp(233))} & \citet{2419}$^*$\shownote{J4412}\\
\code{J4413} & TrGCJ &  \kpp{MACROOH} + \kpp{hv} $\rightarrow$ \kpp{ACETOL} + \kpp{CO} + \kpp{HO2} + \kpp{OH} & \code{2.77*jx(ip_HOCH2CHO)*EXP(LOG(1.25)*mcexp(234))} & see note\shownote{J4413}\\
\code{J4414} & TrGCJ &  \kpp{MACROH} + \kpp{hv} $\rightarrow$ \kpp{ACETOL} + \kpp{CO} + \kpp{HO2} + \kpp{HO2} & \code{2.77*jx(ip_HOCH2CHO)*EXP(LOG(1.25)*mcexp(235))} & see note\shownote{J4414}\\
\code{J4415} & TrGCJ &  \kpp{MACO3H} + \kpp{hv} $\rightarrow$ \kpp{CH3CO3} + \kpp{HCHO} + \kpp{OH} + \kpp{CO2} & \code{jx(ip_CH3OOH)*EXP(LOG(1.25)*mcexp(236))} & \citet{2419}$^*$\shownote{J4415}\\
\code{J4416} & TrGCJ &  \kpp{LHMVKABOOH} + \kpp{hv} $\rightarrow$ .3 \kpp{MGLYOX} + .7 \kpp{CH3CO3} + .7 \kpp{HOCH2CHO} + .3 \kpp{HCHO} + .3 \kpp{HO2} + \kpp{OH} & \code{jx(ip_CH3OOH)*EXP(LOG(1.25)*mcexp(237))} & \citet{2419}$^*$\shownote{J4416}\\
\code{J4417} & TrGCJ &  \kpp{MVKOH} + \kpp{hv} $\rightarrow$ .5 \kpp{HCHO} + .5 \kpp{HO2} + .5 \kpp{HOCH2CO3} + \kpp{CO} + 1.5 \kpp{LCARBON} & \code{jx(ip_MVK)*EXP(LOG(1.25)*mcexp(238))} & \citet{2419}$^*$\shownote{J4417}\\
\code{J4418} & TrGCJ &  \kpp{CO2H3CHO} + \kpp{hv} $\rightarrow$ \kpp{MGLYOX} + \kpp{CO} + \kpp{HO2} + \kpp{HO2} & \code{jx(ip_HOCH2CHO)*EXP(LOG(1.25)*mcexp(239))} & \citet{2419}$^*$\shownote{J4418}\\
\code{J4419} & TrGCJ &  \kpp{HO12CO3C4} + \kpp{hv} $\rightarrow$ \kpp{CH3CO3} + \kpp{HOCH2CHO} + \kpp{HO2} & \code{J_ACETOL} & \citet{2419}$^*$\shownote{J4419}\\
\code{J4420} & TrGCJ &  \kpp{BIACETOH} + \kpp{hv} $\rightarrow$ \kpp{CH3CO3} + \kpp{HOCH2CO3} & \code{2.15*jx(ip_MGLYOX)*EXP(LOG(1.25)*mcexp(240))} & see note\shownote{J4420}\\
\code{J4502} & TrGCJ &  \kpp{LISOPACOOH} + \kpp{hv} $\rightarrow$ \kpp{LHC4ACCHO} + \kpp{HO2} + \kpp{OH} & \code{jx(ip_CH3OOH)*EXP(LOG(1.25)*mcexp(241))} & \citet{2419}$^*$\shownote{J4502}\\
\code{J4503} & TrGNCJ &  \kpp{LISOPACNO3} + \kpp{hv} $\rightarrow$ \kpp{LHC4ACCHO} + \kpp{HO2} + \kpp{NO2} & \code{0.59*J_IC3H7NO3} & see note\shownote{J4503}\\
\code{J4504} & TrGCJ &  \kpp{ISOPBOOH} + \kpp{hv} $\rightarrow$ .75 \kpp{MVK} + .25 \kpp{MVKOH} + .75 \kpp{HCHO} + .75 \kpp{HO2} + .25 \kpp{CH3O2} + \kpp{OH} & \code{jx(ip_CH3OOH)*EXP(LOG(1.25)*mcexp(242))} & \citet{2419}$^*$\shownote{J4504}\\
\code{J4505} & TrGNCJ &  \kpp{ISOPBNO3} + \kpp{hv} $\rightarrow$ .75 \kpp{MVK} + .25 \kpp{MVKOH} + .75 \kpp{HCHO} + .75 \kpp{HO2} + .25 \kpp{CH3O2} + \kpp{NO2} & \code{2.84*J_IC3H7NO3} & see note\shownote{J4505}\\
\code{J4506} & TrGCJ &  \kpp{ISOPDOOH} + \kpp{hv} $\rightarrow$ \kpp{MACR} + \kpp{HCHO} + \kpp{HO2} + \kpp{OH} & \code{jx(ip_CH3OOH)*EXP(LOG(1.25)*mcexp(243))} & \citet{2419}$^*$\shownote{J4506}\\
\code{J4507} & TrGNCJ &  \kpp{ISOPDNO3} + \kpp{hv} $\rightarrow$ \kpp{MACR} + \kpp{HCHO} + \kpp{HO2} + \kpp{NO2} & \code{J_IC3H7NO3} & see note\shownote{J4507}\\
\code{J4508} & TrGNCJ &  \kpp{NISOPOOH} + \kpp{hv} $\rightarrow$ \kpp{NC4CHO} + \kpp{HO2} + \kpp{OH} & \code{jx(ip_CH3OOH)*EXP(LOG(1.25)*mcexp(244))} & \citet{2419}$^*$\shownote{J4508}\\
\code{J4509} & TrGNCJ &  \kpp{NC4CHO} + \kpp{hv} $\rightarrow$ \kpp{NOA} + 2 \kpp{CO} + 2 \kpp{HO2} & \code{jx(ip_MACR)*EXP(LOG(1.25)*mcexp(245))} & see note\shownote{J4509}\\
\code{J4510} & TrGNCJ &  \kpp{LNISOOH} + \kpp{hv} $\rightarrow$ \kpp{NOA} + \kpp{OH} + .5 \kpp{GLYOX} + .5 \kpp{CO} + \kpp{HO2} + .5 \kpp{CO2} & \code{jx(ip_CH3OOH)*EXP(LOG(1.25)*mcexp(246))} & \citet{2272}$^*$\shownote{J4510}\\
\code{J4511} & TrGCJ &  \kpp{LHC4ACCHO} + \kpp{hv} $\rightarrow$ .5 \kpp{LHC4ACCO3} + .25 \kpp{ACETOL} + .25 \kpp{HOCH2CHO} + .25 \kpp{CH3CO3} + .75 \kpp{CO} + 1.25 \kpp{HO2} & \code{jx(ip_MACR)*EXP(LOG(1.25)*mcexp(247))} & \citet{2419}$^*$\shownote{J4511}\\
\code{J4512} & TrGCJ &  \kpp{LC578OOH} + \kpp{hv} $\rightarrow$ .5 \kpp{ACETOL} + .5 \kpp{MGLYOX} + .5 \kpp{GLYOX} + .5 \kpp{HOCH2CHO} + \kpp{HO2} + \kpp{OH} & \code{jx(ip_CH3OOH)*EXP(LOG(1.25)*mcexp(248))} & \citet{2272}$^*$\shownote{J4512}\\
\code{J4513} & TrGCJ &  \kpp{LHC4ACCO3H} + \kpp{hv} $\rightarrow$ .5 \kpp{ACETOL} + .5 \kpp{HOCH2CHO} + .5 \kpp{CH3CO3} + .5 \kpp{CO} + .5 \kpp{HO2} + \kpp{OH} + \kpp{CO2} & \code{jx(ip_CH3OOH)*EXP(LOG(1.25)*mcexp(249))} & \citet{2419}$^*$\shownote{J4513}\\
\code{J4514} & TrGNCJ &  \kpp{LC5PAN1719} + \kpp{hv} $\rightarrow$ .5 \kpp{MACROH} + .5 \kpp{HO12CO3C4} + \kpp{CO} + \kpp{NO2} & \code{jx(ip_PAN)*EXP(LOG(1.25)*mcexp(250))} & see note\shownote{J4514}\\
\code{J4515} & TrGCJ &  \kpp{HCOC5} + \kpp{hv} $\rightarrow$ \kpp{CH3CO3} + \kpp{HCHO} + \kpp{HOCH2CO3} & \code{0.5*jx(ip_MVK)*EXP(LOG(1.25)*mcexp(251))} & see note\shownote{J4515}\\
\code{J4516} & TrGCJ &  \kpp{C59OOH} + \kpp{hv} $\rightarrow$ \kpp{ACETOL} + \kpp{HOCH2CO3} + \kpp{OH} & \code{J_ACETOL+jx(ip_CH3OOH)*EXP(LOG(1.25)*mcexp(252))} & \citet{2419}$^*$\shownote{J4516}\\
\myhline
\myhline
\myhline

% this file was created automatically by eqn2tex, do not edit!

\end{longtable}

\begin{multicols}{3}
$^*$Notes:

J-values are calculated with an external module and then supplied to the
MECCA chemistry

\note{J6100}{\citet{1741} claim that the combination of absorption cross
  sections from \citet{1746} and the \chem{Cl_2O_2} formation rate
  coefficient by \citet{1555} can approximately reproduce the observed
  \chem{Cl_2O_2}/\chem{ClO} ratios and ozone depletion. They give an
  almost zenith-angle independent ratio of 1.4 for \citet{1746} to
  \citet{1555} J-values. The IUPAC recommendation for the \chem{Cl_2O_2}
  formation rate is about 5 to 15 \% less than the value by \citet{1555}
  but more than 20 \% larger than the value by \citet{1284}. The
  J-values by \citet{1746} are within the uncertainty range of the IUPAC
  recommendation.}

\note{J7301}{The quantum yields are from \citet{1555}.}

\end{multicols}

\clearpage

\begin{longtable}{llrl}
\caption{Henry's law coefficients}\\
\hline
substance & 
$\rule[-2.5ex]{0ex}{2ex}\DS\frac{k_{\rm H}^{\ominus}}{\rm M/atm}$ &
$\DS\frac{-\Delta_{\rm soln}H/R}{\rm K}$ &
reference\\
\hline 
\endfirsthead
\caption{Henry's law coefficients (... continued)}\\
\hline
substance & 
$\rule[-2.5ex]{0ex}{2ex}\DS\frac{k_{\rm H}^{\ominus}}{\rm M/atm}$ &
$\DS\frac{-\Delta_{\rm soln}H/R}{\rm K}$ &
reference\\
\hline 
\endhead
\hline
\endfoot
% this file was created automatically by henry2tex, do not edit!
\kpp{O2} & 1.3\E{-3} & 1500. & \citet{190} \\
\kpp{O3} & 1.2\E{-2} & 2560. & \citet{87} \\
\kpp{OH} & 3.0\E{1} & 4300. & \citet{515} \\
\kpp{HO2} & 3.9\E{3} & 5900. & \citet{515} \\
\kpp{H2O2} & 1.\E{5} & 6338. & \citet{311} \\
\kpp{NH3} & 58. & 4085. & \citet{87} \\
\kpp{NO} & 1.9\E{-3} & 1480. & \citet{449} \\
\kpp{NO2} & 7.0\E{-3} & 2500. & \citet{59}$^*$\showhenrynote{NO2} \\
\kpp{NO3} & 2. & 2000. & \citet{219} \\
\kpp{N2O5} & $\infty$ & 0. & see note\showhenrynote{N2O5} \\
\kpp{HONO} & 4.9\E{1} & 4780. & \citet{449} \\
\kpp{HNO3} & 2.45\E{6}/1.5\E{1} & 8694. & \citet{530}$^*$\showhenrynote{HNO3} \\
\kpp{HNO4} & 1.2\E{4} & 6900. & \citet{797} \\
\kpp{CH3O2} & 6. & 5600. & \citet{46}$^*$\showhenrynote{CH3O2} \\
\kpp{CH3OOH} & 3.0\E{2} & 5322. & \citet{311} \\
\kpp{CO2} & 3.1\E{-2} & 2423. & \citet{87} \\
\kpp{HCHO} & 7.0\E{3} & 6425. & \citet{87} \\
\kpp{HCOOH} & 3.7\E{3} & 5700. & \citet{87} \\

\end{longtable}

\begin{multicols}{3}
$^*$Notes:

The temperature dependence of the Henry constants is: $$K_{\rm H} =
K_{\rm H}^{\ominus} \times \exp \DS\left( \frac{-\Delta_{\rm soln}H}{R}
  \left( \frac{1}{T} - \frac{1}{T^{\ominus}} \right) \right)$$
where $\Delta_{\rm soln}H$~= molar enthalpy of dissolution
[\unit{J/mol}] and $R$~= 8.314 \unit{J/(mol~K)}.

\henrynote{NO2}{The temperature dependence is from \citet{87}.}

\henrynote{HNO3}{Calculated using the acidity constant from \citet{450}.}

\henrynote{CH3O2}{This value was estimated by \citet{46}.}

\henrynote{HBr}{Calculated using the acidity constant from \citet{194}.}

\henrynote{HOBr}{This value was estimated by \citet{446}.}

\henrynote{IO}{Assumed to be the same as $K_{\rm H}(\kpp{HOI})$.}

\henrynote{HOI}{Lower limit.}

\henrynote{ICl}{Calculated using thermodynamic data from \citet{489}.}

\henrynote{IBr}{Calculated using thermodynamic data from \citet{489}.}

\henrynote{H2SO4}{To account for the very high Henry's law coefficient
  of \kpp{H2SO4}, a very high value was chosen arbitrarily.}

\henrynote{DMSO}{Lower limit cited from another reference.}

\henrynote{HgBr2}{Assumed to be the same as for \kpp{HgCl2}}

\henrynote{ClHgBr}{Assumed to be the same as for \kpp{HgCl2}}

\henrynote{BrHgOBr}{Assumed to be the same as for \kpp{HgCl2}}

\henrynote{ClHgOBr}{Assumed to be the same as for \kpp{HgCl2}}

\end{multicols}

\clearpage

\begin{longtable}{llrl}
\caption{Accommodation coefficients}\\
\hline
substance & 
$\alpha^{\ominus}$ &
$\DS\frac{-\Delta_{\rm obs}H/R}{\rm K}$ &
reference\\
\hline 
\endfirsthead
\caption{Accommodation coefficients (... continued)}\\
\hline
substance & 
$\alpha^{\ominus}$ &
$\DS\frac{-\Delta_{\rm obs}H/R}{\rm K}$ &
reference\\
\hline 
\endhead
\hline
\endfoot 
% this file was created automatically by alpha2tex, do not edit!
\kpp{O2} & 0.01 & 2000. & see note\showalphanote{O2} \\
\kpp{O3} & 0.002 & 0. & \citet{826}$^*$\showalphanote{O3} \\
\kpp{OH} & 0.01 & 0. & \citet{1047}$^*$\showalphanote{OH} \\
\kpp{HO2} & 0.5 & 0. & \citet{1864} \\
\kpp{H2O2} & 0.077 & 3127. & \citet{32} \\
\kpp{NH3} & 0.06 & 0. & \citet{826}$^*$\showalphanote{NH3} \\
\kpp{NO} & 5.0\E{-5} & 0. & \citet{448}$^*$\showalphanote{NO} \\
\kpp{NO2} & 0.0015 & 0. & \citet{176}$^*$\showalphanote{NO2} \\
\kpp{NO3} & 0.04 & 0. & \citet{1048}$^*$\showalphanote{NO3} \\
\kpp{N2O5} & 0.1 & 0. & \citet{826}$^*$\showalphanote{N2O5} \\
\kpp{HONO} & 0.04 & 0. & \citet{826}$^*$\showalphanote{HONO} \\
\kpp{HNO3} & 0.5 & 0. & \citet{930}$^*$\showalphanote{HNO3} \\
\kpp{HNO4} & 0.1 & 0. & \citet{826}$^*$\showalphanote{HNO4} \\
\kpp{CH3O2} & 0.01 & 2000. & see note\showalphanote{CH3O2} \\
\kpp{CH3OOH} & 0.0046 & 3273. & \citet{844} \\
\kpp{HCHO} & 0.04 & 0. & \citet{826}$^*$\showalphanote{HCHO} \\
\kpp{HCOOH} & 0.014 & 3978. & \citet{826} \\
\kpp{CO2} & 0.01 & 2000. & see note\showalphanote{CO2} \\

\end{longtable}

\begin{multicols}{3}
$^*$Notes:

The temperature dependence of the accommodation coefficients is given by
\citep{155}:

\begin{eqnarray*}
  \frac{\alpha}{1-\alpha} & = & \exp \left( \frac{-\Delta_{\rm obs}G}{RT}
  \right)\\
  & = & \exp \left( \frac{-\Delta_{\rm obs}H}{RT} + \frac{\Delta_{\rm
  obs}S}{R} \right)
\end{eqnarray*}

where $\Delta_{\rm obs}G$ is the Gibbs free energy barrier of the
transition state toward solution \citep{155}, and $\Delta_{\rm obs}H$
and $\Delta_{\rm obs}S$ are the corresponding enthalpy and entropy,
respectively. The equation can be rearranged to:

\begin{eqnarray*}
  \ln\left( \frac{\alpha}{1-\alpha} \right) & = & 
  \frac{-\Delta_{\rm obs}H}{R} \times \frac{1}{T} + \frac{-\Delta_{\rm
  obs}S}{R}\\
\end{eqnarray*}

and further:

\begin{eqnarray*}
  \dd\ln\left( \frac{\alpha}{1-\alpha} \right) \left/ \dd\left(
    \frac{1}{T} \right)\right. & = & \frac{-\Delta_{\rm obs}H}{R}
\end{eqnarray*}

If no data were available, a value of $\alpha$ = 0.01, $\alpha$ = 0.1,
or $\alpha$ = 0.5, and a temperature dependence of $-\Delta_{\rm
  obs}H/R$~= 2000~\unit{K} has been assumed.

\alphanote{O2}{Estimate.}

\alphanote{O3}{Value measured at 292 \unit{K}.}

\alphanote{OH}{Value measured at 293 \unit{K}.}

\alphanote{HO2}{Value for aqueous salts at 293 \unit{K}.}

\alphanote{NH3}{Value measured at 295 \unit{K}.}

\alphanote{NO}{Value measured between 193 and 243 \unit{K}.}

\alphanote{NO2}{Value measured at 298 \unit{K}.}

\alphanote{NO3}{Value is a lower limit, measured at 273 \unit{K}.}

\alphanote{N2O5}{Value for sulfuric acid, measured between 195 and 300
\unit{K}.}

\alphanote{HONO}{Value measured between 247 and 297 \unit{K}.}

\alphanote{HNO3}{Value measured at room temperature. \citet{930} say
$\gamma>0.2$. Here $\alpha=0.5$ is used.}

\alphanote{HNO4}{Value measured at 200 \unit{K} for water ice.}

\alphanote{CH3O2}{Estimate.}

\alphanote{HCHO}{Value measured between 260 and 270 \unit{K}.}

\alphanote{CO2}{Estimate.}

\alphanote{HCl}{Temperature dependence derived from published data at 2
different temperatures}

\alphanote{HOCl}{Assumed to be the same as $\alpha(\kpp{HOBr})$.}

\alphanote{ClNO3}{Value measured at 274.5 \unit{K}.}

\alphanote{HBr}{Temperature dependence derived from published data at 2
different temperatures}

\alphanote{HOBr}{Value measured at room temperature. \citet{930} say
$\gamma>0.2$. Here $\alpha=0.5$ is used.}

\alphanote{BrNO3}{Value measured at 273 \unit{K}.}

\alphanote{BrCl}{Assumed to be the same as $\alpha(\kpp{Cl2})$.}

\alphanote{I2}{Estimate.}

\alphanote{IO}{Estimate.}

\alphanote{OIO}{Estimate.}

\alphanote{I2O2}{Estimate.}

\alphanote{HI}{Temperature dependence derived from published data at 2
different temperatures}

\alphanote{HOI}{Assumed to be the same as $\alpha(\kpp{HOBr})$. See also
  \citet{1565} and \citet{1572}.}

\alphanote{HIO3}{Estimate.}

\alphanote{INO2}{Estimate.}

\alphanote{INO3}{Estimate.}

\alphanote{ICl}{Estimate.}

\alphanote{IBr}{Assumed to be the same as $\alpha(\kpp{ICl})$.}

\alphanote{H2SO4}{Value measured at 303 \unit{K}.}

\alphanote{Hg}{Estimate.}

\alphanote{HgO}{Estimate.}

\alphanote{HgCl2}{Estimate.}

\alphanote{HgBr2}{Estimate.}

\alphanote{ClHgBr}{Estimate.}

\alphanote{BrHgOBr}{Estimate.}

\alphanote{ClHgOBr}{Estimate.}

\end{multicols}

\clearpage

\begin{longtable}{llp{8cm}p{5cm}p{55mm}}
\caption{Henry's law equilibria}\\
\hline
\# & labels & reaction & rate coefficient & reference\\
\hline
\endhead
\hline
\endfoot
% this file was created automatically by eqn2tex, do not edit!

\end{longtable}

\begin{multicols}{3}
$^*$Notes:

The forward (\verb|k_exf|) and backward (\verb|k_exb|) rate coefficients
are calculated in the file \verb|messy_mecca_aero.f90| using the
accommodation coefficients in subroutine \verb|mecca_aero_alpha| and
Henry's law constants in subroutine \verb|mecca_aero_henry|.

$k_{\rm mt}$ = mass transfer coefficient

$\rm lwc$ = liquid water content of aerosol mode

H3201, H6300, H6301, H6302, H7300, H7301, H7302, H7601, H7602: For
uptake of $X$ (= \chem{N_2O_5}, \chem{ClNO_3}, \chem{BrNO_3}) and
subsequent reaction with \chem{H_2O}, \chem{Cl^-}, and \chem{Br^-}, we
define $k_{\rm exf}(X) = k_{\rm mt}(X)\times lwc / ( [H_2O] + 5.0E2
[Cl^-] + 3.0E5 [Br^-] )$.

H6301, H6302, H7601: The total uptake is determined by $k_{\rm
  mt}(\chem{ClNO_3})$. The relative rates are assumed to be the same as
for \chem{N_2O_5} (H3201, H6300, H7300).

H7301, H7302, H7602: The total uptake is determined by $k_{\rm
  mt}(\chem{BrNO_3})$. The relative rates are assumed to be the same as
for \chem{N_2O_5} (H3201, H6300, H7300).

\end{multicols}

\clearpage

\begin{longtable}{llp{9cm}p{7cm}p{5cm}}
\caption{Heterogeneous reactions}\\
\hline
\# & labels & reaction & rate coefficient & reference\\
\hline
\endfirsthead
\caption{Heterogeneous reactions (... continued)}\\
\hline
\# & labels & reaction & rate coefficient & reference\\
\hline
\endhead
\hline
\endfoot
% this file was created automatically by eqn2tex, do not edit!

\end{longtable}

$^*$Notes:

Heterogeneous reaction rates are calculated with an external module and
then supplied to the MECCA chemistry (see \url{www.messy-interface.org}
for details)

\clearpage

\def\aq{}
\begin{longtable}{llp{7cm}p{3cm}p{25mm}p{6cm}}
\caption{Acid-base and other eqilibria}\\
\hline
\# & labels & reaction & $K_0[M^{m-n}]$ & -$\Delta H / R [K]$ & reference\\
\hline
\endhead
\hline
\endfoot
% this file was created automatically by eqn2tex, do not edit!

\end{longtable}

\begin{multicols}{2}
$^*$Notes: 

\note{EQ40}{For $pK_a$(\chem{CO_2}), see also \citet{1777}.}

\note{EQ72}{For $pK_a$(\chem{HOBr}), see also \citet{1778}.}

\note{EQ82}{Thermodynamic calculations on the \chem{IBr}/\chem{ICl}
  equilibrium according to the data tables from \citet{489}:\\
  \begin{tabular}{ccccccc}
    \chem{ICl} & + & \chem{Br^-} & $\rightleftharpoons$ &
    \chem{IBr} & + & \chem{Cl^-}\\
    -17.1 & & -103.96 & = & -4.2 & & -131.228
  \end{tabular}
  $$\frac{\Delta G}{[\unit{kJ/mol}]} = -4.2 - 131.228 - (-17.1 - 103.96)
  = -14.368$$
  $$K = \frac{[\chem{IBr}] \times [\chem{Cl^-}]}{[\chem{ICl}] \times
    [\chem{Br^-}]} = \exp\left(\frac{-\Delta G}{RT}\right) =
  \exp\left(\frac{14368}{8.314\times 298}\right) = 330$$
  This means we have equal amounts of \chem{IBr} and \chem{ICl} when the
  [\chem{Cl^-}]/[\chem{Br^-}] ratio equals 330.}

\end{multicols}

\clearpage

\def\aq{}
\begin{longtable}{llp{8cm}p{3cm}p{25mm}p{5cm}}
\caption{Aqueous phase reactions}\\
\hline
\# & labels & reaction & $k_0~[M^{1-n}s^{-1}]$ & $-E_a / R [K] $& reference\\
\hline
\endfirsthead
\caption{Aqueous phase reactions (...continued)}\\
\hline
\# & labels & reaction & $k_0~[M^{1-n}s^{-1}]$ & $-E_a / R [K] $& reference\\
\hline
\endhead
\hline
\endfoot
% this file was created automatically by eqn2tex, do not edit!

\end{longtable}

\begin{multicols}{3}
$^*$Notes:

\note{A6102}{\citet{1008} found an upper limit of \code{6E9} and cite an
  upper limit from another study of \code{2E9}. Here, we set the rate
  coefficient to \code{1E9}.}

\note{A6301}{There is also an earlier study by \citet{69} which found a
  smaller rate coefficient but did not consider the back reaction.}

\note{A7400}{Assumed to be the same as for \chem{Br_2^-} + \chem{H_2O_2}.}

\note{A9105}{The rate coefficient for the sum of the paths (leading to
  either \chem{HSO_5^-} or \chem{SO_4^{2-}}) is from \citet{75}, the
  ratio 0.28/0.72 is from \citet{111}.}

\note{A9106}{See also: \citep{75,1}. If this reaction produces a lot of
  \chem{SO_4^-}, it will have an effect. However, we currently assume
  only the stable \chem{S_2O_8^{2-}} as product. Since
  \chem{S_2O_8^{2-}} is not treated explicitly in the mechanism, we use
  \chem{SO_4^{2-}} as a proxy. Note that this destroys the mass
  consistency for sulfur species.}

\note{A9205}{D.\ Sedlak, pers.\ comm.\ (1993).}

\note{A9208}{D.\ Sedlak, pers.\ comm.\ (1993).}

\note{A9605}{assumed to be the same as for \chem{SO_3^{2-}} +
  \chem{HOCl}.}

\note{A9705}{assumed to be the same as for \chem{SO_3^{2-}} +
  \chem{HOBr}.}

\end{multicols}

\clearpage

\begin{multicols}{3}
\bibliographystyle{egu} % bst file
\bibliography{meccalit} % bib files
\end{multicols}

\end{document}
