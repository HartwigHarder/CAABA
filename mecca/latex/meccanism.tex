\documentclass[landscape]{article}

\makeatletter
\def\shownote#1{\expandafter\gdef\csname note#1\endcsname{}}%
\def\note#1#2{\@ifundefined{note#1}{}{#1: #2}}%
\def\showhenrynote#1{\expandafter\gdef\csname henrynote#1\endcsname{}}%
\def\henrynote#1#2{\@ifundefined{henrynote#1}{}{\kpp{#1}: #2}}%
\def\showalphanote#1{\expandafter\gdef\csname alphanote#1\endcsname{}}%
\def\alphanote#1#2{\@ifundefined{alphanote#1}{}{\kpp{#1}: #2}}%
\makeatother

\def\myhline{\ifvmode\hline\fi}
\def\aq{(aq)}

\usepackage{multicol}
\usepackage{longtable}
\usepackage{natbib}
\usepackage{chem}
\usepackage{rotating} % loads graphics
%\usepackage{color}

\textwidth26cm
\textheight16cm
\topmargin-10mm
\oddsidemargin-15mm
\parindent0mm
\parskip1.0ex plus0.5ex minus0.5ex

\setlength{\LTcapwidth}{\textwidth}

\begin{document}

% this file was created automatically by ncv, do not edit
\def\meccaversion{2.5m}


\thispagestyle{empty}
\begin{rotate}{-90}
\begin{minipage}{15cm}
\vspace{-30cm}
\begin{center}
  \LARGE {\bf The Chemical Mechanism of MECCA}\\[3mm]
  \Large KPP version: {\kppversion}\\[2mm]
  \Large MECCA version: {\meccaversion}\\[2mm]
  \Large Date: \today.\\[2mm]
  \Large Selected reactions:\\
  ``\wanted''\\[2mm]
  Number of aerosol phases: \apn\\[2mm]
  Number of species in selected mechanism:\\
  \begin{tabular}{lr}
  Gas phase:     & \gasspc\\
  Aqueous phase: & \aqspc\\
  All species:   & \allspc\\
  \end{tabular}\\[2mm]
  Number of reactions in selected mechanism:\\
  \begin{tabular}{lr}
    Gas phase (Gnnn):         & \Geqns\\
    Aqueous phase (Annn):     & \Aeqns\\
    Henry (Hnnn):             & \Heqns\\
    Photolysis (Jnnn):        & \Jeqns\\
    Heterogeneous (HETnnn):   & \HETeqns\\
    Equilibria (EQnn):        & \EQeqns\\
    Isotope exchange (IEXnnn):& \IEXeqns\\
    Dummy (Dnn):              & \Deqns\\
    All equations:            & \alleqns
  \end{tabular}\\[30mm]
  Further information can be found in the article ``Technical Note: The
  new comprehensive atmospheric chemistry module MECCA'' by R.\ Sander
  et al. (Atmos.\ Chem.\ Phys.\ {\bf 5}, 445-450, 2005), available at
  \url{http://www.atmos-chem-phys.net/5/445}.
\end{center}
\end{minipage}
\end{rotate}
\newpage

% define latex names of kpp species
% this file was created automatically by spc2tex, do not edit!
\makeatletter
\def\defkpp#1#2{\expandafter\def\csname #1\endcsname{\chem{#2}}}%
\def\kpp#1{\@ifundefined{#1}{\errmessage{#1 undefined}}{\csname #1\endcsname}}%
\defkpp{hv}{h\nu}%
\defkpp{PROD}{products}%
\defkpp{O1D}{O(^1D)}%
\defkpp{O3P}{O(^3P)}%
\defkpp{O2}{O_2}%
\defkpp{O3}{O_3}%
\defkpp{H}{H}%
\defkpp{H2}{H_2}%
\defkpp{OH}{OH}%
\defkpp{HO2}{HO_2}%
\defkpp{H2O}{H_2O}%
\defkpp{H2O2}{H_2O_2}%
\defkpp{N}{N}%
\defkpp{N2}{N_2}%
\defkpp{NH3}{NH_3}%
\defkpp{N2O}{N_2O}%
\defkpp{NO}{NO}%
\defkpp{NO2}{NO_2}%
\defkpp{NO3}{NO_3}%
\defkpp{N2O5}{N_2O_5}%
\defkpp{HONO}{HONO}%
\defkpp{HNO3}{HNO_3}%
\defkpp{HNO4}{HNO_4}%
\defkpp{NH2}{NH_2}%
\defkpp{HNO}{HNO}%
\defkpp{NHOH}{NHOH}%
\defkpp{NH2O}{NH_2O}%
\defkpp{NH2OH}{NH_2OH}%
\defkpp{CH3O2}{CH_3O_2}%
\defkpp{CH3OH}{CH_3OH}%
\defkpp{CH3OOH}{CH_3OOH}%
\defkpp{CH4}{CH_4}%
\defkpp{CO}{CO}%
\defkpp{CO2}{CO_2}%
\defkpp{HCHO}{HCHO}%
\defkpp{HCOOH}{HCOOH}%
\defkpp{LCARBON}{LCARBON}%
\defkpp{C2H2}{C_2H_2}%
\defkpp{C2H4}{C_2H_4}%
\defkpp{C2H5O2}{C_2H_5O_2}%
\defkpp{C2H5OOH}{C_2H_5OOH}%
\defkpp{C2H6}{C_2H_6}%
\defkpp{CH3CHO}{CH_3CHO}%
\defkpp{CH3CO2H}{CH_3COOH}%
\defkpp{CH3CO3}{CH_3C(O)OO}%
\defkpp{CH3CO3H}{CH_3C(O)OOH}%
\defkpp{ETHGLY}{ETHGLY}%
\defkpp{ETHOHNO3}{ETHOHNO3}%
\defkpp{GLYOX}{GLYOX}%
\defkpp{HCOCO2H}{HCOCO_2H}%
\defkpp{HCOCO3}{HCOCO_3}%
\defkpp{HCOCO3H}{HCOCO_3H}%
\defkpp{HOCH2CH2O}{HOCH_2CH_2O}%
\defkpp{HOCH2CH2O2}{HOCH_2CH_2O_2}%
\defkpp{HOCH2CHO}{HOCH_2CHO}%
\defkpp{HOCH2CO2H}{HOCH_2CO_2H}%
\defkpp{HOCH2CO3}{HOCH_2CO_3}%
\defkpp{HOCH2CO3H}{HOCH_2CO_3H}%
\defkpp{HYETHO2H}{HYETHO2H}%
\defkpp{PAN}{PAN}%
\defkpp{PHAN}{PHAN}%
\defkpp{ACETOL}{CH_3COCH_2OH}%
\defkpp{C3H6}{C_3H_6}%
\defkpp{C3H8}{C_3H_8}%
\defkpp{CH3COCH2O2}{CH_3COCH_2O_2}%
\defkpp{CH3COCH3}{CH_3COCH_3}%
\defkpp{HOCH2COCHO}{HOCH2COCHO}%
\defkpp{HOCH2COCO2H}{HOCH2COCO2H}%
\defkpp{HYPERACET}{CH_3COCH_2O_2H}%
\defkpp{HYPROPO2}{HYPROPO2}%
\defkpp{HYPROPO2H}{HYPROPO2H}%
\defkpp{IC3H7NO3}{iC_3H_7ONO_2}%
\defkpp{IC3H7O2}{iC_3H_7O_2}%
\defkpp{IC3H7OOH}{iC_3H_7OOH}%
\defkpp{MGLYOX}{MGLYOX}%
\defkpp{NOA}{NOA}%
\defkpp{PR2O2HNO3}{PR2O2HNO3}%
\defkpp{PRONO3BO2}{PRONO3BO2}%
\defkpp{BIACET}{BIACET}%
\defkpp{BIACETOH}{BIACETOH}%
\defkpp{CO2H3CHO}{CO2H3CHO}%
\defkpp{CO2H3CO3}{CO2H3CO3}%
\defkpp{CO2H3CO3H}{CO2H3CO3H}%
\defkpp{HO12CO3C4}{HO12CO3C4}%
\defkpp{LC4H9NO3}{LC4H9NO3}%
\defkpp{LC4H9O2}{LC_4H_9O_2}%
\defkpp{LC4H9OOH}{LC_4H_9OOH}%
\defkpp{LHMVKABO2}{LHMVKABO2}%
\defkpp{LHMVKABOOH}{LHMVKABOOH}%
\defkpp{LMVKOHABO2}{LMVKOHABO2}%
\defkpp{LMVKOHABOOH}{LMVKOHABOOH}%
\defkpp{MACO2H}{MACO2H}%
\defkpp{MACO3}{MACO3}%
\defkpp{MACO3H}{MACO3H}%
\defkpp{MACR}{MACR}%
\defkpp{MACRO2}{MACRO2}%
\defkpp{MACROH}{MACROH}%
\defkpp{MACROOH}{MACROOH}%
\defkpp{MEK}{MEK}%
\defkpp{LMEKO2}{LMEKO2}%
\defkpp{LMEKOOH}{LMEKOOH}%
\defkpp{MPAN}{MPAN}%
\defkpp{MVK}{MVK}%
\defkpp{MVKOH}{MVKOH}%
\defkpp{NC4H10}{nC_4H_{10}}%
\defkpp{C59O2}{C59O2}%
\defkpp{C59OOH}{C59OOH}%
\defkpp{C5H8}{C_5H_8}%
\defkpp{HCOC5}{HCOC5}%
\defkpp{ISOPAOH}{ISOPAOH}%
\defkpp{ISOPBNO3}{ISOPBNO3}%
\defkpp{ISOPBO2}{ISOPBO2}%
\defkpp{ISOPBOH}{ISOPBOH}%
\defkpp{ISOPBOOH}{ISOPBOOH}%
\defkpp{ISOPDNO3}{ISOPDNO3}%
\defkpp{ISOPDO2}{ISOPDO2}%
\defkpp{ISOPDOH}{ISOPDOH}%
\defkpp{ISOPDOOH}{ISOPDOOH}%
\defkpp{LC578O2}{LC578O2}%
\defkpp{LC578OOH}{LC578OOH}%
\defkpp{LC5PAN1719}{LC5PAN1719}%
\defkpp{LHC4ACCHO}{LHC4ACCHO}%
\defkpp{LHC4ACCO2H}{LHC4ACCO2H}%
\defkpp{LHC4ACCO3}{LHC4ACCO3}%
\defkpp{LHC4ACCO3H}{LHC4ACCO3H}%
\defkpp{LISOPACNO3}{LISOPACNO3}%
\defkpp{LISOPACO2}{LISOPACO2}%
\defkpp{LISOPACOOH}{LISOPACOOH}%
\defkpp{LNISO3}{LNISO3}%
\defkpp{LNISOOH}{LNISOOH}%
\defkpp{NC4CHO}{NC4CHO}%
\defkpp{NISOPO2}{NISOPO2}%
\defkpp{NISOPOOH}{NISOPOOH}%
\defkpp{Cl}{Cl}%
\defkpp{Cl2}{Cl_2}%
\defkpp{ClO}{ClO}%
\defkpp{HCl}{HCl}%
\defkpp{HOCl}{HOCl}%
\defkpp{Cl2O2}{Cl_2O_2}%
\defkpp{OClO}{OClO}%
\defkpp{ClNO2}{ClNO_2}%
\defkpp{ClNO3}{ClNO_3}%
\defkpp{CCl4}{CCl_4}%
\defkpp{CH3Cl}{CH_3Cl}%
\defkpp{CH3CCl3}{CH_3CCl_3}%
\defkpp{CF2Cl2}{CF_2Cl_2}%
\defkpp{CFCl3}{CFCl_3}%
\defkpp{Br}{Br}%
\defkpp{Br2}{Br_2}%
\defkpp{BrO}{BrO}%
\defkpp{HBr}{HBr}%
\defkpp{HOBr}{HOBr}%
\defkpp{BrNO2}{BrNO_2}%
\defkpp{BrNO3}{BrNO_3}%
\defkpp{BrCl}{BrCl}%
\defkpp{CH3Br}{CH_3Br}%
\defkpp{CF3Br}{CF_3Br}%
\defkpp{CF2ClBr}{CF_2ClBr}%
\defkpp{CHCl2Br}{CHCl_2Br}%
\defkpp{CHClBr2}{CHClBr_2}%
\defkpp{CH2ClBr}{CH_2ClBr}%
\defkpp{CH2Br2}{CH_2Br_2}%
\defkpp{CHBr3}{CHBr_3}%
\defkpp{I}{I}%
\defkpp{I2}{I_2}%
\defkpp{IO}{IO}%
\defkpp{OIO}{OIO}%
\defkpp{I2O2}{I_2O_2}%
\defkpp{HI}{HI}%
\defkpp{HOI}{HOI}%
\defkpp{HIO3}{HIO_3}%
\defkpp{INO2}{INO_2}%
\defkpp{INO3}{INO_3}%
\defkpp{CH3I}{CH_3I}%
\defkpp{CH2I2}{CH_2I_2}%
\defkpp{C3H7I}{C_3H_7I}%
\defkpp{ICl}{ICl}%
\defkpp{CH2ClI}{CH_2ClI}%
\defkpp{IBr}{IBr}%
\defkpp{S}{S}%
\defkpp{SO}{SO}%
\defkpp{SO2}{SO_2}%
\defkpp{SH}{SH}%
\defkpp{H2SO4}{H_2SO_4}%
\defkpp{CH3SO3H}{CH_3SO_3H}%
\defkpp{DMS}{DMS}%
\defkpp{DMSO}{DMSO}%
\defkpp{CH3SO2}{CH_3SO_2}%
\defkpp{CH3SO3}{CH_3SO_3}%
\defkpp{OCS}{OCS}%
\defkpp{SF6}{SF_6}%
\defkpp{Hg}{Hg}%
\defkpp{HgO}{HgO}%
\defkpp{HgCl}{HgCl}%
\defkpp{HgCl2}{HgCl_2}%
\defkpp{HgBr}{HgBr}%
\defkpp{HgBr2}{HgBr_2}%
\defkpp{ClHgBr}{ClHgBr}%
\defkpp{BrHgOBr}{BrHgOBr}%
\defkpp{ClHgOBr}{ClHgOBr}%
\defkpp{NO3m_cs}{NO_3^-\aq}%
\defkpp{Hp_cs}{H^+\aq}%
\defkpp{RGM_cs}{Hg\aq}%
\defkpp{IPART}{I_{part}}%
\defkpp{Dummy}{Dummy}%
\defkpp{O3s}{O_3(s)}%
\defkpp{LO3s}{LO_3(s)}%
\defkpp{LHOC3H6O2}{CH_3CH(O_2)CH_2OH}%
\defkpp{LHOC3H6OOH}{CH_3CH(OOH)CH_2OH}%
\defkpp{ISO2}{ISO2}%
\defkpp{ISON}{ISON}%
\defkpp{ISOOH}{ISOOH}%
\defkpp{MVKO2}{MVKO2}%
\defkpp{MVKOOH}{MVKOOH}%
\defkpp{NACA}{NACA}%
\defkpp{O2_a01}{O_2\aq}%
\defkpp{O3_a01}{O_3\aq}%
\defkpp{OH_a01}{OH\aq}%
\defkpp{HO2_a01}{HO_2\aq}%
\defkpp{H2O_a01}{H_2O\aq}%
\defkpp{H2O2_a01}{H_2O_2\aq}%
\defkpp{NH3_a01}{NH_3\aq}%
\defkpp{NO_a01}{NO\aq}%
\defkpp{NO2_a01}{NO_2\aq}%
\defkpp{NO3_a01}{NO_3\aq}%
\defkpp{HONO_a01}{HONO\aq}%
\defkpp{HNO3_a01}{HNO_3\aq}%
\defkpp{HNO4_a01}{HNO_4\aq}%
\defkpp{N2O5_a01}{N_2O_5\aq}%
\defkpp{CH3OH_a01}{CH_3OH\aq}%
\defkpp{HCOOH_a01}{HCOOH\aq}%
\defkpp{HCHO_a01}{HCHO\aq}%
\defkpp{CH3O2_a01}{CH_3OO\aq}%
\defkpp{CH3OOH_a01}{CH_3OOH\aq}%
\defkpp{CO2_a01}{CO_2\aq}%
\defkpp{CH3CO2H_a01}{CH_3COOH\aq}%
\defkpp{PAN_a01}{PAN\aq}%
\defkpp{C2H5O2_a01}{C_2H_5O_2\aq}%
\defkpp{CH3CHO_a01}{CH_3CHO\aq}%
\defkpp{CH3COCH3_a01}{CH_3COCH_3\aq}%
\defkpp{Cl_a01}{Cl\aq}%
\defkpp{Cl2_a01}{Cl_2\aq}%
\defkpp{HCl_a01}{HCl\aq}%
\defkpp{HOCl_a01}{HOCl\aq}%
\defkpp{Br_a01}{Br\aq}%
\defkpp{Br2_a01}{Br_2\aq}%
\defkpp{HBr_a01}{HBr\aq}%
\defkpp{HOBr_a01}{HOBr\aq}%
\defkpp{BrCl_a01}{BrCl\aq}%
\defkpp{I2_a01}{I_2\aq}%
\defkpp{IO_a01}{IO\aq}%
\defkpp{HI_a01}{HI\aq}%
\defkpp{HOI_a01}{HOI\aq}%
\defkpp{ICl_a01}{ICl\aq}%
\defkpp{IBr_a01}{IBr\aq}%
\defkpp{HIO3_a01}{HIO_3\aq}%
\defkpp{SO2_a01}{SO_2\aq}%
\defkpp{H2SO4_a01}{H_2SO_4\aq}%
\defkpp{DMS_a01}{DMS\aq}%
\defkpp{DMSO_a01}{DMSO\aq}%
\defkpp{Hg_a01}{Hg\aq}%
\defkpp{HgO_a01}{HgO\aq}%
\defkpp{HgOH_a01}{HgOH\aq}%
\defkpp{HgOHOH_a01}{Hg(OH)_2\aq}%
\defkpp{HgOHCl_a01}{Hg(OH)Cl\aq}%
\defkpp{HgCl2_a01}{HgCl_2\aq}%
\defkpp{HgBr2_a01}{HgBr_2\aq}%
\defkpp{HgSO3_a01}{HgSO_3\aq}%
\defkpp{ClHgBr_a01}{ClHgBr\aq}%
\defkpp{BrHgOBr_a01}{BrHgOBr\aq}%
\defkpp{ClHgOBr_a01}{ClHgOBr\aq}%
\defkpp{O2m_a01}{O_2^-\aq}%
\defkpp{OHm_a01}{OH^-\aq}%
\defkpp{Hp_a01}{H^+\aq}%
\defkpp{NH4p_a01}{NH_4^+\aq}%
\defkpp{NO2m_a01}{NO_2^-\aq}%
\defkpp{NO3m_a01}{NO_3^-\aq}%
\defkpp{NO4m_a01}{NO_4^-\aq}%
\defkpp{CO3m_a01}{CO_3^-\aq}%
\defkpp{HCOOm_a01}{HCOO^-\aq}%
\defkpp{HCO3m_a01}{HCO_3^-\aq}%
\defkpp{CH3COOm_a01}{CH_3COO^-\aq}%
\defkpp{Clm_a01}{Cl^-\aq}%
\defkpp{Cl2m_a01}{Cl_2^-\aq}%
\defkpp{ClOm_a01}{ClO^-\aq}%
\defkpp{ClOHm_a01}{ClOH^-\aq}%
\defkpp{Brm_a01}{Br^-\aq}%
\defkpp{Br2m_a01}{Br_2^-\aq}%
\defkpp{BrOm_a01}{BrO^-\aq}%
\defkpp{BrOHm_a01}{BrOH^-\aq}%
\defkpp{BrCl2m_a01}{BrCl_2^-\aq}%
\defkpp{Br2Clm_a01}{Br_2Cl^-\aq}%
\defkpp{Im_a01}{I^-\aq}%
\defkpp{IO2m_a01}{IO_2^-\aq}%
\defkpp{IO3m_a01}{IO_3^-\aq}%
\defkpp{ICl2m_a01}{ICl_2^-\aq}%
\defkpp{IClBrm_a01}{IClBr^-\aq}%
\defkpp{IBr2m_a01}{IBr_2^-\aq}%
\defkpp{SO3m_a01}{SO_3^-\aq}%
\defkpp{SO3mm_a01}{SO_3^{2-}\aq}%
\defkpp{SO4m_a01}{SO_4^-\aq}%
\defkpp{SO4mm_a01}{SO_4^{2-}\aq}%
\defkpp{SO5m_a01}{SO_5^-\aq}%
\defkpp{HSO3m_a01}{HSO_3^-\aq}%
\defkpp{HSO4m_a01}{HSO_4^-\aq}%
\defkpp{HSO5m_a01}{HSO_5^-\aq}%
\defkpp{CH3SO3m_a01}{CH_3SO_3^-\aq}%
\defkpp{CH2OHSO3m_a01}{CH_2OHSO_3^-\aq}%
\defkpp{Hgp_a01}{Hg^+\aq}%
\defkpp{Hgpp_a01}{Hg^{2+}\aq}%
\defkpp{HgOHp_a01}{HgOH^+\aq}%
\defkpp{HgClp_a01}{HgCl^+\aq}%
\defkpp{HgCl3m_a01}{HgCl_3^-\aq}%
\defkpp{HgCl4mm_a01}{HgCl_4^{2-}\aq}%
\defkpp{HgBrp_a01}{HgBr^+\aq}%
\defkpp{HgBr3m_a01}{HgBr_3^-\aq}%
\defkpp{HgBr4mm_a01}{HgBr_4^{2-}\aq}%
\defkpp{HgSO32mm_a01}{Hg(SO_3)_2^{2-}\aq}%
\defkpp{D1O_a01}{D_1O\aq}%
\defkpp{D2O_a01}{D_2O\aq}%
\defkpp{DAHp_a01}{DAH^+\aq}%
\defkpp{DA_a01}{DA\aq}%
\defkpp{DAm_a01}{DA^-\aq}%
\defkpp{DGtAi_a01}{DGtAi\aq}%
\defkpp{DGtAs_a01}{DGtAs\aq}%
\defkpp{PROD1_a01}{PROD1\aq}%
\defkpp{PROD2_a01}{PROD2\aq}%
\defkpp{Nap_a01}{Na^+\aq}%
\defkpp{RRG1000}{RRG1000}%
\defkpp{RRG1001}{RRG1001}%
\defkpp{RRG2100}{RRG2100}%
\defkpp{RRG2104}{RRG2104}%
\defkpp{RRG2105}{RRG2105}%
\defkpp{RRG2107}{RRG2107}%
\defkpp{RRG2109}{RRG2109}%
\defkpp{RRG2110}{RRG2110}%
\defkpp{RRG2111}{RRG2111}%
\defkpp{RRG2112}{RRG2112}%
\defkpp{RRG3101}{RRG3101}%
\defkpp{RRG3103}{RRG3103}%
\defkpp{RRG3106}{RRG3106}%
\defkpp{RRG3108}{RRG3108}%
\defkpp{RRG3109}{RRG3109}%
\defkpp{RRG3110}{RRG3110}%
\defkpp{RRG3200}{RRG3200}%
\defkpp{RRG3201}{RRG3201}%
\defkpp{RRG3202}{RRG3202}%
\defkpp{RRG3203}{RRG3203}%
\defkpp{RRG3204}{RRG3204}%
\defkpp{RRG3205}{RRG3205}%
\defkpp{RRG3206}{RRG3206}%
\defkpp{RRG3207}{RRG3207}%
\defkpp{RRG3208}{RRG3208}%
\defkpp{RRG3209}{RRG3209}%
\defkpp{RRG3210}{RRG3210}%
\defkpp{RRG3211}{RRG3211}%
\defkpp{RRG3212}{RRG3212}%
\defkpp{RRG3213}{RRG3213}%
\defkpp{RRG3214}{RRG3214}%
\defkpp{RRG3215}{RRG3215}%
\defkpp{RRG3216}{RRG3216}%
\defkpp{RRG3217}{RRG3217}%
\defkpp{RRG3218}{RRG3218}%
\defkpp{RRG3219}{RRG3219}%
\defkpp{RRG3220}{RRG3220}%
\defkpp{RRG3221}{RRG3221}%
\defkpp{RRG3222}{RRG3222}%
\defkpp{RRG3223}{RRG3223}%
\defkpp{RRG3224}{RRG3224}%
\defkpp{RRG4101}{RRG4101}%
\defkpp{RRG4102}{RRG4102}%
\defkpp{RRG4103}{RRG4103}%
\defkpp{RRG4104}{RRG4104}%
\defkpp{RRG4105}{RRG4105}%
\defkpp{RRG4106a}{RRG4106a}%
\defkpp{RRG4106b}{RRG4106b}%
\defkpp{RRG4107}{RRG4107}%
\defkpp{RRG4108}{RRG4108}%
\defkpp{RRG4109}{RRG4109}%
\defkpp{RRG4110}{RRG4110}%
\defkpp{RRG4111}{RRG4111}%
\defkpp{RRG4200}{RRG4200}%
\defkpp{RRG4201}{RRG4201}%
\defkpp{RRG4202}{RRG4202}%
\defkpp{RRG4203}{RRG4203}%
\defkpp{RRG4204}{RRG4204}%
\defkpp{RRG4205}{RRG4205}%
\defkpp{RRG4206}{RRG4206}%
\defkpp{RRG4207}{RRG4207}%
\defkpp{RRG4208}{RRG4208}%
\defkpp{RRG4209}{RRG4209}%
\defkpp{RRG4210}{RRG4210}%
\defkpp{RRG4211a}{RRG4211a}%
\defkpp{RRG4211b}{RRG4211b}%
\defkpp{RRG4212}{RRG4212}%
\defkpp{RRG4213}{RRG4213}%
\defkpp{RRG4214}{RRG4214}%
\defkpp{RRG4217}{RRG4217}%
\defkpp{RRG4218}{RRG4218}%
\defkpp{RRG4220}{RRG4220}%
\defkpp{RRG4221}{RRG4221}%
\defkpp{RRG4222}{RRG4222}%
\defkpp{RRG4223}{RRG4223}%
\defkpp{RRG4224}{RRG4224}%
\defkpp{RRG4225}{RRG4225}%
\defkpp{RRG4226}{RRG4226}%
\defkpp{RRG4227}{RRG4227}%
\defkpp{RRG4228}{RRG4228}%
\defkpp{RRG4229}{RRG4229}%
\defkpp{RRG4230}{RRG4230}%
\defkpp{RRG4231}{RRG4231}%
\defkpp{RRG4232}{RRG4232}%
\defkpp{RRG4233}{RRG4233}%
\defkpp{RRG4234}{RRG4234}%
\defkpp{RRG4235}{RRG4235}%
\defkpp{RRG4236}{RRG4236}%
\defkpp{RRG4237}{RRG4237}%
\defkpp{RRG4238}{RRG4238}%
\defkpp{RRG4239}{RRG4239}%
\defkpp{RRG4240}{RRG4240}%
\defkpp{RRG4241}{RRG4241}%
\defkpp{RRG4242}{RRG4242}%
\defkpp{RRG4243}{RRG4243}%
\defkpp{RRG4244}{RRG4244}%
\defkpp{RRG4245}{RRG4245}%
\defkpp{RRG4246a}{RRG4246a}%
\defkpp{RRG4246b}{RRG4246b}%
\defkpp{RRG4247a}{RRG4247a}%
\defkpp{RRG4247b}{RRG4247b}%
\defkpp{RRG4248}{RRG4248}%
\defkpp{RRG4300}{RRG4300}%
\defkpp{RRG4301}{RRG4301}%
\defkpp{RRG4302}{RRG4302}%
\defkpp{RRG4303}{RRG4303}%
\defkpp{RRG4304}{RRG4304}%
\defkpp{RRG4305}{RRG4305}%
\defkpp{RRG4306}{RRG4306}%
\defkpp{RRG4307}{RRG4307}%
\defkpp{RRG4311}{RRG4311}%
\defkpp{RRG4312}{RRG4312}%
\defkpp{RRG4313}{RRG4313}%
\defkpp{RRG4314}{RRG4314}%
\defkpp{RRG4315a}{RRG4315a}%
\defkpp{RRG4315b}{RRG4315b}%
\defkpp{RRG4316}{RRG4316}%
\defkpp{RRG4317}{RRG4317}%
\defkpp{RRG4320}{RRG4320}%
\defkpp{RRG4321}{RRG4321}%
\defkpp{RRG4322}{RRG4322}%
\defkpp{RRG4323}{RRG4323}%
\defkpp{RRG4324}{RRG4324}%
\defkpp{RRG4325}{RRG4325}%
\defkpp{RRG4326a}{RRG4326a}%
\defkpp{RRG4326b}{RRG4326b}%
\defkpp{RRG4327}{RRG4327}%
\defkpp{RRG4328}{RRG4328}%
\defkpp{RRG4329}{RRG4329}%
\defkpp{RRG4330a}{RRG4330a}%
\defkpp{RRG4330b}{RRG4330b}%
\defkpp{RRG4331}{RRG4331}%
\defkpp{RRG4332}{RRG4332}%
\defkpp{RRG4333}{RRG4333}%
\defkpp{RRG4334}{RRG4334}%
\defkpp{RRG4335}{RRG4335}%
\defkpp{RRG4400}{RRG4400}%
\defkpp{RRG4401}{RRG4401}%
\defkpp{RRG4402}{RRG4402}%
\defkpp{RRG4403}{RRG4403}%
\defkpp{RRG4404}{RRG4404}%
\defkpp{RRG4405}{RRG4405}%
\defkpp{RRG4406}{RRG4406}%
\defkpp{RRG4413}{RRG4413}%
\defkpp{RRG4414}{RRG4414}%
\defkpp{RRG4415}{RRG4415}%
\defkpp{RRG4416}{RRG4416}%
\defkpp{RRG4417}{RRG4417}%
\defkpp{RRG4418}{RRG4418}%
\defkpp{RRG4419}{RRG4419}%
\defkpp{RRG4420}{RRG4420}%
\defkpp{RRG4421}{RRG4421}%
\defkpp{RRG4422}{RRG4422}%
\defkpp{RRG4423}{RRG4423}%
\defkpp{RRG4424}{RRG4424}%
\defkpp{RRG4425}{RRG4425}%
\defkpp{RRG4426}{RRG4426}%
\defkpp{RRG4427}{RRG4427}%
\defkpp{RRG4428}{RRG4428}%
\defkpp{RRG4429}{RRG4429}%
\defkpp{RRG4430}{RRG4430}%
\defkpp{RRG4431}{RRG4431}%
\defkpp{RRG4432}{RRG4432}%
\defkpp{RRG4433}{RRG4433}%
\defkpp{RRG4434}{RRG4434}%
\defkpp{RRG4435}{RRG4435}%
\defkpp{RRG4436}{RRG4436}%
\defkpp{RRG4437}{RRG4437}%
\defkpp{RRG4438}{RRG4438}%
\defkpp{RRG4439}{RRG4439}%
\defkpp{RRG4440}{RRG4440}%
\defkpp{RRG4441}{RRG4441}%
\defkpp{RRG4442}{RRG4442}%
\defkpp{RRG4443}{RRG4443}%
\defkpp{RRG4444}{RRG4444}%
\defkpp{RRG4445}{RRG4445}%
\defkpp{RRG4446}{RRG4446}%
\defkpp{RRG4447}{RRG4447}%
\defkpp{RRG4448}{RRG4448}%
\defkpp{RRG4449}{RRG4449}%
\defkpp{RRG4450}{RRG4450}%
\defkpp{RRG4451}{RRG4451}%
\defkpp{RRG4452}{RRG4452}%
\defkpp{RRG4453}{RRG4453}%
\defkpp{RRG4454}{RRG4454}%
\defkpp{RRG4455}{RRG4455}%
\defkpp{RRG4456}{RRG4456}%
\defkpp{RRG4500}{RRG4500}%
\defkpp{RRG4501}{RRG4501}%
\defkpp{RRG4509}{RRG4509}%
\defkpp{RRG4510}{RRG4510}%
\defkpp{RRG4511}{RRG4511}%
\defkpp{RRG4512}{RRG4512}%
\defkpp{RRG4513}{RRG4513}%
\defkpp{RRG4514}{RRG4514}%
\defkpp{RRG4515}{RRG4515}%
\defkpp{RRG4516}{RRG4516}%
\defkpp{RRG4517}{RRG4517}%
\defkpp{RRG4518}{RRG4518}%
\defkpp{RRG4519}{RRG4519}%
\defkpp{RRG4520}{RRG4520}%
\defkpp{RRG4521}{RRG4521}%
\defkpp{RRG4522}{RRG4522}%
\defkpp{RRG4523}{RRG4523}%
\defkpp{RRG4524}{RRG4524}%
\defkpp{RRG4525}{RRG4525}%
\defkpp{RRG4526}{RRG4526}%
\defkpp{RRG4527}{RRG4527}%
\defkpp{RRG4528}{RRG4528}%
\defkpp{RRG4529}{RRG4529}%
\defkpp{RRG4530}{RRG4530}%
\defkpp{RRG4531}{RRG4531}%
\defkpp{RRG4532}{RRG4532}%
\defkpp{RRG4533}{RRG4533}%
\defkpp{RRG4534}{RRG4534}%
\defkpp{RRG4535}{RRG4535}%
\defkpp{RRG4536}{RRG4536}%
\defkpp{RRG4537}{RRG4537}%
\defkpp{RRG4538}{RRG4538}%
\defkpp{RRG4539}{RRG4539}%
\defkpp{RRG4540}{RRG4540}%
\defkpp{RRG4541}{RRG4541}%
\defkpp{RRG4542}{RRG4542}%
\defkpp{RRG4543}{RRG4543}%
\defkpp{RRG4544}{RRG4544}%
\defkpp{RRG4545}{RRG4545}%
\defkpp{RRG4546}{RRG4546}%
\defkpp{RRG4547}{RRG4547}%
\defkpp{RRG4548}{RRG4548}%
\defkpp{RRG4549}{RRG4549}%
\defkpp{RRG4550}{RRG4550}%
\defkpp{RRG4551}{RRG4551}%
\defkpp{RRG4552}{RRG4552}%
\defkpp{RRG4553}{RRG4553}%
\defkpp{RRG4554}{RRG4554}%
\defkpp{RRG4555}{RRG4555}%
\defkpp{RRG4556}{RRG4556}%
\defkpp{RRG4557}{RRG4557}%
\defkpp{RRG4558}{RRG4558}%
\defkpp{RRG4559}{RRG4559}%
\defkpp{RRG4560}{RRG4560}%
\defkpp{RRG4561}{RRG4561}%
\defkpp{RRG4562}{RRG4562}%
\defkpp{RRG4563}{RRG4563}%
\defkpp{RRG4564}{RRG4564}%
\defkpp{RRG4565}{RRG4565}%
\defkpp{RRJ1000}{RRJ1000}%
\defkpp{RRJ1001a}{RRJ1001a}%
\defkpp{RRJ1001b}{RRJ1001b}%
\defkpp{RRJ2101}{RRJ2101}%
\defkpp{RRJ3101}{RRJ3101}%
\defkpp{RRJ3103a}{RRJ3103a}%
\defkpp{RRJ3103b}{RRJ3103b}%
\defkpp{RRJ3104a}{RRJ3104a}%
\defkpp{RRJ3200}{RRJ3200}%
\defkpp{RRJ3201}{RRJ3201}%
\defkpp{RRJ3202}{RRJ3202}%
\defkpp{RRJ4100}{RRJ4100}%
\defkpp{RRJ4101a}{RRJ4101a}%
\defkpp{RRJ4101b}{RRJ4101b}%
\defkpp{RRJ4200}{RRJ4200}%
\defkpp{RRJ4201}{RRJ4201}%
\defkpp{RRJ4202}{RRJ4202}%
\defkpp{RRJ4204}{RRJ4204}%
\defkpp{RRJ4205}{RRJ4205}%
\defkpp{RRJ4206}{RRJ4206}%
\defkpp{RRJ4207}{RRJ4207}%
\defkpp{RRJ4208}{RRJ4208}%
\defkpp{RRJ4209}{RRJ4209}%
\defkpp{RRJ4210}{RRJ4210}%
\defkpp{RRJ4211}{RRJ4211}%
\defkpp{RRJ4212}{RRJ4212}%
\defkpp{RRJ4300}{RRJ4300}%
\defkpp{RRJ4301}{RRJ4301}%
\defkpp{RRJ4302}{RRJ4302}%
\defkpp{RRJ4303}{RRJ4303}%
\defkpp{RRJ4304}{RRJ4304}%
\defkpp{RRJ4306}{RRJ4306}%
\defkpp{RRJ4307}{RRJ4307}%
\defkpp{RRJ4308}{RRJ4308}%
\defkpp{RRJ4309}{RRJ4309}%
\defkpp{RRJ4310}{RRJ4310}%
\defkpp{RRJ4311}{RRJ4311}%
\defkpp{RRJ4400}{RRJ4400}%
\defkpp{RRJ4401}{RRJ4401}%
\defkpp{RRJ4403}{RRJ4403}%
\defkpp{RRJ4404}{RRJ4404}%
\defkpp{RRJ4405}{RRJ4405}%
\defkpp{RRJ4406}{RRJ4406}%
\defkpp{RRJ4407}{RRJ4407}%
\defkpp{RRJ4408}{RRJ4408}%
\defkpp{RRJ4409}{RRJ4409}%
\defkpp{RRJ4410}{RRJ4410}%
\defkpp{RRJ4411}{RRJ4411}%
\defkpp{RRJ4412}{RRJ4412}%
\defkpp{RRJ4413}{RRJ4413}%
\defkpp{RRJ4414}{RRJ4414}%
\defkpp{RRJ4415}{RRJ4415}%
\defkpp{RRJ4416}{RRJ4416}%
\defkpp{RRJ4417}{RRJ4417}%
\defkpp{RRJ4418}{RRJ4418}%
\defkpp{RRJ4419}{RRJ4419}%
\defkpp{RRJ4420}{RRJ4420}%
\defkpp{RRJ4502}{RRJ4502}%
\defkpp{RRJ4503}{RRJ4503}%
\defkpp{RRJ4504}{RRJ4504}%
\defkpp{RRJ4505}{RRJ4505}%
\defkpp{RRJ4506}{RRJ4506}%
\defkpp{RRJ4507}{RRJ4507}%
\defkpp{RRJ4508}{RRJ4508}%
\defkpp{RRJ4509}{RRJ4509}%
\defkpp{RRJ4510}{RRJ4510}%
\defkpp{RRJ4511}{RRJ4511}%
\defkpp{RRJ4512}{RRJ4512}%
\defkpp{RRJ4513}{RRJ4513}%
\defkpp{RRJ4514}{RRJ4514}%
\defkpp{RRJ4515}{RRJ4515}%
\defkpp{RRJ4516}{RRJ4516}%
\makeatother


\begin{longtable}{llp{9cm}p{7cm}p{5cm}}
\caption{Gas phase reactions}\\
\hline
\# & labels & reaction & rate coefficient & reference\\
\hline
\endfirsthead
\caption{Gas phase reactions (... continued)}\\
\hline
\# & labels & reaction & rate coefficient & reference\\
\hline
\endhead
\hline
\endfoot
% this file was created automatically by eqn2tex, do not edit!
\code{G1000} & StTrG &  \kpp{O2} + \kpp{O1D} $\rightarrow$ \kpp{O3P} + \kpp{O2} & \code{3.3E-11*EXP(55./temp)} & \citet{1945}\\
\code{G1001} & StTrG &  \kpp{O2} + \kpp{O3P} $\rightarrow$ \kpp{O3} & \code{6.E-34*((temp/300.)**(-2.4))*cair} & \citet{1945}\\
\myhline
\code{G2100} & StTrG &  \kpp{H} + \kpp{O2} $\rightarrow$ \kpp{HO2} & \code{k_3rd(temp,cair,4.4E-32,1.3,4.7E-11,0.2,0.6)} & \citet{1945}\\
\code{G2104} & StTrG &  \kpp{OH} + \kpp{O3} $\rightarrow$ \kpp{HO2} + \kpp{O2} & \code{1.7E-12*EXP(-940./temp)} & \citet{1945}\\
\code{G2105} & StTrG &  \kpp{OH} + \kpp{H2} $\rightarrow$ \kpp{H2O} + \kpp{H} & \code{2.8E-12*EXP(-1800./temp)} & \citet{1945}\\
\code{G2107} & StTrG &  \kpp{HO2} + \kpp{O3} $\rightarrow$ \kpp{OH} + 2 \kpp{O2} & \code{1.E-14*EXP(-490./temp)} & \citet{1945}\\
\code{G2109} & StTrG &  \kpp{HO2} + \kpp{OH} $\rightarrow$ \kpp{H2O} + \kpp{O2} & \code{4.8E-11*EXP(250./temp)} & \citet{1945}\\
\code{G2110} & StTrG &  \kpp{HO2} + \kpp{HO2} $\rightarrow$ \kpp{H2O2} + \kpp{O2} & \code{k_HO2_HO2} & \citet{1599}, \citet{165}$^*$\shownote{G2110}\\
\code{G2111} & StTrG &  \kpp{H2O} + \kpp{O1D} $\rightarrow$ 2 \kpp{OH} & \code{1.63E-10*EXP(60./temp)} & \citet{1945}\\
\code{G2112} & StTrG &  \kpp{H2O2} + \kpp{OH} $\rightarrow$ \kpp{H2O} + \kpp{HO2} & \code{1.8E-12} & \citet{1945}\\
\myhline
\code{G3101} & StTrG &  \kpp{N2} + \kpp{O1D} $\rightarrow$ \kpp{O3P} + \kpp{N2} & \code{2.15E-11*EXP(110./temp)} & \citet{1945}\\
\code{G3103} & StTrGN &  \kpp{NO} + \kpp{O3} $\rightarrow$ \kpp{NO2} + \kpp{O2} & \code{3.E-12*EXP(-1500./temp)} & \citet{1945}\\
\code{G3106} & StTrGN &  \kpp{NO2} + \kpp{O3} $\rightarrow$ \kpp{NO3} + \kpp{O2} & \code{1.2E-13*EXP(-2450./temp)} & \citet{1945}\\
\code{G3108} & StTrGN &  \kpp{NO3} + \kpp{NO} $\rightarrow$ 2 \kpp{NO2} & \code{1.5E-11*EXP(170./temp)} & \citet{1945}\\
\code{G3109} & StTrGN &  \kpp{NO3} + \kpp{NO2} $\rightarrow$ \kpp{N2O5} & \code{k_NO3_NO2} & \citet{1945}$^*$\shownote{G3109}\\
\code{G3110} & StTrGN &  \kpp{N2O5} $\rightarrow$ \kpp{NO2} + \kpp{NO3} & \code{k_NO3_NO2/(2.7E-27*EXP(11000./temp))} & \citet{1945}$^*$\shownote{G3110}\\
\code{G3200} & TrGN &  \kpp{NO} + \kpp{OH} $\rightarrow$ \kpp{HONO} & \code{k_3rd(temp,cair,7.0E-31,2.6,3.6E-11,0.1,0.6)} & \citet{1945}\\
\code{G3201} & StTrGN &  \kpp{NO} + \kpp{HO2} $\rightarrow$ \kpp{NO2} + \kpp{OH} & \code{3.5E-12*EXP(250./temp)} & \citet{1945}\\
\code{G3202} & StTrGN &  \kpp{NO2} + \kpp{OH} $\rightarrow$ \kpp{HNO3} & \code{k_3rd(temp,cair,1.8E-30,3.0,2.8E-11,0.,0.6)} & \citet{1945}\\
\code{G3203} & StTrGN &  \kpp{NO2} + \kpp{HO2} $\rightarrow$ \kpp{HNO4} & \code{k_NO2_HO2} & \citet{1945}$^*$\shownote{G3203}\\
\code{G3204} & TrGN &  \kpp{NO3} + \kpp{HO2} $\rightarrow$ \kpp{NO2} + \kpp{OH} + \kpp{O2} & \code{3.5E-12} & \citet{1945}\\
\code{G3205} & TrGN &  \kpp{HONO} + \kpp{OH} $\rightarrow$ \kpp{NO2} + \kpp{H2O} & \code{1.8E-11*EXP(-390./temp)} & \citet{1945}\\
\code{G3206} & StTrGN &  \kpp{HNO3} + \kpp{OH} $\rightarrow$ \kpp{H2O} + \kpp{NO3} & \code{k_HNO3_OH} & \citet{1945}$^*$\shownote{G3206}\\
\code{G3207} & StTrGN &  \kpp{HNO4} $\rightarrow$ \kpp{NO2} + \kpp{HO2} & \code{k_NO2_HO2/(2.1E-27*EXP(10900./temp))} & \citet{1945}$^*$\shownote{G3207}\\
\code{G3208} & StTrGN &  \kpp{HNO4} + \kpp{OH} $\rightarrow$ \kpp{NO2} + \kpp{H2O} & \code{1.3E-12*EXP(380./temp)} & \citet{1945}\\
\code{G3209} & TrGN &  \kpp{NH3} + \kpp{OH} $\rightarrow$ \kpp{NH2} + \kpp{H2O} & \code{1.7E-12*EXP(-710./temp)} & \citet{2415}\\
\code{G3210} & TrGN &  \kpp{NH2} + \kpp{O3} $\rightarrow$ \kpp{NH2O} + \kpp{O2} & \code{4.3E-12*EXP(-930./temp)} & \citet{2415}\\
\code{G3211} & TrGN &  \kpp{NH2} + \kpp{HO2} $\rightarrow$ \kpp{NH2O} + \kpp{OH} & \code{4.8E-07*EXP(-628./temp)*temp**(-1.32)} & \citet{2415}\\
\code{G3212} & TrGN &  \kpp{NH2} + \kpp{HO2} $\rightarrow$ \kpp{HNO} + \kpp{H2O} & \code{9.4E-09*EXP(-356./temp)*temp**(-1.12)} & \citet{2415}\\
\code{G3213} & TrGN &  \kpp{NH2} + \kpp{NO} $\rightarrow$ \kpp{HO2} + \kpp{OH} + \kpp{N2} & \code{1.92E-12*((temp/298.)**(-1.5))} & \citet{2415}\\
\code{G3214} & TrGN &  \kpp{NH2} + \kpp{NO} $\rightarrow$ \kpp{N2} + \kpp{H2O} & \code{1.41E-11*((temp/298.)**(-1.5))} & \citet{2415}\\
\code{G3215} & TrGN &  \kpp{NH2} + \kpp{NO2} $\rightarrow$ \kpp{N2O} + \kpp{H2O} & \code{1.2E-11*((temp/298.)**(-2.0))} & \citet{2415}\\
\code{G3216} & TrGN &  \kpp{NH2} + \kpp{NO2} $\rightarrow$ \kpp{NH2O} + \kpp{NO} & \code{0.8E-11*((temp/298.)**(-2.0))} & \citet{2415}\\
\code{G3217} & TrGN &  \kpp{NH2O} + \kpp{O3} $\rightarrow$ \kpp{NH2} + \kpp{O2} & \code{1.2E-14} & \citet{2415}\\
\code{G3218} & TrGN &  \kpp{NH2O} $\rightarrow$ \kpp{NHOH} & \code{1.3E+3} & \citet{2415}\\
\code{G3219} & TrGN &  \kpp{HNO} + \kpp{OH} $\rightarrow$ \kpp{NO} + \kpp{H2O} & \code{8.0E-11*EXP(-500./temp)} & \citet{2415}\\
\code{G3220} & TrGN &  \kpp{HNO} + \kpp{NHOH} $\rightarrow$ \kpp{NH2OH} + \kpp{NO} & \code{1.66E-12*EXP(-1500./temp)} & \citet{2415}\\
\code{G3221} & TrGN &  \kpp{HNO} + \kpp{NO2} $\rightarrow$ \kpp{HONO} + \kpp{NO} & \code{1.0E-12*EXP(-1000./temp)} & \citet{2415}\\
\code{G3222} & TrGN &  \kpp{NHOH} + \kpp{OH} $\rightarrow$ \kpp{HNO} + \kpp{H2O} & \code{1.66E-12} & \citet{2415}\\
\code{G3223} & TrGN &  \kpp{NH2OH} + \kpp{OH} $\rightarrow$ \kpp{NHOH} + \kpp{H2O} & \code{4.13E-11*EXP(-2138./temp)} & \citet{2415}\\
\code{G3224} & TrGN &  \kpp{HNO} + \kpp{O2} $\rightarrow$ \kpp{HO2} + \kpp{NO} & \code{3.65E-14*EXP(-4600./temp)} & \citet{2415}\\
\myhline
\code{G4101} & StTrG &  \kpp{CH4} + \kpp{OH} $\rightarrow$ \kpp{CH3O2} + \kpp{H2O} & \code{1.85E-20*EXP(2.82*log(temp)-987./temp)} & \citet{1627}$^*$\shownote{G4101}\\
\code{G4102} & TrG &  \kpp{CH3OH} + \kpp{OH} $\rightarrow$ \kpp{HCHO} + \kpp{HO2} & \code{2.9E-12*EXP(-345./temp)} & \citet{1945}\\
\code{G4103} & StTrG &  \kpp{CH3O2} + \kpp{HO2} $\rightarrow$ \kpp{CH3OOH} + \kpp{O2} & \code{4.1E-13*EXP(750./temp)} & \citet{1945}$^*$\shownote{G4103}\\
\code{G4104} & StTrGN &  \kpp{CH3O2} + \kpp{NO} $\rightarrow$ \kpp{HCHO} + \kpp{NO2} + \kpp{HO2} & \code{2.8E-12*EXP(300./temp)} & \citet{1945}\\
\code{G4105} & TrGN &  \kpp{CH3O2} + \kpp{NO3} $\rightarrow$ \kpp{HCHO} + \kpp{HO2} + \kpp{NO2} & \code{1.3E-12} & \citet{1759}\\
\code{G4106a} & StTrG &  \kpp{CH3O2} $\rightarrow$ \kpp{HCHO} + \kpp{HO2} & \code{2.*RO2*9.5E-14*EXP(390./temp)/(1.+1./26.2*EXP(1130./temp))} & \citet{1945}\\
\code{G4106b} & StTrG &  \kpp{CH3O2} $\rightarrow$ .5 \kpp{HCHO} + .5 \kpp{CH3OH} + .5 \kpp{O2} & \code{2.*RO2*9.5E-14*EXP(390./temp)/(1.+26.2*EXP(-1130./temp))} & \citet{1945}\\
\code{G4107} & StTrG &  \kpp{CH3OOH} + \kpp{OH} $\rightarrow$ .7 \kpp{CH3O2} + .3 \kpp{HCHO} + .3 \kpp{OH} + \kpp{H2O} & \code{k_CH3OOH_OH} & \citet{1945}$^*$\shownote{G4107}\\
\code{G4108} & StTrG &  \kpp{HCHO} + \kpp{OH} $\rightarrow$ \kpp{CO} + \kpp{H2O} + \kpp{HO2} & \code{9.52E-18*EXP(2.03*log(temp)+636./temp)} & \citet{1634}\\
\code{G4109} & TrGN &  \kpp{HCHO} + \kpp{NO3} $\rightarrow$ \kpp{HNO3} + \kpp{CO} + \kpp{HO2} & \code{3.4E-13*EXP(-1900./temp)} & \citet{1945}$^*$\shownote{G4109}\\
\code{G4110} & StTrG &  \kpp{CO} + \kpp{OH} $\rightarrow$ \kpp{H} + \kpp{CO2} & \code{(1.57E-13+cair*3.54E-33)} & \citet{1628}\\
\code{G4111} & TrG &  \kpp{HCOOH} + \kpp{OH} $\rightarrow$ \kpp{CO2} + \kpp{HO2} + \kpp{H2O} & \code{4.0E-13} & \citet{1945}\\
\myhline
\myhline
\myhline
\myhline
\myhline

\end{longtable}

\newpage\begin{multicols}{3}
$^*$Notes:

Rate coefficients for three-body reactions are defined via the function
\code{k_3rd}($T$, $M$, $k_0^{300}$, $n$, $k_{\rm inf}^{300}$, $m$,
$f_{\rm c}$). In the code, the temperature $T$ is called \code{temp} and
the concentration of ``air molecules'' $M$ is called \code{cair}. Using
the auxiliary variables $k_0(T)$, $k_{\rm inf}(T)$, and $k_{\rm ratio}$,
\code{k_3rd} is defined as:
\begin{eqnarray}
  k_0(T)              & = & k_0^{300} \times \left( \frac{300
                            \unit{K}}{T} \right)^n\\
  k_{\rm inf}(T)      & = & k_{\rm inf}^{300} \times \left( \frac{300
                            \unit{K}}{T} \right)^m\\
  k_{\rm ratio}       & = & \frac{k_0(T) M}{k_{\rm inf}(T)}\\
  \mbox{\code{k_3rd}} & = & \frac{k_0(T) M}{1+k_{\rm ratio}} \times f_{\rm
                            c}^{\left( \frac{1}{1+(\log_{10}(k_{\rm
                            ratio}))^2} \right)}
\end{eqnarray}

A similar function, called \code{k_3rd_iupac} here, is used by
\citet{1745} for three-body reactions. It has the same function
parameters as \code{k_3rd} and it is defined as:
\begin{eqnarray}
  k_0(T)                    & = & k_0^{300} \times \left( \frac{300
                                  \unit{K}}{T} \right)^n\\
  k_{\rm inf}(T)            & = & k_{\rm inf}^{300} \times \left( \frac{300
                                  \unit{K}}{T} \right)^m\\
  k_{\rm ratio}             & = & \frac{k_0(T) M}{k_{\rm inf}(T)}\\
  N                         & = & 0.75 - 1.27 \times \log_{10}(f_{\rm c})\\
  \mbox{\code{k_3rd_iupac}} & = & \frac{k_0(T) M}{1+k_{\rm ratio}} \times f_{\rm
                                  c}^{\left( \frac{1}{1+(\log_{10}(k_{\rm
                                  ratio})/N)^2} \right)}
\end{eqnarray}

% O

\note{G1002}{The path leading to 2 \kpp{O3P} + \kpp{O2} results in a
  null cycle regarding odd oxygen and is neglected.}

% H

\note{G2110}{The rate coefficient is: \code{k_HO2_HO2} =
  \code{(1.5E-12*EXP(19./temp)+1.7E-33*EXP(1000./temp)*cair)*
    (1.+1.4E-21*EXP(2200./temp)*C(ind_H2O))}. The value for the first
  (pressure-independent) part is from \citet{1599}, the water term from
  \citet{165}.}

% N

\note{G3109}{The rate coefficient is: \code{k_NO3_NO2} =
  \code{k_3rd(temp,cair,2.E-30,4.4,1.4E-12,0.7,0.6)}.}

\note{G3110}{The rate coefficient is defined as backward reaction
  divided by equilibrium constant.}

\note{G3203}{The rate coefficient is: \code{k_NO2_HO2} =
  \code{k_3rd(temp,cair,1.8E-31,3.2,4.7E-12,1.4,0.6)}.}

\note{G3206}{The rate coefficient is: \code{k_HNO3_OH} = \code{2.4E-14 *
    EXP(460./temp) + 1./ ( 1./(6.5E-34 * EXP(1335./temp)*cair) +
    1./(2.7E-17 * EXP(2199./temp)) )}}

\note{G3207}{The rate coefficient is defined as backward reaction
  divided by equilibrium constant.}

% C1

\note{G4103}{\citet{1945} recommend a zero product yield for
  \chem{HCHO}.}

\note{G4107}{The rate coefficient is: \code{k_CH3OOH_OH} =
  \code{3.8E-12*EXP(200./temp)}.}

\note{G4109}{The same temperature dependence assumed as for
  \kpp{CH3CHO}+\kpp{NO3}.}

% C2

\note{G4201}{The product distribution is from \citet{2419}, after
  substitution of the Criegee intermediate by its decomposition
  products.}

\note{G4206}{The product \chem{C_2H_5OH}, which reacts only with OH, is
  substituted by its degradation products $\approx$ 0.1 \kpp{HOCH2CH2O2}
  + 0.9 \kpp{CH3CHO} + 0.9 \kpp{HO2}.}

\note{G4207}{The rate constant \code{8.01E-12} is for the H abstraction
  in alpha to the \chem{-OOH} group \citep{2419} and
  \code{0.6*k_CH3OOH_OH} is for the \chem{C_2H_5O_2} channel. The
  branching ratios are calculated from the terms of the rate coefficient
  at 298\,\unit{K}.}

\note{G4213}{The rate coefficient is: \code{k_CH3CO3_NO2} =
  \code{k_3rd(temp,cair,9.7E-29,5.6,9.3E-12,1.5,0.6)}.}

\note{G4218}{The rate coefficient is the same as for the \kpp{CH3O2}
  channel in G4107 (\kpp{CH3OOH}+\kpp{OH}).}

\note{G4221}{The rate coefficient is}{\code{k_PAN_M} =
  \code{k_CH3CO3_NO2/9.E-29*EXP(-14000./temp)}, i.e.\ the rate
  coefficient is defined as backward reaction divided by equilibrium
  constant.}

\note{G4243}{\citet{2425} estimated that about 25\% of the
  \kpp{HOCH2CH2O} in this reaction is produced with sufficient excess
  energy that it decomposes promptly. The decomposition products are 2
  \kpp{HCHO} + \kpp{HO2}.}

\note{G4246a}{The rate coefficient is the same as for the \chem{CH_3O_2}
  channel in G4107: \kpp{CH3OOH}+\kpp{OH}.}

% C3

\note{G4300}{The product NC3H7O2 is substituted with its degradation
  products \kpp{C2H5O2} + \kpp{CO2} + \kpp{HO2}.}

\note{G4301}{The product distribution is for terminal olefin carbons
  from \citet{1623}.}

\note{G4304}{The value for the generic \chem{RO_2} + \chem{HO_2}
  reaction from \citet{964} is used here.}

\note{G4306}{The MCM \citep{2419} products are 0.2 IPROPOL + 0.2
  \kpp{CH3COCH3} + 0.6 IC3H7O. IPROPOL and IC3H7O are substituted with
  their degradation products. We assume IPROPOL to be oxidized entirely
  to \kpp{CH3COCH3} + \kpp{HO2} by \kpp{OH}. IC3H7O + \kpp{O2} produces
  the same products.}
 
\note{G4307}{Analogous to G4207 for both rate coefficient and branching
  ratios.}

\note{G4315a}{The same value as for G4107 (\kpp{CH3OOH} + \kpp{OH}) is
  used, multiplied by the branching ratio of the \kpp{CH3O2} channel.}

% C4

\note{G4400}{\kpp{LC4H9O2} represents 0.127 NC4H9O2 + 0.873 SC4H9O2.}

\note{G4401}{NC4H9O and SC4H9O are substituted with 2 \kpp{CO2} +
  \kpp{C2H5O2} and 0.636 \kpp{MEK} + \kpp{HO2} and 0.364 \kpp{CH3CHO} +
  \kpp{C2H5O2}, repectively. The stoichiometric coefficients on the
  right side are weighted averages.}

\note{G4403}{The alkyl nitrate yield is the weighted average yield for
  the two isomers forming from NC4H9O2 and SC4H9O2.}

\note{G4404}{The product distribution is the weighted average of the
  single isomer hydroperoxides. It is calculated from the rate constants
  of single channels and the ratio of the isomers NC4H9O2 and SC4H9O2.
  The overall rate constant for this reaction is calculated as weighted
  average of the channels rate constants. The relative weight of the
  products from NC4H9OOH and SC4H9OOH are then 0.0887 and 0.9113. The
  channels producing \chem{RO_2} are given the rate coefficient
  \code{0.6*k_CH3OOH_OH} as for G4107. For NC4H9OOH the products are
  0.327 NC4H9O2 + 0.673 C3H7CHO + 0.673 OH. C3H7CHO is then substituted
  with 2 \kpp{CO2} + \kpp{C2H5O2}. Hence, 0.327 NC4H9O2 + 1.346
  \kpp{CO2} + 0.673 \kpp{C2H5O2} + 0.673 OH. For SC4H9OOH the products
  are 0.219 SC4H9O2 + 0.781 MEK + 0.781 OH.}

\note{G4413}{\kpp{LMEKO2} represents 0.459 MEKAO2 + 0.462 MEKBO2 + 0.079
  MEKCO2.}

\note{G4415}{Alkyl nitrate formation is neglected. The products of MEKAO
  and MEKCO are substituted with \kpp{HCHO} + \kpp{CO2} +
  \kpp{HOCH2CH2O2} and \kpp{HCHO} + \kpp{CO2} + \kpp{C2H5O2}.}

\note{G4416}{\kpp{LMEKOOH} is assumed having the composition 0.459
  MEKAOOH + 0.462 MEKBOOH + 0.079 MEKCOOH. MEKAOOH + \kpp{OH} gives 0.89
  CO2C3CHO + 0.89 \kpp{OH} + 0.11 MEKAO2 + \kpp{H2O}. CO2C3CHO is
  substituted with \kpp{CH3COCH2O2} + \kpp{CO2} and the products become
  0.89 \kpp{CH3COCH2O2} + 0.89 \kpp{CO2} + 0.89 \kpp{OH} + 0.11 MEKAO2 +
  \kpp{H2O}. MEKBOOH + \kpp{OH} gives 0.758 \kpp{BIACET} + 0.758
  \kpp{OH} + 0.242 MEKBO2 + \kpp{H2O}. MEKCOOH + \kpp{OH} gives 0.614
  EGLYOX + 0.614 \kpp{OH} + 0.386 MEKCO2 + \kpp{H2O}. EGLYOX is
  substituted with \kpp{C2H5O2} + 2 \kpp{CO2} and the products become
  0.614 \kpp{C2H5O2} + 1.228 \kpp{CO2} + 0.614 \kpp{OH} + 0.386 MEKCO2 +
  \kpp{H2O}.}

\note{G4417}{The rate coefficient is the combination of the ones for the
  two isomers weighted by the relative abundances for NC4H9NO3 and
  SC4H9NO3, respectively. Product distribution is calculated
  accordingly. NC4H9NO3 + \kpp{OH} gives C3H7CHO + \kpp{NO2} + \kpp{H2O}
  with C3H7CHO being substituted with 2 \kpp{CO2} + \kpp{C2H5O2}. After
  substitution is obtained 2 \kpp{CO2} + \kpp{C2H5O2} + \kpp{NO2} +
  \kpp{H2O}. SC4H9NO3 + \kpp{OH} gives \kpp{MEK} + \kpp{NO2} + \kpp{H2O}
  For the product distribution NC4H9NO3 and SC4H9NO3 account for 0.08577
  and 0.91423, respectively.}

\note{G4419}{The same value as for \kpp{PAN} is assumed.}

\note{G4420}{Products are as in G4415. Only the main channels for each
  isomer are considered. Rate constant is the weighted average for the
  isomers.}

\note{G4437}{\kpp{LHMVKABO2} is a lumped species of virtual composition
  0.3 HMVKAO2 + 0.7 HMVKBO2. The products are the weighted average for
  the permutation reactions of each single RO2 in the MCM \citep{2419}.}

\note{G4439}{products are the weighted average for the decomposition of
  0.3 HMVKAO + 0.7 HMVKBO.}

\note{G4440}{as for G4439}

\note{G4441}{The rate coefficient and products are 30\% for HMVKAOOH and
  70\% for HMVKBOOH.}

\note{G4444}{\kpp{LMVKOHABO2} is a lumped species of virtual composition
  0.3 MVKOHAO2 + 0.7 MVKOHBO2. The products are the weighted average for
  the permutation reactions of each single RO2 in the MCM \citep{2419}.}

\note{G4446}{products are the weighted average for the decomposition of
  0.3 MVKOHAO + 0.7 MVKOHBO.}

\note{G4447}{as for G4446}

\note{G4448}{The rate coefficient and products are 30\% for MVKOHAOOH
  and 70\% for MVKOHBOOH.}

% Cl

\note{G6102}{The rate coefficient is: \code{k_ClO_ClO} =
  \code{k_3rd_iupac(temp,cair,2.E-32,4.,1.E-11,0.,0.45)}.}

\note{G6103}{The rate coefficient is defined as backward reaction
  divided by equilibrium constant.}

\note{G6204}{At low temperatures, there may be a minor reaction channel
  leading to \kpp{O3}+\kpp{HCl}. See \citet{1626} for details. It is
  neglected here.}

\note{G6402}{The initial products are probably \chem{HCl} and
  \chem{CH_2OOH} \citep{1759}. It is assumed that \chem{CH_2OOH}
  dissociates into \chem{HCHO} and \chem{OH}.}

\note{G6405}{Average of reactions with \chem{CH_3Br} and \chem{CH_3F}
  from \citet{1945} (B.\ Steil, pers.\ comm.).}

\note{G6407}{Rough extrapolation from reactions with \chem{CH_3CF_3},
  \chem{CH_3CClF_2}, and \chem{CH_3CCl_2F} from \citet{1945}.}

\note{G6409}{It is assumed that the reaction liberates all \kpp{Cl}
  atoms in the form of \kpp{HCl}.}

% Br

\note{G7302}{The rate coefficient is: \code{k_BrO_NO2} =
  \code{k_3rd(temp,cair,5.2E-31,3.2,6.9E-12,2.9,0.6)}.}

\note{G7303}{The rate coefficient is defined as backward reaction
  \citep{1845} divided by equilibrium constant \citep{1054}.}

\note{G7404}{It is assumed that the reaction liberates all \kpp{Br}
  atoms in the form of \kpp{HBr}.}

\note{G7407}{It is assumed that the reaction liberates all \kpp{Br}
  atoms. The fate of the carbon atom is currently not considered.}

\note{G7408}{It is assumed that the reaction liberates all \kpp{Br}
  atoms. The fate of the carbon atom is currently not considered.}

\note{G7605}{Same value as for G7408: \kpp{CH2Br2}+\kpp{OH} assumed. It
  is assumed that the reaction liberates all \kpp{Br} atoms but not
  \kpp{Cl}. The fate of the carbon atom is currently not considered.}

\note{G7606}{Same value as for G7408: \kpp{CH2Br2}+\kpp{OH} assumed. It
  is assumed that the reaction liberates all \kpp{Br} atoms but not
  \kpp{Cl}. The fate of the carbon atom is currently not considered.}

\note{G7607}{It is assumed that the reaction liberates all \kpp{Br}
  atoms but not \kpp{Cl}. The fate of the carbon atom is currently not
  considered.}

% I

\note{G8102}{It is assumed that the reaction produces new particles.}

\note{G8103}{The yield of 38~\unit{\%} \chem{OIO} is from \citet{1845}.
  It is assumed here that the remaining 62~\unit{\%} produce 2 \chem{I}
  + \chem{O_2}.}

\note{G8300}{The rate coefficient is: \code{k_I_NO2} =
  \code{k_3rd_iupac(temp,cair,3.E-31,1.,6.6E-11,0.,0.63)}.}

\note{G8305}{The rate coefficient is defined as backward reaction
  \citep{1845} divided by equilibrium constant \citep{1644}.}

\note{G8306}{According to John Plane and John Crowley (pers.\ comm.\
  2007), the rate coefficient of \code{1.1E15*EXP(-12060./temp)}
  suggested by \citet{1845} is wrong.}

\note{G8401}{The rate coefficient is from \citet{2079}, the yield of
  \chem{I} atoms is a lower limit given on page 2170 of \citet{2089}.}

\note{G8402}{The products are from \citet{2087}.}

\note{G8701}{80\% \chem{Br} + \chem{OIO} production is from
  \citet{1845}. The remaining channels are assumed to produce \chem{Br}
  + \chem{I} + \chem{O_2}.}

% S

\note{G9400a}{Abstraction path. The assumed reaction sequence (omitting
  \chem{H_2O} and \chem{O_2} as products) according to \citet{243} is:
  \begin{eqnarray*}
    \chem{DMS} + \chem{OH}         & \TO & \chem{CH_3SCH_2}\\
    \chem{CH_3SCH_2} + \chem{O_2}  & \TO & \chem{CH_3SCH_2OO}\\
    \chem{CH_3SCH_2OO} + \chem{NO} & \TO & \chem{CH_3SCH_2O} + \chem{NO_2}\\
    \chem{CH_3SCH_2O}              & \TO & \chem{CH_3S} + \chem{HCHO}\\
    \chem{CH_3S} + \chem{O_3}      & \TO & \chem{CH_3SO}\\
    \chem{CH_3SO} + \chem{O_3}     & \TO & \chem{CH_3SO_2}\\
    \hline                                                  
    \chem{DMS}+\chem{OH}+\chem{NO}+2\chem{O_3} & \TO &
    \chem{CH_3SO_2}+\chem{HCHO}+\chem{NO_2}
  \end{eqnarray*}
  Neglecting the effect on \chem{O_3} and \chem{NO_x}, the remaining
  reaction is:
  $$\chem{DMS} + \chem{OH} + \chem{O_3} \TO \chem{CH_3SO_2} + \chem{HCHO}$$}

\note{G9400}{Addition path. The rate coefficient is: \code{k_DMS_OH} =
  \code{1.0E-39*EXP(5820./temp)*C(ind_O2)/
    (1.+5.0E-30*EXP(6280./temp)*C(ind_O2))}.}

% Hg

\note{G10201}{Upper limit.}

\end{multicols}

\clearpage

\begin{longtable}{llp{10cm}p{6cm}p{4cm}}
\caption{Photolysis reactions}\\
\hline
\# & labels & reaction & rate coefficient & reference\\
\hline
\endfirsthead
\caption{Photolysis reactions (... continued)}\\
\hline
\# & labels & reaction & rate coefficient & reference\\
\hline
\endhead
\hline
\endfoot
% this file was created automatically by eqn2tex, do not edit!
\code{J1000} & StTrGJ &  \kpp{O2} + \kpp{hv} $\rightarrow$ \kpp{O3P} + \kpp{O3P} & \code{jx(ip_O2)*EXP(LOG(1.25)*mcexp(190))} & see note\shownote{J1000}\\
\code{J1001a} & StTrGJ &  \kpp{O3} + \kpp{hv} $\rightarrow$ \kpp{O1D} & \code{jx(ip_O1D)*EXP(LOG(1.25)*mcexp(191))} & see note\shownote{J1001}\\
\code{J1001b} & StTrGJ &  \kpp{O3} + \kpp{hv} $\rightarrow$ \kpp{O3P} & \code{jx(ip_O3P)*EXP(LOG(1.25)*mcexp(192))} & see note\shownote{J1001}\\
\myhline
\code{J2101} & StTrGJ &  \kpp{H2O2} + \kpp{hv} $\rightarrow$ 2 \kpp{OH} & \code{jx(ip_H2O2)*EXP(LOG(1.25)*mcexp(193))} & see note\shownote{J2101}\\
\myhline
\code{J3101} & StTrGNJ &  \kpp{NO2} + \kpp{hv} $\rightarrow$ \kpp{NO} + \kpp{O3P} & \code{jx(ip_NO2)*EXP(LOG(1.25)*mcexp(194))} & see note\shownote{J3101}\\
\code{J3103a} & StTrGNJ &  \kpp{NO3} + \kpp{hv} $\rightarrow$ \kpp{NO2} + \kpp{O3P} & \code{jx(ip_NO2O)*EXP(LOG(1.25)*mcexp(195))} & see note\shownote{J3103}\\
\code{J3103b} & StTrGNJ &  \kpp{NO3} + \kpp{hv} $\rightarrow$ \kpp{NO} & \code{jx(ip_NOO2)*EXP(LOG(1.25)*mcexp(196))} & see note\shownote{J3103}\\
\code{J3104a} & StTrGNJ &  \kpp{N2O5} + \kpp{hv} $\rightarrow$ \kpp{NO2} + \kpp{NO3} & \code{jx(ip_N2O5)*EXP(LOG(1.25)*mcexp(197))} & see note\shownote{J3104}\\
\code{J3200} & TrGJ &  \kpp{HONO} + \kpp{hv} $\rightarrow$ \kpp{NO} + \kpp{OH} & \code{jx(ip_HONO)*EXP(LOG(1.25)*mcexp(198))} & see note\shownote{J3200}\\
\code{J3201} & StTrGNJ &  \kpp{HNO3} + \kpp{hv} $\rightarrow$ \kpp{NO2} + \kpp{OH} & \code{jx(ip_HNO3)*EXP(LOG(1.25)*mcexp(199))} & see note\shownote{J3201}\\
\code{J3202} & StTrGNJ &  \kpp{HNO4} + \kpp{hv} $\rightarrow$ .667 \kpp{NO2} + .667 \kpp{HO2} + .333 \kpp{NO3} + .333 \kpp{OH} & \code{jx(ip_HNO4)*EXP(LOG(1.25)*mcexp(200))} & see note\shownote{J3202}\\
\myhline
\code{J4100} & StTrGJ &  \kpp{CH3OOH} + \kpp{hv} $\rightarrow$ \kpp{HCHO} + \kpp{OH} + \kpp{HO2} & \code{jx(ip_CH3OOH)*EXP(LOG(1.25)*mcexp(201))} & see note\shownote{J4100}\\
\code{J4101a} & StTrGJ &  \kpp{HCHO} + \kpp{hv} $\rightarrow$ \kpp{H2} + \kpp{CO} & \code{jx(ip_COH2)*EXP(LOG(1.25)*mcexp(202))} & see note\shownote{J4101}\\
\code{J4101b} & StTrGJ &  \kpp{HCHO} + \kpp{hv} $\rightarrow$ \kpp{H} + \kpp{CO} + \kpp{HO2} & \code{jx(ip_CHOH)*EXP(LOG(1.25)*mcexp(203))} & see note\shownote{J4101}\\
\code{J4200} & TrGCJ &  \kpp{C2H5OOH} + \kpp{hv} $\rightarrow$ \kpp{CH3CHO} + \kpp{HO2} + \kpp{OH} & \code{jx(ip_CH3OOH)*EXP(LOG(1.25)*mcexp(204))} & \citet{1612}$^*$\shownote{J4200}\\
\code{J4201} & TrGCJ &  \kpp{CH3CHO} + \kpp{hv} $\rightarrow$ \kpp{CH3O2} + \kpp{HO2} + \kpp{CO} & \code{jx(ip_CH3CHO)*EXP(LOG(1.25)*mcexp(205))} & see note\shownote{J4201}\\
\code{J4202} & TrGCJ &  \kpp{CH3CO3H} + \kpp{hv} $\rightarrow$ \kpp{CH3O2} + \kpp{OH} + \kpp{CO2} & \code{jx(ip_CH3CO3H)*EXP(LOG(1.25)*mcexp(206))} & see note\shownote{J4202}\\
\code{J4204} & TrGNCJ &  \kpp{PAN} + \kpp{hv} $\rightarrow$ \kpp{CH3CO3} + \kpp{NO2} & \code{jx(ip_PAN)*EXP(LOG(1.25)*mcexp(207))} & see note\shownote{J4204}\\
\code{J4205} & TrGCJ &  \kpp{HOCH2CHO} + \kpp{hv} $\rightarrow$ \kpp{HO2} + \kpp{HCHO} + \kpp{HO2} + \kpp{CO} & \code{jx(ip_HOCH2CHO)*EXP(LOG(1.25)*mcexp(208))} & see note\shownote{J4205}\\
\code{J4206} & TrGCJ &  \kpp{HOCH2CO3H} + \kpp{hv} $\rightarrow$ \kpp{HCHO} + \kpp{HO2} + \kpp{OH} + \kpp{CO2} & \code{jx(ip_CH3OOH)*EXP(LOG(1.25)*mcexp(209))} & \citet{2419}$^*$\shownote{J4206}\\
\code{J4207} & TrGCJ &  \kpp{PHAN} + \kpp{hv} $\rightarrow$ \kpp{HOCH2CO3} + \kpp{NO2} & \code{jx(ip_PAN)*EXP(LOG(1.25)*mcexp(210))} & see note\shownote{J4207}\\
\code{J4208} & TrGCJ &  \kpp{GLYOX} + \kpp{hv} $\rightarrow$ 2 \kpp{CO} + 2 \kpp{HO2} & \code{jx(ip_GLYOX)*EXP(LOG(1.25)*mcexp(211))} & see note\shownote{J4208}\\
\code{J4209} & TrGNCJ &  \kpp{HCOCO2H} + \kpp{hv} $\rightarrow$ 2 \kpp{HO2} + \kpp{CO} + \kpp{CO2} & \code{jx(ip_MGLYOX)*EXP(LOG(1.25)*mcexp(212))} & \citet{2419}$^*$\shownote{J4209}\\
\code{J4210} & TrGNCJ &  \kpp{HCOCO3H} + \kpp{hv} $\rightarrow$ \kpp{HO2} + \kpp{CO} + \kpp{OH} + \kpp{CO2} & \code{(jx(ip_CH3OOH)+jx(ip_HOCH2CHO))*EXP(LOG(1.25)*mcexp(213))} & \citet{2419}$^*$\shownote{J4210}\\
\code{J4211} & TrGCJ &  \kpp{HYETHO2H} + \kpp{hv} $\rightarrow$ \kpp{HOCH2CH2O} + \kpp{OH} & \code{jx(ip_CH3OOH)*EXP(LOG(1.25)*mcexp(214))} & \citet{2419}$^*$\shownote{J4211}\\
\code{J4212} & TrGCJ &  \kpp{ETHOHNO3} + \kpp{hv} $\rightarrow$ \kpp{HO2} + 2 \kpp{HCHO} + \kpp{NO2} & \code{J_IC3H7NO3} & see note\shownote{J4212}\\
\code{J4300} & TrGCJ &  \kpp{IC3H7OOH} + \kpp{hv} $\rightarrow$ \kpp{CH3COCH3} + \kpp{HO2} + \kpp{OH} & \code{jx(ip_CH3OOH)*EXP(LOG(1.25)*mcexp(215))} & \citet{1612}$^*$\shownote{J4300}\\
\code{J4301} & TrGCJ &  \kpp{CH3COCH3} + \kpp{hv} $\rightarrow$ \kpp{CH3CO3} + \kpp{CH3O2} & \code{jx(ip_CH3COCH3)*EXP(LOG(1.25)*mcexp(216))} & see note\shownote{J4301}\\
\code{J4302} & TrGCJ &  \kpp{ACETOL} + \kpp{hv} $\rightarrow$ \kpp{CH3CO3} + \kpp{HCHO} + \kpp{HO2} & \code{J_ACETOL} & see note\shownote{J4302}\\
\code{J4303} & TrGCJ &  \kpp{MGLYOX} + \kpp{hv} $\rightarrow$ \kpp{CH3CO3} + \kpp{CO} + \kpp{HO2} & \code{jx(ip_MGLYOX)*EXP(LOG(1.25)*mcexp(217))} & see note\shownote{J4303}\\
\code{J4304} & TrGCJ &  \kpp{HYPERACET} + \kpp{hv} $\rightarrow$ \kpp{CH3CO3} + \kpp{HCHO} + \kpp{OH} & \code{jx(ip_CH3OOH)*EXP(LOG(1.25)*mcexp(218))+J_ACETOL} & \citet{2419}$^*$\shownote{J4304}\\
\code{J4306} & TrGNCJ &  \kpp{IC3H7NO3} + \kpp{hv} $\rightarrow$ \kpp{CH3COCH3} + \kpp{NO2} + \kpp{HO2} & \code{J_IC3H7NO3} & \citet{1584}$^*$\shownote{J4306}\\
\code{J4307} & TrGCJ &  \kpp{NOA} + \kpp{hv} $\rightarrow$ \kpp{CH3CO3} + \kpp{HCHO} + \kpp{NO2} & \code{J_IC3H7NO3+jx(ip_CH3COCH3)*EXP(LOG(1.25)*mcexp(219))} & see note\shownote{J4307}\\
\code{J4308} & TrGCJ &  \kpp{HOCH2COCO2H} + \kpp{hv} $\rightarrow$ \kpp{HOCH2CO3} + \kpp{HO2} + \kpp{CO2} & \code{jx(ip_MGLYOX)*EXP(LOG(1.25)*mcexp(220))} & \citet{2419}$^*$\shownote{J4308}\\
\code{J4309} & TrGCJ &  \kpp{HYPROPO2H} + \kpp{hv} $\rightarrow$ \kpp{CH3CHO} + \kpp{HCHO} + \kpp{HO2} + \kpp{OH} & \code{jx(ip_CH3OOH)*EXP(LOG(1.25)*mcexp(221))} & \citet{2419}$^*$\shownote{J4309}\\
\code{J4310} & TrGNCJ &  \kpp{PR2O2HNO3} + \kpp{hv} $\rightarrow$ \kpp{NOA} + \kpp{HO2} + \kpp{OH} & \code{jx(ip_CH3OOH)*EXP(LOG(1.25)*mcexp(222))} & \citet{2419}$^*$\shownote{J4310}\\
\code{J4311} & TrGCJ &  \kpp{HOCH2COCHO} + \kpp{hv} $\rightarrow$ \kpp{HOCH2CO3} + \kpp{CO} + \kpp{HO2} & \code{jx(ip_MGLYOX)*EXP(LOG(1.25)*mcexp(223))} & \citet{2419}$^*$\shownote{J4311}\\
\code{J4400} & TrGCJ &  \kpp{LC4H9OOH} + \kpp{hv} $\rightarrow$ \kpp{OH} + 0.254 \kpp{CO2} + 0.5552 \kpp{MEK} + 0.5552 \kpp{HO2} + 0.3178 \kpp{CH3CHO} + 0.4448 \kpp{C2H5O2} & \code{jx(ip_CH3OOH)*EXP(LOG(1.25)*mcexp(224))} & \citet{2419}$^*$\shownote{J4400}\\
\code{J4401} & TrGCJ &  \kpp{MVK} + \kpp{hv} $\rightarrow$ .5 \kpp{C3H6} + .5 \kpp{CH3CO3} + .5 \kpp{HCHO} + \kpp{CO} + .5 \kpp{HO2} & \code{jx(ip_MVK)*EXP(LOG(1.25)*mcexp(225))} & see note\shownote{J4401}\\
\code{J4403} & TrGCJ &  \kpp{MEK} + \kpp{hv} $\rightarrow$ \kpp{CH3CO3} + \kpp{C2H5O2} & \code{0.42*jx(ip_CHOH)*EXP(LOG(1.25)*mcexp(226))} & \citet{1584}$^*$\shownote{J4403}\\
\code{J4404} & TrGCJ &  \kpp{LMEKOOH} + \kpp{hv} $\rightarrow$ 0.538 \kpp{HCHO} + 0.538 \kpp{CO2} + 0.459 \kpp{HOCH2CH2O2} + 0.079 \kpp{C2H5O2} + 0.462 \kpp{CH3CO3} + 0.462 \kpp{CH3CHO} + \kpp{OH} & \code{jx(ip_CH3OOH)*EXP(LOG(1.25)*mcexp(227))} & \citet{2419}$^*$\shownote{J4404}\\
\code{J4405} & TrGCJ &  \kpp{BIACET} + \kpp{hv} $\rightarrow$ 2 \kpp{CH3CO3} & \code{2.15*jx(ip_MGLYOX)*EXP(LOG(1.25)*mcexp(228))} & see note\shownote{J4405}\\
\code{J4406} & TrGNCJ &  \kpp{LC4H9NO3} + \kpp{hv} $\rightarrow$ \kpp{NO2} + 0.254 \kpp{CO2} + 0.5552 \kpp{MEK} + 0.5552 \kpp{HO2} + 0.3178 \kpp{CH3CHO} + 0.4448 \kpp{C2H5O2} & \code{J_IC3H7NO3} & see note\shownote{J4406}\\
\code{J4407} & TrGNCJ &  \kpp{MPAN} + \kpp{hv} $\rightarrow$ \kpp{MACO3} + \kpp{NO2} & \code{jx(ip_PAN)*EXP(LOG(1.25)*mcexp(229))} & see note\shownote{J4407}\\
\code{J4408} & TrGCJ &  \kpp{LMVKOHABOOH} + \kpp{hv} $\rightarrow$ .3 \kpp{HOCH2COCHO} + .3 \kpp{HCHO} + .3 \kpp{HO2} + .7 \kpp{HOCH2CHO} + .7 \kpp{HOCH2CO3} + \kpp{OH} & \code{J_ACETOL+jx(ip_CH3OOH)*EXP(LOG(1.25)*mcexp(230))} & \citet{2419}$^*$\shownote{J4408}\\
\code{J4409} & TrGCJ &  \kpp{CO2H3CO3H} + \kpp{hv} $\rightarrow$ \kpp{MGLYOX} + \kpp{HO2} + \kpp{OH} + \kpp{CO2} & \code{jx(ip_CH3OOH)*EXP(LOG(1.25)*mcexp(231))} & \citet{2419}$^*$\shownote{J4409}\\
\code{J4410} & TrGCJ &  \kpp{CO2H3CO3H} + \kpp{hv} $\rightarrow$ \kpp{CH3CO3} + \kpp{HO2} + \kpp{HCOCO3H} & \code{J_ACETOL} & \citet{2419}$^*$\shownote{J4410}\\
\code{J4411} & TrGCJ &  \kpp{MACR} + \kpp{hv} $\rightarrow$ .5 \kpp{MACO3} + .5 \kpp{CH3CO3} + .5 \kpp{HCHO} + .5 \kpp{CO} + \kpp{HO2} & \code{jx(ip_MACR)*EXP(LOG(1.25)*mcexp(232))} & see note\shownote{J4411}\\
\code{J4412} & TrGCJ &  \kpp{MACROOH} + \kpp{hv} $\rightarrow$ \kpp{ACETOL} + \kpp{HCHO} + \kpp{HO2} + \kpp{OH} & \code{jx(ip_CH3OOH)*EXP(LOG(1.25)*mcexp(233))} & \citet{2419}$^*$\shownote{J4412}\\
\code{J4413} & TrGCJ &  \kpp{MACROOH} + \kpp{hv} $\rightarrow$ \kpp{ACETOL} + \kpp{CO} + \kpp{HO2} + \kpp{OH} & \code{2.77*jx(ip_HOCH2CHO)*EXP(LOG(1.25)*mcexp(234))} & see note\shownote{J4413}\\
\code{J4414} & TrGCJ &  \kpp{MACROH} + \kpp{hv} $\rightarrow$ \kpp{ACETOL} + \kpp{CO} + \kpp{HO2} + \kpp{HO2} & \code{2.77*jx(ip_HOCH2CHO)*EXP(LOG(1.25)*mcexp(235))} & see note\shownote{J4414}\\
\code{J4415} & TrGCJ &  \kpp{MACO3H} + \kpp{hv} $\rightarrow$ \kpp{CH3CO3} + \kpp{HCHO} + \kpp{OH} + \kpp{CO2} & \code{jx(ip_CH3OOH)*EXP(LOG(1.25)*mcexp(236))} & \citet{2419}$^*$\shownote{J4415}\\
\code{J4416} & TrGCJ &  \kpp{LHMVKABOOH} + \kpp{hv} $\rightarrow$ .3 \kpp{MGLYOX} + .7 \kpp{CH3CO3} + .7 \kpp{HOCH2CHO} + .3 \kpp{HCHO} + .3 \kpp{HO2} + \kpp{OH} & \code{jx(ip_CH3OOH)*EXP(LOG(1.25)*mcexp(237))} & \citet{2419}$^*$\shownote{J4416}\\
\code{J4417} & TrGCJ &  \kpp{MVKOH} + \kpp{hv} $\rightarrow$ .5 \kpp{HCHO} + .5 \kpp{HO2} + .5 \kpp{HOCH2CO3} + \kpp{CO} + 1.5 \kpp{LCARBON} & \code{jx(ip_MVK)*EXP(LOG(1.25)*mcexp(238))} & \citet{2419}$^*$\shownote{J4417}\\
\code{J4418} & TrGCJ &  \kpp{CO2H3CHO} + \kpp{hv} $\rightarrow$ \kpp{MGLYOX} + \kpp{CO} + \kpp{HO2} + \kpp{HO2} & \code{jx(ip_HOCH2CHO)*EXP(LOG(1.25)*mcexp(239))} & \citet{2419}$^*$\shownote{J4418}\\
\code{J4419} & TrGCJ &  \kpp{HO12CO3C4} + \kpp{hv} $\rightarrow$ \kpp{CH3CO3} + \kpp{HOCH2CHO} + \kpp{HO2} & \code{J_ACETOL} & \citet{2419}$^*$\shownote{J4419}\\
\code{J4420} & TrGCJ &  \kpp{BIACETOH} + \kpp{hv} $\rightarrow$ \kpp{CH3CO3} + \kpp{HOCH2CO3} & \code{2.15*jx(ip_MGLYOX)*EXP(LOG(1.25)*mcexp(240))} & see note\shownote{J4420}\\
\code{J4502} & TrGCJ &  \kpp{LISOPACOOH} + \kpp{hv} $\rightarrow$ \kpp{LHC4ACCHO} + \kpp{HO2} + \kpp{OH} & \code{jx(ip_CH3OOH)*EXP(LOG(1.25)*mcexp(241))} & \citet{2419}$^*$\shownote{J4502}\\
\code{J4503} & TrGNCJ &  \kpp{LISOPACNO3} + \kpp{hv} $\rightarrow$ \kpp{LHC4ACCHO} + \kpp{HO2} + \kpp{NO2} & \code{0.59*J_IC3H7NO3} & see note\shownote{J4503}\\
\code{J4504} & TrGCJ &  \kpp{ISOPBOOH} + \kpp{hv} $\rightarrow$ .75 \kpp{MVK} + .25 \kpp{MVKOH} + .75 \kpp{HCHO} + .75 \kpp{HO2} + .25 \kpp{CH3O2} + \kpp{OH} & \code{jx(ip_CH3OOH)*EXP(LOG(1.25)*mcexp(242))} & \citet{2419}$^*$\shownote{J4504}\\
\code{J4505} & TrGNCJ &  \kpp{ISOPBNO3} + \kpp{hv} $\rightarrow$ .75 \kpp{MVK} + .25 \kpp{MVKOH} + .75 \kpp{HCHO} + .75 \kpp{HO2} + .25 \kpp{CH3O2} + \kpp{NO2} & \code{2.84*J_IC3H7NO3} & see note\shownote{J4505}\\
\code{J4506} & TrGCJ &  \kpp{ISOPDOOH} + \kpp{hv} $\rightarrow$ \kpp{MACR} + \kpp{HCHO} + \kpp{HO2} + \kpp{OH} & \code{jx(ip_CH3OOH)*EXP(LOG(1.25)*mcexp(243))} & \citet{2419}$^*$\shownote{J4506}\\
\code{J4507} & TrGNCJ &  \kpp{ISOPDNO3} + \kpp{hv} $\rightarrow$ \kpp{MACR} + \kpp{HCHO} + \kpp{HO2} + \kpp{NO2} & \code{J_IC3H7NO3} & see note\shownote{J4507}\\
\code{J4508} & TrGNCJ &  \kpp{NISOPOOH} + \kpp{hv} $\rightarrow$ \kpp{NC4CHO} + \kpp{HO2} + \kpp{OH} & \code{jx(ip_CH3OOH)*EXP(LOG(1.25)*mcexp(244))} & \citet{2419}$^*$\shownote{J4508}\\
\code{J4509} & TrGNCJ &  \kpp{NC4CHO} + \kpp{hv} $\rightarrow$ \kpp{NOA} + 2 \kpp{CO} + 2 \kpp{HO2} & \code{jx(ip_MACR)*EXP(LOG(1.25)*mcexp(245))} & see note\shownote{J4509}\\
\code{J4510} & TrGNCJ &  \kpp{LNISOOH} + \kpp{hv} $\rightarrow$ \kpp{NOA} + \kpp{OH} + .5 \kpp{GLYOX} + .5 \kpp{CO} + \kpp{HO2} + .5 \kpp{CO2} & \code{jx(ip_CH3OOH)*EXP(LOG(1.25)*mcexp(246))} & \citet{2272}$^*$\shownote{J4510}\\
\code{J4511} & TrGCJ &  \kpp{LHC4ACCHO} + \kpp{hv} $\rightarrow$ .5 \kpp{LHC4ACCO3} + .25 \kpp{ACETOL} + .25 \kpp{HOCH2CHO} + .25 \kpp{CH3CO3} + .75 \kpp{CO} + 1.25 \kpp{HO2} & \code{jx(ip_MACR)*EXP(LOG(1.25)*mcexp(247))} & \citet{2419}$^*$\shownote{J4511}\\
\code{J4512} & TrGCJ &  \kpp{LC578OOH} + \kpp{hv} $\rightarrow$ .5 \kpp{ACETOL} + .5 \kpp{MGLYOX} + .5 \kpp{GLYOX} + .5 \kpp{HOCH2CHO} + \kpp{HO2} + \kpp{OH} & \code{jx(ip_CH3OOH)*EXP(LOG(1.25)*mcexp(248))} & \citet{2272}$^*$\shownote{J4512}\\
\code{J4513} & TrGCJ &  \kpp{LHC4ACCO3H} + \kpp{hv} $\rightarrow$ .5 \kpp{ACETOL} + .5 \kpp{HOCH2CHO} + .5 \kpp{CH3CO3} + .5 \kpp{CO} + .5 \kpp{HO2} + \kpp{OH} + \kpp{CO2} & \code{jx(ip_CH3OOH)*EXP(LOG(1.25)*mcexp(249))} & \citet{2419}$^*$\shownote{J4513}\\
\code{J4514} & TrGNCJ &  \kpp{LC5PAN1719} + \kpp{hv} $\rightarrow$ .5 \kpp{MACROH} + .5 \kpp{HO12CO3C4} + \kpp{CO} + \kpp{NO2} & \code{jx(ip_PAN)*EXP(LOG(1.25)*mcexp(250))} & see note\shownote{J4514}\\
\code{J4515} & TrGCJ &  \kpp{HCOC5} + \kpp{hv} $\rightarrow$ \kpp{CH3CO3} + \kpp{HCHO} + \kpp{HOCH2CO3} & \code{0.5*jx(ip_MVK)*EXP(LOG(1.25)*mcexp(251))} & see note\shownote{J4515}\\
\code{J4516} & TrGCJ &  \kpp{C59OOH} + \kpp{hv} $\rightarrow$ \kpp{ACETOL} + \kpp{HOCH2CO3} + \kpp{OH} & \code{J_ACETOL+jx(ip_CH3OOH)*EXP(LOG(1.25)*mcexp(252))} & \citet{2419}$^*$\shownote{J4516}\\
\myhline
\myhline
\myhline

\input{mecca_eqn_ph.tex}
\end{longtable}

\begin{multicols}{3}
$^*$Notes:

J-values are calculated with an external module and then supplied to the
MECCA chemistry. 

Values that originate from the Master Chemical Mechanism (MCM) by
\citet{2419} are translated according in the following way:\\
J(11)             $\rightarrow$ \code{jx(ip_COH2)}     \\
J(12)             $\rightarrow$ \code{jx(ip_CHOH)}     \\
J(15)             $\rightarrow$ \code{jx(ip_HOCH2CHO)} \\
J(18)             $\rightarrow$ \code{jx(ip_MACR)}     \\
J(22)             $\rightarrow$ \code{jx(ip_ACETOL)}   \\
J(23)+J(24)       $\rightarrow$ \code{jx(ip_MVK)}      \\
J(31)+J(32)+J(33) $\rightarrow$ \code{jx(ip_GLYOX)}    \\
J(34)             $\rightarrow$ \code{jx(ip_MGLYOX)}   \\
J(41)             $\rightarrow$ \code{jx(ip_CH3OOH)}   \\
J(53)             $\rightarrow$ J(\kpp{IC3H7NO3})      \\
J(54)             $\rightarrow$ J(\kpp{IC3H7NO3})      \\
J(55)             $\rightarrow$ J(\kpp{IC3H7NO3})      \\
J(56)+J(57)       $\rightarrow$ \code{jx(ip_NOA)}

\note{J4207}{It is assumed that J(\kpp{PHAN}) is the same as J(\kpp{PAN}).}

\note{J4212}{It is assumed that J(\kpp{ETHOHNO3}) is the same as
  J(\kpp{IC3H7NO3}).}

\note{J4302}{Following \citet{1584}, we use J(\kpp{ACETOL}) =
  \code{0.11*jx(ip_CHOH)}. As an additional factor, the quantum yield of
  0.65 is taken from \citet{2510}.}

\note{J4306}{Following \citet{1584}, we use J(\kpp{IC3H7NO3}) =
  \code{3.7*jx(ip_PAN)}.}

\note{J4307}{\kpp{NOA} contains the cromophores of both \kpp{CH3COCH3}
  and a nitrate group. It is assumed here that the J values are
  additive, i.e.: J(\kpp{NOA}) = J(\kpp{CH3COCH3}) + J(\kpp{IC3H7NO3}).}

\note{J4406}{It is assumed that J(\kpp{LC4H9NO3}) is the same as
  J(\kpp{IC3H7NO3}).}

\note{J4407}{It is assumed that J(\kpp{MPAN}) is the same as J(\kpp{PAN}).}

\note{J4405}{It is assumed that J(\kpp{BIACET}) is 2.15 times larger
  than J(\kpp{MGLYOX}), consistent with the photolysis rate coefficients
  used in the MCM \citep{2419}.}

\note{J4413}{It is assumed that J(\kpp{MACROOH}) is 2.77 times larger
  than J(\kpp{HOCH2CHO}), consistent with the photolysis rate
  coefficients used in the MCM \citep{2419}.}

\note{J4414}{It is assumed that J(\kpp{MACROH}) is 2.77 times larger
  than J(\kpp{HOCH2CHO}), consistent with the photolysis rate
  coefficients used in the MCM \citep{2419}.}

\note{J4420}{It is assumed that J(\kpp{BIACETOH}) is 2.15 times larger
  than J(\kpp{MGLYOX}), consistent with the photolysis rate coefficients
  used in the MCM \citep{2419}.}

\note{J4503}{It is assumed that J(\kpp{LISOPACNO3}) = 0.59 $\times$
  J(\kpp{IC3H7NO3}), consistent with the photolysis rate coefficients
  used in the MCM \citep{2419}.}

\note{J4505}{It is assumed that J(\kpp{ISOPBNO3}) = 2.84 $\times$
  J(\kpp{IC3H7NO3}), consistent with the photolysis rate coefficients
  used in the MCM \citep{2419}.}

\note{J4509}{It is assumed that J(\kpp{NC4CHO}) is the same as
  J(\kpp{MACR}).}

\note{J4514}{It is assumed that J(\kpp{LC5PAN1719}) is the same as
  J(\kpp{PAN}).}

\note{J4515}{Consistent with the MCM \citep{2419}, we assume that
  J(\kpp{HCOC5}) is half as large as J(\kpp{MVK}).}

\note{J6100}{\citet{1741} claim that the combination of absorption cross
  sections from \citet{1746} and the \chem{Cl_2O_2} formation rate
  coefficient by \citet{1555} can approximately reproduce the observed
  \chem{Cl_2O_2}/\chem{ClO} ratios and ozone depletion. They give an
  almost zenith-angle independent ratio of 1.4 for \citet{1746} to
  \citet{1555} J-values. The IUPAC recommendation for the \chem{Cl_2O_2}
  formation rate is about 5 to 15 \% less than the value by \citet{1555}
  but more than 20 \% larger than the value by \citet{1284}. The
  J-values by \citet{1746} are within the uncertainty range of the IUPAC
  recommendation.}

\note{J7301}{The quantum yields are from \citet{1555}.}

\end{multicols}

\clearpage

\begin{longtable}{llrl}
\caption{Henry's law coefficients}\\
\hline
substance & 
$\rule[-2.5ex]{0ex}{2ex}\DS\frac{k_{\rm H}^{\ominus}}{\rm M/atm}$ &
$\DS\frac{-\Delta_{\rm soln}H/R}{\rm K}$ &
reference\\
\hline 
\endfirsthead
\caption{Henry's law coefficients (... continued)}\\
\hline
substance & 
$\rule[-2.5ex]{0ex}{2ex}\DS\frac{k_{\rm H}^{\ominus}}{\rm M/atm}$ &
$\DS\frac{-\Delta_{\rm soln}H/R}{\rm K}$ &
reference\\
\hline 
\endhead
\hline
\endfoot
% this file was created automatically by henry2tex, do not edit!
\kpp{O2} & 1.3\E{-3} & 1500. & \citet{190} \\
\kpp{O3} & 1.2\E{-2} & 2560. & \citet{87} \\
\kpp{OH} & 3.0\E{1} & 4300. & \citet{515} \\
\kpp{HO2} & 3.9\E{3} & 5900. & \citet{515} \\
\kpp{H2O2} & 1.\E{5} & 6338. & \citet{311} \\
\kpp{NH3} & 58. & 4085. & \citet{87} \\
\kpp{NO} & 1.9\E{-3} & 1480. & \citet{449} \\
\kpp{NO2} & 7.0\E{-3} & 2500. & \citet{59}$^*$\showhenrynote{NO2} \\
\kpp{NO3} & 2. & 2000. & \citet{219} \\
\kpp{HONO} & 4.9\E{1} & 4780. & \citet{449} \\
\kpp{HNO3} & 2.45\E{6}/1.5\E{1} & 8694. & \citet{530}$^*$\showhenrynote{HNO3} \\
\kpp{HNO4} & 1.2\E{4} & 6900. & \citet{797} \\
\kpp{CH3O2} & 6. & 5600. & \citet{46}$^*$\showhenrynote{CH3O2} \\
\kpp{CH3OOH} & 3.0\E{2} & 5322. & \citet{311} \\
\kpp{HCHO} & 7.0\E{3} & 6425. & \citet{87} \\
\kpp{HCOOH} & 3.7\E{3} & 5700. & \citet{87} \\
\kpp{CO2} & 3.1\E{-2} & 2423. & \citet{87} \\

\end{longtable}

\begin{multicols}{3}
$^*$Notes:

The temperature dependence of the Henry constants is: $$K_{\rm H} =
K_{\rm H}^{\ominus} \times \exp \DS\left( \frac{-\Delta_{\rm soln}H}{R}
  \left( \frac{1}{T} - \frac{1}{T^{\ominus}} \right) \right)$$
where $\Delta_{\rm soln}H$~= molar enthalpy of dissolution
[\unit{J/mol}] and $R$~= 8.314 \unit{J/(mol~K)}.

\henrynote{NO2}{The temperature dependence is from \citet{87}.}

\henrynote{HNO3}{Calculated using the acidity constant from \citet{450}.}

\henrynote{CH3O2}{This value was estimated by \citet{46}.}

\henrynote{HBr}{Calculated using the acidity constant from \citet{194}.}

\henrynote{HOBr}{This value was estimated by \citet{446}.}

\henrynote{IO}{Assumed to be the same as $K_{\rm H}(\kpp{HOI})$.}

\henrynote{HOI}{Lower limit.}

\henrynote{ICl}{Calculated using thermodynamic data from \citet{489}.}

\henrynote{IBr}{Calculated using thermodynamic data from \citet{489}.}

\henrynote{H2SO4}{To account for the very high Henry's law coefficient
  of \kpp{H2SO4}, a very high value was chosen arbitrarily.}

\henrynote{DMSO}{Lower limit cited from another reference.}

\henrynote{HgBr2}{Assumed to be the same as for \kpp{HgCl2}}

\henrynote{ClHgBr}{Assumed to be the same as for \kpp{HgCl2}}

\henrynote{BrHgOBr}{Assumed to be the same as for \kpp{HgCl2}}

\henrynote{ClHgOBr}{Assumed to be the same as for \kpp{HgCl2}}

\end{multicols}

\clearpage

\begin{longtable}{llrl}
\caption{Accommodation coefficients}\\
\hline
substance & 
$\alpha^{\ominus}$ &
$\DS\frac{-\Delta_{\rm obs}H/R}{\rm K}$ &
reference\\
\hline 
\endfirsthead
\caption{Accommodation coefficients (... continued)}\\
\hline
substance & 
$\alpha^{\ominus}$ &
$\DS\frac{-\Delta_{\rm obs}H/R}{\rm K}$ &
reference\\
\hline 
\endhead
\hline
\endfoot 
% this file was created automatically by alpha2tex, do not edit!
\kpp{O2} & 0.01 & 2000. & see note\showalphanote{O2} \\
\kpp{O3} & 0.002 & 0. & \citet{826}$^*$\showalphanote{O3} \\
\kpp{OH} & 0.01 & 0. & \citet{1047}$^*$\showalphanote{OH} \\
\kpp{HO2} & 0.5 & 0. & \citet{1864} \\
\kpp{H2O2} & 0.077 & 3127. & \citet{32} \\
\kpp{NH3} & 0.06 & 0. & \citet{826}$^*$\showalphanote{NH3} \\
\kpp{NO} & 5.0\E{-5} & 0. & \citet{448}$^*$\showalphanote{NO} \\
\kpp{NO2} & 0.0015 & 0. & \citet{176}$^*$\showalphanote{NO2} \\
\kpp{NO3} & 0.04 & 0. & \citet{1048}$^*$\showalphanote{NO3} \\
\kpp{N2O5} & 0.1 & 0. & \citet{826}$^*$\showalphanote{N2O5} \\
\kpp{HONO} & 0.04 & 0. & \citet{826}$^*$\showalphanote{HONO} \\
\kpp{HNO3} & 0.5 & 0. & \citet{930}$^*$\showalphanote{HNO3} \\
\kpp{HNO4} & 0.1 & 0. & \citet{826}$^*$\showalphanote{HNO4} \\
\kpp{CH3O2} & 0.01 & 2000. & see note\showalphanote{CH3O2} \\
\kpp{CH3OOH} & 0.0046 & 3273. & \citet{844} \\
\kpp{HCHO} & 0.04 & 0. & \citet{826}$^*$\showalphanote{HCHO} \\
\kpp{HCOOH} & 0.014 & 3978. & \citet{826} \\
\kpp{CO2} & 0.01 & 2000. & see note\showalphanote{CO2} \\

\end{longtable}

\begin{multicols}{3}
$^*$Notes:

If no data are available, the following default values are used:\\
$\alpha^{\ominus}$~= 0.1\\
$-\Delta_{\rm obs}H/R$~= 0 K

The temperature dependence of the accommodation coefficients is given by
\citep{155}:

\begin{eqnarray*}
  \frac{\alpha}{1-\alpha} & = & \exp \left( \frac{-\Delta_{\rm obs}G}{RT}
  \right)\\
  & = & \exp \left( \frac{-\Delta_{\rm obs}H}{RT} + \frac{\Delta_{\rm
  obs}S}{R} \right)
\end{eqnarray*}

where $\Delta_{\rm obs}G$ is the Gibbs free energy barrier of the
transition state toward solution \citep{155}, and $\Delta_{\rm obs}H$
and $\Delta_{\rm obs}S$ are the corresponding enthalpy and entropy,
respectively. The equation can be rearranged to:

\begin{eqnarray*}
  \ln\left( \frac{\alpha}{1-\alpha} \right) & = & 
  \frac{-\Delta_{\rm obs}H}{R} \times \frac{1}{T} + \frac{-\Delta_{\rm
  obs}S}{R}\\
\end{eqnarray*}

and further:

\begin{eqnarray*}
  \dd\ln\left( \frac{\alpha}{1-\alpha} \right) \left/ \dd\left(
    \frac{1}{T} \right)\right. & = & \frac{-\Delta_{\rm obs}H}{R}
\end{eqnarray*}

If no data were available, a value of $\alpha$ = 0.01, $\alpha$ = 0.1,
or $\alpha$ = 0.5, and a temperature dependence of $-\Delta_{\rm
  obs}H/R$~= 2000~\unit{K} has been assumed.

\alphanote{O2}{Estimate.}

\alphanote{O3}{Value measured at 292 \unit{K}.}

\alphanote{OH}{Value measured at 293 \unit{K}.}

\alphanote{HO2}{Value for aqueous salts at 293 \unit{K}.}

\alphanote{NH3}{Value measured at 295 \unit{K}.}

\alphanote{NO}{Value measured between 193 and 243 \unit{K}.}

\alphanote{NO2}{Value measured at 298 \unit{K}.}

\alphanote{NO3}{Value is a lower limit, measured at 273 \unit{K}.}

\alphanote{N2O5}{Value for sulfuric acid, measured between 195 and 300
\unit{K}.}

\alphanote{HONO}{Value measured between 247 and 297 \unit{K}.}

\alphanote{HNO3}{Value measured at room temperature. \citet{930} say
$\gamma>0.2$. Here $\alpha=0.5$ is used.}

\alphanote{HNO4}{Value measured at 200 \unit{K} for water ice.}

\alphanote{CH3O2}{Estimate.}

\alphanote{HCHO}{Value measured between 260 and 270 \unit{K}.}

\alphanote{CO2}{Estimate.}

\alphanote{HCl}{Temperature dependence derived from published data at 2
different temperatures}

\alphanote{HOCl}{Assumed to be the same as $\alpha(\kpp{HOBr})$.}

\alphanote{ClNO3}{Value measured at 274.5 \unit{K}.}

\alphanote{HBr}{Temperature dependence derived from published data at 2
different temperatures}

\alphanote{HOBr}{Value measured at room temperature. \citet{930} say
$\gamma>0.2$. Here $\alpha=0.5$ is used.}

\alphanote{BrNO3}{Value measured at 273 \unit{K}.}

\alphanote{BrCl}{Assumed to be the same as $\alpha(\kpp{Cl2})$.}

\alphanote{I2}{Estimate.}

\alphanote{IO}{Estimate.}

\alphanote{OIO}{Estimate.}

\alphanote{I2O2}{Estimate.}

\alphanote{HI}{Temperature dependence derived from published data at 2
different temperatures}

\alphanote{HOI}{Assumed to be the same as $\alpha(\kpp{HOBr})$. See also
  \citet{1565} and \citet{1572}.}

\alphanote{HIO3}{Estimate.}

\alphanote{INO2}{Estimate.}

\alphanote{INO3}{Estimate.}

\alphanote{ICl}{Estimate.}

\alphanote{IBr}{Assumed to be the same as $\alpha(\kpp{ICl})$.}

\alphanote{H2SO4}{Value measured at 303 \unit{K}.}

\alphanote{Hg}{Estimate.}

\alphanote{HgO}{Estimate.}

\alphanote{HgCl2}{Estimate.}

\alphanote{HgBr2}{Estimate.}

\alphanote{ClHgBr}{Estimate.}

\alphanote{BrHgOBr}{Estimate.}

\alphanote{ClHgOBr}{Estimate.}

\end{multicols}

\clearpage

\begin{longtable}{llp{8cm}p{5cm}p{55mm}}
\caption{Reversible (Henry's law) equilibria and irreversible
  (``heterogenous'') uptake}\\
\hline
\# & labels & reaction & rate coefficient & reference\\
\hline
\endhead
\hline
\endfoot
\input{mecca_eqn_h.tex}
\end{longtable}

\begin{multicols}{3}
$^*$Notes:

The forward (\verb|k_exf|) and backward (\verb|k_exb|) rate coefficients
are calculated in the file \verb|messy_mecca_aero.f90| using the
accommodation coefficients in subroutine \verb|mecca_aero_alpha| and
Henry's law constants in subroutine \verb|mecca_aero_henry|.

$k_{\rm mt}$ = mass transfer coefficient

$\rm lwc$ = liquid water content of aerosol mode

H3201, H6300, H6301, H6302, H7300, H7301, H7302, H7601, H7602: For
uptake of X (X = \chem{N_2O_5}, \chem{ClNO_3}, or \chem{BrNO_3}) and
subsequent reaction with \chem{H_2O}, \chem{Cl^-}, and \chem{Br^-}, we
define:
$$k_{\rm exf}(\chem{X}) = \frac{k_{\rm mt}(\chem{X})\times {\rm LWC}}
{[\chem{H_2O}] + 5\E2 [\chem{Cl^-}] + 3\E5 [\chem{Br^-}]}$$
The total uptake rate of X is only determined by $k_{\rm mt}$. The
factors only affect the branching between hydrolysis and the halide
reactions. The factor 5\E2 was chosen such that the chloride reaction
dominates over hydrolysis at about [\chem{Cl^-}] $>$~0.1~\unit{M} (see
Fig.~3 in \citet{536}), i.e.\ when the ratio [\chem{H_2O}]/[\chem{Cl^-}]
is less than 5\E2. The ratio 5\E2/3\E5 was chosen such that the
reactions with chloride and bromide are roughly equal for sea water
composition \citep{358}.

\end{multicols}

\clearpage

\begin{longtable}{llp{9cm}p{7cm}p{5cm}}
\caption{Heterogeneous reactions}\\
\hline
\# & labels & reaction & rate coefficient & reference\\
\hline
\endfirsthead
\caption{Heterogeneous reactions (... continued)}\\
\hline
\# & labels & reaction & rate coefficient & reference\\
\hline
\endhead
\hline
\endfoot
\input{mecca_eqn_het.tex}
\end{longtable}

$^*$Notes:

Heterogeneous reaction rates are calculated with an external module and
then supplied to the MECCA chemistry (see \url{www.messy-interface.org}
for details)

\clearpage

\def\aq{}
\begin{longtable}{llp{7cm}p{3cm}p{25mm}p{6cm}}
\caption{Acid-base and other eqilibria}\\
\hline
\# & labels & reaction & $K_0[M^{m-n}]$ & -$\Delta H / R [K]$ & reference\\
\hline
\endhead
\hline
\endfoot
\input{mecca_eqn_eq.tex}
\end{longtable}

\begin{multicols}{2}
$^*$Notes: 

\note{EQ40}{For $pK_a$(\chem{CO_2}), see also \citet{1777}.}

\note{EQ72}{For $pK_a$(\chem{HOBr}), see also \citet{1778}.}

\note{EQ82}{Thermodynamic calculations on the \chem{IBr}/\chem{ICl}
  equilibrium according to the data tables from \citet{489}:\\
  \begin{tabular}{ccccccc}
    \chem{ICl} & + & \chem{Br^-} & $\rightleftharpoons$ &
    \chem{IBr} & + & \chem{Cl^-}\\
    -17.1 & & -103.96 & = & -4.2 & & -131.228
  \end{tabular}
  $$\frac{\Delta G}{[\unit{kJ/mol}]} = -4.2 - 131.228 - (-17.1 - 103.96)
  = -14.368$$
  $$K = \frac{[\chem{IBr}] \times [\chem{Cl^-}]}{[\chem{ICl}] \times
    [\chem{Br^-}]} = \exp\left(\frac{-\Delta G}{RT}\right) =
  \exp\left(\frac{14368}{8.314\times 298}\right) = 330$$
  This means we have equal amounts of \chem{IBr} and \chem{ICl} when the
  [\chem{Cl^-}]/[\chem{Br^-}] ratio equals 330.}

\end{multicols}

\clearpage

\def\aq{}
\begin{longtable}{llp{8cm}p{3cm}p{25mm}p{5cm}}
\caption{Aqueous phase reactions}\\
\hline
\# & labels & reaction & $k_0~[M^{1-n}s^{-1}]$ & $-E_a / R [K] $& reference\\
\hline
\endfirsthead
\caption{Aqueous phase reactions (...continued)}\\
\hline
\# & labels & reaction & $k_0~[M^{1-n}s^{-1}]$ & $-E_a / R [K] $& reference\\
\hline
\endhead
\hline
\endfoot
% this file was created automatically by eqn2tex, do not edit!

\end{longtable}

\begin{multicols}{3}
$^*$Notes:

\note{A6102}{\citet{1008} found an upper limit of \code{6E9} and cite an
  upper limit from another study of \code{2E9}. Here, we set the rate
  coefficient to \code{1E9}.}

\note{A6301}{There is also an earlier study by \citet{69} which found a
  smaller rate coefficient but did not consider the back reaction.}

\note{A7400}{Assumed to be the same as for \chem{Br_2^-} + \chem{H_2O_2}.}

\note{A9105}{The rate coefficient for the sum of the paths (leading to
  either \chem{HSO_5^-} or \chem{SO_4^{2-}}) is from \citet{75}, the
  ratio 0.28/0.72 is from \citet{111}.}

\note{A9106}{See also: \citep{75,1}. If this reaction produces a lot of
  \chem{SO_4^-}, it will have an effect. However, we currently assume
  only the stable \chem{S_2O_8^{2-}} as product. Since
  \chem{S_2O_8^{2-}} is not treated explicitly in the mechanism, we use
  \chem{SO_4^{2-}} as a proxy. Note that this destroys the mass
  consistency for sulfur species.}

\note{A9205}{D.\ Sedlak, pers.\ comm.\ (1993).}

\note{A9208}{D.\ Sedlak, pers.\ comm.\ (1993).}

\note{A9605}{assumed to be the same as for \chem{SO_3^{2-}} +
  \chem{HOCl}.}

\note{A9705}{assumed to be the same as for \chem{SO_3^{2-}} +
  \chem{HOBr}.}

\end{multicols}

\clearpage

\begin{multicols}{3}
\bibliographystyle{egu} % bst file
\bibliography{meccalit} % bib files
\end{multicols}

\end{document}
